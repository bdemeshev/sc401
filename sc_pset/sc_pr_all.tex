\Opensolutionfile{solution_file}[all_solutions]
% в квадратных скобках фактическое имя файла

\section{Счетность множеств и пр.}

\begin{problem}
 % in sc_skelet
Какова мощность декартова произведения счетного количества счетных множеств?

\begin{sol}
Континуум, так как оно равномощно множеству последовательностей из натуральных чисел.
\end{sol}
\end{problem}

\begin{problem}
Пусть $A$ — некоторое множество. Пусть $F$ — множество функций, принимающих только значения ноль или один с областью определения $A$. Пусть $M$ — множество всех подмножеств $A$. Равномощны ли $F$ и $M$?

\begin{sol}
да
\end{sol}
\end{problem}

\begin{problem}
 An enemy submarine is somewhere on the number line (consider only integers for this problem). It is moving at some rate (again, integral units per minute). You know neither its position nor its velocity.\\
You can launch a torpedo each minute at any integer on the number line. If the the submarine is there, you hit it and it sinks. You have all the time and torpedoes you want. You must sink this enemy sub — devise a strategy that is guaranteed to eventually hit the enemy sub.

\begin{sol}
Идея: счетное количество счетных множеств, считаем змейкой
\end{sol}
\end{problem}

\begin{problem}
Пусть $A$ — множество всех подмножеств натуральных чисел, а $ B $ — множество бесконечных последовательностей из 0 и 1. Для примера: $\{5,6,178\}\in A$, $01010101010101\ldots \in B$. Сравните мощность множеств $ A $ и $ B $.

\begin{sol}
Одинаковая, биекция: $\{2,4,5\}\sim 01011000000\ldots$, т.е. единички соответствуют включенным в подмножество числам
\end{sol}
\end{problem}

\begin{problem}
Сравните мощности множеств $ A $ и $ B $, если $ A=\mathbb{Q} $ — рациональные числа, $ B=\mathbb{Q}^{2} $ — пары рациональных чисел.

\begin{sol}
Одинаковая. $ \mathbb{Q}\sim \mathbb{N}\sim \mathbb{N}^{2}\sim \mathbb{Q}^{2} $
\end{sol}
\end{problem}

\begin{problem}
Сравните мощности множеств $A$ и $B$, если $A=\mathbb{N}$, $B$ — множество бесконечных вправо последовательностей из 0 и 1.

\begin{sol}
$A<B$
\end{sol}
\end{problem}

\begin{problem}
Сравните мощности множеств $A$ и $B$, если $A=\mathbb{N}$, $B$ — множество подмножеств натуральных чисел.

\begin{sol}
$A<B$
\end{sol}
\end{problem}

\begin{problem}
Назовём две бесконечных вправо последовательности из нулей и единиц «похожими», если они отличаются на конечное количество членов. Это отношение «похожести» разбивает все последовательности на классы похожих последовательностей.

\begin{enumerate}
\item Какова мощность множества последовательностей похожих на последовательность из одних нулей?
\item Какова мощность множества классов похожих последовательностей?
\end{enumerate}

\begin{sol}
Множество последовательностей похожих на любую заданную счётно. Мощность множества классов похожих последовательностей — континуум.
\end{sol}
\end{problem}

\begin{problem}
Злобный Дракон поймал бесконечное счётное количество гномов. Расставил их в шеренгу так, что первый видит всех остальных, второй — всех, начиная с третьего гнома, и т.д. Далее Дракон надевает каждому гному либо чёрный, либо белый колпак. Гномы одновременно пытаются угадать цвет своего колпака. Гномы, не угадавшие цвет своего колпака, съедаются Драконом. Есть ли у гномов стратегия, позволяющая им иметь конечные боевые потери при встрече со Злобным Драконом?

\begin{sol}
Назовём две последовательности из чёрных и белых колпаков «похожими», если они отличаются на конечное количество колпаков. Гномы могут заранее в каждом классе похожих последовательностей выбрать одну «представительскую» последовательность. Глядя на колпаки впереди стоящих гномов, можно идентифицировать класс, которому принадлежит фактическая последовательность колпаков. Далее гномы гадают согласно «представительской» последовательности.
\end{sol}
\end{problem}


\begin{problem}
В отеле бесконечное счётное количество номеров. Все номера заняты туристами.
\begin{enumerate}
\item Приехал ещё один турист. Как переразместить всех постояльцев, чтобы всем хватило места?
\item Приехало ещё счётное количество туристов. Как переразместить всех постояльцев, чтобы всем хватило места?
\end{enumerate}

\begin{sol}
Например, переселить всех текущих постояльцев в чётные номера. Останутся свободными все нечётные, их хватит.
\end{sol}
\end{problem}


\begin{problem}
Дед Мороз пришёл к детишкам на Новый Год с мешком, 
в котором счётное количество пронумерованных конфет. 

Конфеты можно есть только после наступления Нового Года. 
Ровно за часа до Нового Года Дед Мороз выдаёт детишкам конфеты номер 1 и 2 и тут же забирает конфету номер 1 обратно. 
Ровно за пол-часа — выдаёт конфеты номер 3 и 4 и забирает конфету номер 2. 
Ровно за четверть часа — выдаёт конфеты номер 5 и 6 и забирает конфету номер 3. 
И так далее, ускоряясь, выдаёт из мешка две очередные конфеты и забирает у детишек конфету с наименьшим номером.

\begin{enumerate}
\item На сколько изменяется количество конфет у детишек за одну операцию дарения-забирания?
\item У кого к Новому Году окажется конфета номер 1?
\item У кого к Новому Году окажется конфета номер 2015?
\item Сколько конфет будет у детишек к Новому Году?
\end{enumerate}


\begin{sol}
К Новому Году все конфеты окажутся у Деда Мороза.
\end{sol}
\end{problem}


\begin{problem}
Злой султан зашёл в темницу к трём узникам и объявил: 

— Завтра вам каждому напишут на лбу целое число и всех усадят за круглый стол. 
Разговаривать и подавать знаки нельзя, можете только посмотреть на числа соседей.
Затем каждый может написать на листочке конечное количество целых чисел и передать список мне. 

Если хотя бы у одного в списке окажется число, которое у него на лбу — всех помилую.

\begin{enumerate}
  \item Могут ли узники придумать стратегию, чтобы гарантированно спастись?
  \item Через минуту Злой султан вернулся к узникам и добавил: 
  «Совсем забыл сказать, один из поданных вами листочков со списком чисел я сожгу перед возможным помилованием».

  Смогут ли теперь узники гарантированно спастись?
  \item Ещё через пять минут в темницу заглянула Прекрасная дочь Злого султана и сказала:
  «Мой отец обманул вас: он решил написать не целые числа, а действительные, но может быть кто-то из вас помнит аксиому выбора?». 

  И Прекрасная дочь скрылась. Смогут ли придумать узники что-то на этот раз?
\end{enumerate}


\begin{sol}
  \begin{enumerate}
    \item Например, каждый может назвать число от минимума до максимума тех, что видит. 
    \item Например, каждый может назвать число от минус наибольшего по модулю до плюс наибольшего по модулю. 
    \item 
  \end{enumerate}
\end{sol}
\end{problem}


\section{Сигма-алгебры и измеримость}

\begin{problem}
% in sc_scelet
Игральный кубик подбрасывается один раз. В первый момент времени
наблюдатель узнает, выпала ли четная грань или нет. Во второй
момент времени наблюдатель дополнительно узнает, выпала ли грань больше двух
или нет. В третий момент времени наблюдатель точно узнает, какая
грань выпала. Укажите множество элементарных событий и
соответствующие три алгебры событий.


\begin{sol}


\end{sol}
\end{problem}

\begin{problem}
 Почему алгебра? \\
Значки некоторых операций над множествами ($\cap$, $\cup$,
$\setminus$, $\triangle$, etc) были заменены на привычные значки
операций над числами ($+$, $-$, $\cdot$) так, что все условия выполнены:
\begin{enumerate}
\item $A\cdot(B+C)=A\cdot B+A\cdot C$ \\
\item $A\cdot(B-C)=A\cdot B-A\cdot C$ \\
\item Уравнение $A+X=B$ всегда имеет единственное решение $X=B-A$ \\
\item Существуют такие $1$ и $0$, такие, что: \\
\begin{enumerate}
\item $1\cdot A=A$ \\
\item $0\cdot A=0$ \\
\item $A+0=A$ \\
\item $A-0=A$
\end{enumerate}
\end{enumerate}
\begin{enumerate}
\item Какие операции над множествами скрыты значками $+$, $-$,
$\cdot$?
\item Какие множества подразумеваются под $1$ и $0$?
\item Упростите в этой алгебре выражения $(A+B)(A-B)+A$, $1+1$.
\end{enumerate}
Комментарий: к сожалению, деление в этой алгебре отсутствует,
именно поэтому термин поле (field) некорректен.

\begin{sol}
$+$ и $-$ — это симметрическая разность, $1$ — это $\Omega$, $0$ — $\emptyset$, $(A+B)(A-B)+A=B$, $1+1=0$
\end{sol}
\end{problem}

\begin{problem}
Приведите пример алгебры, которая не является $\sigma$-алгеброй.

\begin{sol}
Пример. $\Omega=[0;1]$, $\mathcal{A}$ состоит из конечных множеств и множеств с конечными дополнениями.
\end{sol}
\end{problem}

\begin{problem}
Сколько $\sigma$-алгебр можно создать на множестве из 3-х элементов? из 4-х элементов?

\begin{sol}
столько же, сколько разбиений, 5 сигма-алгебр для 3-х элементов и 15 сигма-алгебр для 4-х элементов, числа Белла
\end{sol}
\end{problem}

\begin{problem}
Определим $\liminf_{n}
A_{n}=\bigcup_{n=1}^{+\infty}\bigcap_{i=n}^{+\infty}A_{i}$ и
$\limsup_{n}
A_{n}=\bigcap_{n=1}^{+\infty}\bigcup_{i=n}^{+\infty}A_{i}$ \\
Смысл: \\
$\liminf_{n} A_{n}$ — произошли все $A_{n}$, кроме быть может
конечного числа \\
$\limsup_{n} A_{n}$ — произошло бесконечное количество $A_{n}$ \\
Докажите, что $1_{\liminf_{n}A_{n}}=\liminf_{n}1_{A_{n}}$ и
$1_{\limsup_{n}A_{n}}=\limsup_{n}1_{A_{n}}$

\begin{sol}

\end{sol}
\end{problem}

\begin{problem}
Пусть $A$ и $B$ — события. Определим $C_{n}=
\begin{cases}
  A, & n=2k \\
  B, & n=2k+1 \\
\end{cases}$. \\
Найдите $\limsup_{n}C_{n}$ и $\liminf_{n}C_{n}$.

\begin{sol}
$\limsup_{n}C_{n}=A\cup B$, $\liminf_{n}C_{n}=A\cap B$.
\end{sol}
\end{problem}

\begin{problem}
Пусть $A_{1} \subseteq A_{2} \subseteq A_{3} \subseteq \ldots$.
Найдите $\liminf_{n}A_{n}$ и $\limsup_{n}A_{n}$. \\
Пусть $B_{1} \supseteq B_{2} \supseteq B_{3} \supseteq \ldots$.
Найдите $\liminf_{n}B_{n}$ и $\limsup_{n}B_{n}$.


\begin{sol}

\end{sol}
\end{problem}

\begin{problem}
Докажите, что $\limsup_{n}A_{n}=\limsup_{n}A_{n+k}$ и
$\liminf_{n}A_{n}=\liminf_{n}A_{n+k}$ при любом $k\in\mathbb{N}$.
Смысл этого утверждения следующий: при вычислении предела
последовательности можно проигнорировать миллион-другой членов в
ее начале.

\begin{sol}

\end{sol}
\end{problem}

\begin{problem}
Пусть $A_{n} \subseteq B_{n}$ для всех $n\ge N$. Сравните множества: $\liminf_{n}A_{n}$ и $\liminf_{n}B_{n}$;
$\limsup_{n}A_{n}$ и $\limsup_{n}B_{n}$.

\begin{sol}
$\liminf_{n}A_{n} \subseteq \liminf_{n}B_{n}$ и $\limsup_{n}A_{n}
\subseteq \limsup_{n}B_{n}$.
\end{sol}
\end{problem}

\begin{problem}
Верно ли, что $\left(\limsup_{n}A_{n}\right)^{c}=\liminf_{n}A_{n}^{c}$?

\begin{sol}
да
\end{sol}
\end{problem}

\begin{problem}
Разложите на множители $1_{A\cap B}$ и $1_{A\backslash B}$ \\
Найдите $1_{A}+1_{A^{c}}$

\begin{sol}
 $1_{A\cap B}=1_{A}1_{B}$, $1_{A\backslash B}=1_{A}1_{B^{c}}$, $1_{A}+1_{A^{c}}=1$
\end{sol}
\end{problem}

\begin{problem}
Пусть $\mathcal{F}$ — $\sigma$-алгебра подмножеств $\Omega$, и
$B\subseteq\Omega$. Верно ли, что набор множеств
$\mathcal{H}=\{A:A\subseteq B$ или $B^{c}\subseteq A\}$ является
$\sigma$-алгеброй на $\Omega$?

\begin{sol}
да, удобно изобразить $\Omega$ отрезком, «пескари» $A\subseteq B$ и «акулы» $A \supseteq B^{c}$
\end{sol}
\end{problem}

\begin{problem}
Пусть $\mathcal{F}_{i}$, $i\in I$ — $\sigma$-алгебры подмножеств
$\Omega$. \\
\begin{enumerate}
\item Докажите, что $\cap_{i\in I}\mathcal{F}_{i}$ -
$\sigma$-алгебра. \\
\item Приведите пример, показывающий, что
$\mathcal{F}_{1}\cup\mathcal{F}_{2}$ — не обязательно алгебра.
\end{enumerate}

\begin{sol}
 б) $\Omega=\{a,b,c\}$, $\mathcal{F}_{1}=\{\emptyset,\Omega,a,bc\}$, $\mathcal{F}_{2}=\{\emptyset,\Omega,ab,c\}$
\end{sol}
\end{problem}

\begin{problem}
Решеткой будем обозначать количество элементов множества ('\#A').\\
Пусть $A \subseteq \mathbb{N}$. Определим
$\gamma(A)=\lim_{n\rightarrow \infty}\frac{\#(A\bigcap
\{1,2,3,\ldots,n\})}{n}$ в тех случаях, когда этот предел существует.
Величина $\gamma(A)$ (Cesaro density) показывает, какую «долю» от
всех натуральных чисел составляет указанное подмножество. Пусть
$\mathcal{H}=\{A\subseteq \mathbb{N}|\exists \gamma(A)\}$.

\begin{enumerate}
\item Приведите пример множества, у которого не определена доля Чезаро.
\item Верно ли, что у натуральных чисел, в записи которых присутствует
хотя бы одна единица, есть доля? Если да, то чему она равна?
\item Верно ли, что у натуральных чисел, в записи которых присутствует
ровно одна единица, есть доля? Если да, то чему она равна?
\item Верно
ли, что $\mathcal{H}$ -
алгебра?
\end{enumerate}


\begin{sol}

Решение: разобьем натуральный ряд на пары соседних чисел. Можно
так подобрать множества $A$ и $B$, что в каждом из них из каждой
пары взято только одно число. Поэтому
$\gamma(A)=\gamma(B)=\frac{1}{2}$. Подобрав
совпадение-несовпадение в паре, можно сделать так, что
$\gamma(A\cap B)$ не существует.
\end{sol}
\end{problem}

\begin{problem}
Пусть $\mathcal{M}$ это $\pi$-система на $\Omega$. Для любого
множества $A$ из $\lambda(\mathcal{M})$ определим набор
$F(A)=\{L|L\cap A \in \lambda(\mathcal{M})\}$. Т.е. $F(A)$ — это
набор «хороших» относительно $A$ множеств, их $A$ не отбрасывает
за пределы $\lambda(\mathcal{M})$. \\
Последовательно докажите, что:
\begin{enumerate}
\item Если $A \in \mathcal{M}$, то $\mathcal{M}\subseteq F(A)$.
\item Если $A\in\lambda(\mathcal{M})$, то $F(A)$ — $\lambda$-система.
\item Если $A \in \mathcal{M}$, то $\lambda(\mathcal{M})\subseteq
F(A)$.
\item Если $B\in \lambda(\mathcal{M})$, то $\mathcal{M}\subseteq
F(B)$.
\item Если $B\in \lambda(\mathcal{M})$, то
$\lambda(\mathcal{M})\subseteq
F(B)$.
\item $\lambda(\mathcal{M})$ является $\pi$-системой.
\end{enumerate}

\begin{sol}

\end{sol}
\end{problem}

\begin{problem}
Чтобы набор $\mathcal{M}$ являлся $\sigma$-алгеброй необходимо и
достаточно, чтобы он одновременно являлся $\pi$-системой и
$\lambda$-системой.

\begin{sol}

\end{sol}
\end{problem}

\begin{problem}
Если $\mathcal{M}$ это $\pi$-система на $\Omega$, то
$\lambda(\mathcal{M})=\sigma(\mathcal{M})$

\begin{sol}

\end{sol}
\end{problem}

\begin{problem}
(Александр ?) Рассмотрим набор подмножеств
$\mathcal{H}=\{[0;\frac{1}{n}]|n\in \mathbb{N}\}$. Пусть
$\mathcal{A}$ — минимальная алгебра, порождаемая этим набором, а
$\mathcal{F}$ — минимальная
порождаемая $\sigma$-алгебра.
\begin{enumerate}
\item Опишите все множества, входящие в $\mathcal{A}$
\item  Опишите все множества, входящие в $\mathcal{F}$
\item  Верно ли, что $\mathcal{A}$ и $\mathcal{F}$ совпадают?
\item Совпадают ли мощности $\mathcal{A}$ и $\mathcal{F}$?
\end{enumerate}


\begin{sol}

$\mathcal{F}$ имеет мощность континуум, так как равномощно множеству последовательностей из нулей и единиц

$\mathcal{A}$ счетно, так как равномощно множеству последовательностей из нулей и единиц, где конечное количество нулей или конечное количество единиц

\end{sol}
\end{problem}

\begin{problem}
Пусть $\Omega$ — произвольное множество,
$\mathcal{F}=\{\emptyset,\Omega\}$ и $f:\Omega\rightarrow
\mathbb{R}$. При
каком условии функция $f$ будет измеримой?

\begin{sol}
$f$ — константа
\end{sol}
\end{problem}

\begin{problem}
Пусть $X$ — случайная величина. Верно ли, что
$\sigma(X)=\{X^{-1}(B)|B \in \mathcal{B} \}$?

\begin{sol}
да
\end{sol}
\end{problem}

\begin{problem}
При каком соотношении на $n$ и $\lambda$ для любой случайной
величины $X$ справедливо равенство $\{X\ge \lambda\}=\{X\wedge
n\ge\lambda\}$?

\begin{sol}

\end{sol}
\end{problem}

\begin{problem}
Пусть $X$ — случайная величина, а $f:\mathbb{R}\rightarrow
\mathbb{R}$ — измеримая функция.
Верно ли, что $f(X)$ — является $\sigma(X)$-измеримой?

\begin{sol}
да
\end{sol}
\end{problem}

\begin{problem}
Верно ли, что любая непрерывная действительная функция является
измеримой по Борелю?

\begin{sol}
да
\end{sol}
\end{problem}

\begin{problem}
Обозначим $x \wedge y = min\{x,y\}$. Верно ли, что $1_{A} \wedge
1_{B}=1_{A\cap B}$? \\
Обозначим $x \vee y = max\{x,y\}$. Верно ли, что $1_{A} \vee
1_{B}=1_{A\cup B}$?

\begin{sol}

Answer: yes, yes
\end{sol}
\end{problem}

\begin{problem}
Пусть $X_{i}$ — случайные величины и $N$ — случайная величина,
принимающая натуральные значения. Верно ли, что
$\sum_{i=1}^{N}X_{i}$ — случайная величина?

\begin{sol}

\end{sol}
\end{problem}

\begin{problem}
 Правильный кубик подбрасывается один раз. $Y$ — индикатор того, выпала ли четная грань. $Z$ — индикатор того, выпало ли число больше 2-х. \\
\begin{enumerate}
\item Сколько элементов $\sigma(Z)$?
\item Сколько элементов в $\sigma(Y\cdot Z)$?
\item Сколько элементов в $\sigma(Y,Z)$?
\item Как связаны между собой $ \sigma $-алгебры $\sigma(Y\cdot Z)$ и $\sigma(Y,Z)$?
\end{enumerate}


\begin{sol}

\end{sol}
\end{problem}

\begin{problem}
Пусть $\Omega=[0;1]$. $\mathcal{F}=\{A\subseteq$ $\Omega|$ либо $A$ не
более чем счетно, либо $A^{c}$ не более чем счетно $\}$. \\
Верно ли, что $\mathcal{F}$ — $\sigma$-алгебра? \\
Придумайте $B\subset\Omega$, такое что $B \notin\mathcal{F}$.

\begin{sol}
Например, $B$ — Канторово множество, или (гораздо проще)
$B=[0;0,5]$. Оно само более чем счетно и дополнение к нему более
чем счетно. \\
$\mathcal{F}$ действительно $\sigma$-алгебра. \\
$\emptyset$ лежит в $\mathcal{F}$ т.к. имеет ноль элементов (а ноль,
очевидно, не более чем счетное число). \\
Если $A$ не более чем счетно, то $A^{c}$ лежит в $\mathcal{F}$
т.к. дополнение к $A^{c}$ содержит не более чем счетное
число элементов. \\
Если дополнение к $A$ не более чем счетно, то $A^{c}$ лежит
в $\mathcal{F}$ т.к. содержит не более чем счетное число элементов. \\
Проверяем счетное объединение $\bigcup_{i} A_{i}$. \\
Если среди $A_{i}$ встречаются только не более чем счетные, то и
их объединение — не более чем счетно. \\
Если среди $A_{i}$ встретилось хотя бы одно множество с не более
чем счетным дополнением, то $\bigcup_{i} A_{i}$ тоже будет
обладать не более чем счетным дополнением, так как объединение не
может быть меньше ни одного из объединяемых множеств.
\end{sol}
\end{problem}

\begin{problem}
Случайные величины $X$ и $Y$ заданы на $(\Omega,\mathcal{F})$. Верно ли, что $A=\{w|X(w)=Y(w)\}$ лежит в $\sigma$-алгебре $\mathcal{F}$?

\begin{sol}

Введём новую функцию $Z=X-Y$. Тогда интересующее нас множество -
это прообраз нуля $A=Z^{-1}(0)$. Число ноль — борелевское
множество. Докажем более сильное утверждение, что $Z$ — это
случайная величина, т.е. прообраз любого борелевского множества
измерим.
По доказанной в прошлый раз теореме, достаточно проверить, что
прообраз любого множества вида $(-\infty;z)$ измерим. \\
Пусть $r_{x}$ и $r_{y}$ — два рациональных числа. Множества
$A_{r_{x}}=X^{-1}((-\infty;r_{x}))$ и
$C_{r_{y}}=Y^{-1}((r_{y};+\infty))$ измеримы. \\
Следовательно, $D_{r_{x},r_{y}}=A_{r_{x}} \bigcap C_{r_{y}}$ тоже
измеримо. \\
Немного помедитировав замечаем, что $Z^{-1}(-\infty;z)=\bigcup_{r_{x}-r_{y}<z} D_{r_{x},r_{y}}$
\end{sol}
\end{problem}

\begin{problem}
We toss a fair coin infinitely many times. So the set of all oucomes $\Omega$ is the set of all infinite sequences of H and T. Let $A_{n}$ be the event that the coin shows head in the $n$-th toss. Let $\mathcal{F}=\sigma(A_{1},A_{2},A_{3},\ldots)$ — be the minimal $\sigma$-algebra containing all the $A_{n}$. \\
Let $X_{n}$ be the proportion of heads after the first $n$ tosses.
\begin{enumerate}
\item Show that the event $X_{n}<t$ is contained in $\mathcal{F}$ for all $n$ and $t$. \\
\item Express the event $lim_{n\to\infty}X_{n}\le 2/5$ using countable operations with the events of the form $X_{n}<t$.
\end{enumerate}

\begin{sol}

\end{sol}
\end{problem}

\begin{problem}
Пусть $f:\mathbb{R}\to\mathbb{R}$ — дифференцируемая функция. Верно ли, что $f'$ — измерима по Борелю?

\begin{sol}
Да. $ f' $ — непрерывна, а следовательно, измерима.
\end{sol}
\end{problem}

\begin{problem}
Пусть $\Omega=[0;1]$, $ \mathcal{F}=\mathcal{B}[0;1] $. Случайный процесс $X_{t}(w):=1_{t=w}$. Постройте фильтрацию, порождаемую этим процессом.

\begin{sol}
Заметим, что $\sigma(X_{t})=\{\emptyset,\Omega,\{t\},\Omega\backslash\{t\}\}$. Значит в $ \mathcal{F}_{t} $ является минимальной $\sigma  $-алгеброй порожденной набором одноточечных множеств $\{x|x\in [0;t]\}$. Т.е. $\mathcal{F}_{t}$ — это все счетные подмножества отрезка $[0;t]$ плюс все их дополнения.
\end{sol}
\end{problem}

\begin{problem}
В лесу есть три вида грибов: рыжики, лисички и мухоморы. Попадаются они равновероятно и независимо друг от друга. Маша нашла 100 грибов. Пусть $R$ — количество рыжиков, $L$ — количество лисичек, а $M$ — количество мухоморов среди найденных грибов.
\begin{enumerate}
\item Сколько элементов $ \sigma(R)$?
\item Сколько элементов $ \sigma(R,M)$?
\item Измерима ли $L$ относительно $ \sigma(R)$?
\item Измерима ли $L$ относительно $ \sigma(R,M)$?
\item Измерима ли $L$ относительно $ \sigma(R+M)$?
\item Измерима ли $L$ относительно $ \sigma(R-M)$?
\end{enumerate}

\begin{sol}
$ 2^{101} $, $2^{101\cdot 51}$, нет, да
\end{sol}
\end{problem}

\begin{problem}
Сначала подбрасываем кубик. Если выпала нечетная грань, то подбрасываем правильную монетку один раз. 
Если четная — то два. Пусть $X$ — количество подбрасываний монетки, а $Y$ — количество выпавших орлов.

\begin{enumerate}

\item Сколько элементов в $ \sigma(X)$?
\item Сколько элементов в $ \sigma(Y)$?
\item Сколько элементов в $ \sigma(X,Y)$?
\item Измерима ли $ X $ относительно $ \sigma(Y)$?
\item Измерима ли $ Y $ относительно $ \sigma(X)$?

\end{enumerate}

\begin{sol}
 $ 2^{2} $, $ 2^{3} $, $ 2^{5} $, нет, нет
\end{sol}
\end{problem}

\begin{problem}
Пусть $X_{n}$ — независимые случайные величины равновероятно равные плюс или минус единице. $S_{n}=X_{1}+X_{2}+\ldots+X_{n}$. Сравните $\sigma$-алгебры $ \sigma(X_{1},X_{2},\ldots X_{n}) $ и $ \sigma(S_{1},S_{2},\ldots S_{n}) $.

\begin{sol}
Они совпадают
\end{sol}
\end{problem}

\begin{problem}
Монетка подкидывается бесконечное количество раз: $X_{n}$ равно 1, если при $ n $-ом подбрасывании выпадает орел и 0, если решка. 
Определим кучу $ \sigma $-алгебр: $\mathcal{F}_{n}:=\sigma(X_{1},X_{2},\ldots,X_{n})$, $\mathcal{H}_{n}:=\sigma(X_{n},X_{n+1},X_{n+2},\ldots)$.

Приведите по два нетривиальных (отличных от $\Omega$ и $\emptyset$) примера такого события $A$, что:

\begin{enumerate}
\item $ A\in \mathcal{F}_{2010} $
\item $ A\notin \mathcal{F}_{2010} $
\item $A$ лежит в каждой $\mathcal{H}_{n}$
\end{enumerate}

В какие из упомянутых $ \sigma $-алгебр входят события:
\begin{enumerate}
\item $ X_{37}>0$
\item $ X_{37}>X_{2010}$
\item $ X_{37}>X_{2010}>X_{12}$
\end{enumerate}

Упростите выражения: $ \mathcal{F}_{11}\cap \mathcal{F}_{25} $, $ \mathcal{F}_{11}\cup \mathcal{F}_{25} $, $ \mathcal{H}_{11}\cap \mathcal{H}_{25} $, $ \mathcal{H}_{11}\cup \mathcal{H}_{25} $

\begin{sol}

\end{sol}
\end{problem}

\begin{problem}
Может ли в $ \sigma $-алгебре быть ровно 2010 элементов?

\begin{sol}
Нет. В конечной только $2^n$, где $n$ — количество элементов в разбиении $\Omega$
\end{sol}
\end{problem}

\begin{problem}
Пусть $X$ — равномерная на $ [0;1] $ случайная величина. Пусть $ \mathcal{H}_{1} $ — минимальная $ \sigma $-алгебра, содержащая все события вида $ X=t $, а $ \mathcal{H}_{2} $ — минимальная $ \sigma $-алгебра, содержащая все события вида $X<t$. Сравните $\sigma$-алгебры $ \mathcal{H}_{1} $ и $ \mathcal{H}_{2} $.

\begin{sol}
$ \mathcal{H}_{1} \subset \mathcal{H}_{2} $
\end{sol}
\end{problem}

\begin{problem}
Правильная монетка подбрасывается бесконечное количество раз. Вася наблюдает за результатами подбрасываний до тех пор, пока не выпадет 3 орла подряд. Пусть $T$ — случайный момент времени, когда Вася прекратил наблюдения, и $ \mathcal{F}_{T}$ — $\sigma$-алгебра событий различимых Васей. Приведите пример двух нетривиальных (отличных от $ \Omega$ и  $\emptyset$) событий, входящих в $\mathcal{F}_T$.

\begin{sol}

\end{sol}
\end{problem}

\begin{problem}
Let $S_{n}$ be a symmetric random walk, i.e. $ S_{n}=X_{1}+X_{2}+\ldots+X_{n} $ where $ X_{n} $ are independent and identically distributed with $ \P(X_{n}=1)=\P(X_{n}=-1)=0.5 $. Let $T$ be the time of the second local maximum plus one. For example: if the sequence of $ X_{n} $ is given by $ +1,+1,-1,-1,-1,+1,-1,-1,+1,\ldots$ then $T=7$. We define the following sigma-algebras: $ \mathcal{F}_{n}:=\sigma(X_{1},X_{2},\ldots,X_{n})$ and $ \mathcal{F}_{T} $ — all events which may be distninguished by a rational agent who knows all $ X_{i} $ up to time $ T $.
\begin{enumerate}
\item Give an example of event $A$ such that $ A\notin \mathcal{F}_{T} $ but $ A\in \mathcal{F}_{2010} $
\item Give an example of event $B$ such that $ B\in \mathcal{F}_{T} $ but $ B\notin \mathcal{F}_{2010} $
\item Prove that $ \mathcal{F}_{T} $ is different from each $ \mathcal{F}_{n} $
\item Does $T=5$ belongs to $\cF_T$? Does $T=5$ belongs to $\sigma(T)$?
\item Does $\{X_1 = 1, X_2 = -1, X_3 = 1, X_4 = -1 \}$ belongs to $\cF_T$? To $\sigma(T)$?
\item Does $\{X_1 = 1, X_2 = -1, X_3 = 1, X_4 = -1, X_5 = 1 \}$ belongs to $\cF_T$? To $\sigma(T)$?
\item Are sigma-algebras $\cF_T$ and $\sigma(T)$ equal?
\end{enumerate}


\begin{sol}
  \begin{enumerate}
    \item
    \item
    \item
    \item
    \item yes, yes
    \item no, no
    \item no, consider $\{X_1 = 1, X_2 = -1, X_3 = 1, X_4 = 1, X_5 = -1 \}$
  \end{enumerate}
\end{sol}
\end{problem}

\begin{problem}
Монетка подбрасывается бесконечное количество раз. $ X_{n} $ — результат $ n $-го подбрасывания, 1, если решка и 0, если орел. Let $T$ be the first moment of time when the proportion of head is greater than 0.5. We define the following sigma-algebras: $ \mathcal{F}_{n}:=\sigma(X_{1},X_{2},\ldots,X_{n})$ and $ \mathcal{F}_{T} $ — all events which may be distninguished for sure by a rational agent who knows all $ X_{i} $ up to time $ T $.
\begin{enumerate}
\item[a.] Give an example of event $A$ such that $ A\notin \mathcal{F}_{T} $ but $ A\in \mathcal{F}_{2010} $
\item[b.] Give an example of event $B$ such that $ B\in \mathcal{F}_{T} $ but $ B\notin \mathcal{F}_{2010} $
\item[c.] Prove that $ \mathcal{F}_{T} $ is different from each $ \mathcal{F}_{n} $
\end{enumerate}


\begin{sol}

\end{sol}
\end{problem}

\begin{problem}
Опишите $\sigma$-алгебру подмножеств отрезка $[0;1]$, порождённую отрезками $[0;0.1)$ и $(0.5,1]$.

\begin{sol}
Содержит 8 элементов, порождается разбиением.
\end{sol}
\end{problem}

\begin{problem}
Случайная величина $X$ равномерна на отрезке $[0;1]$. 
Величина $Y$ определяется по формуле $Y=1_{X<0.5}+1_{X>0.3}$. 
Опишите $\sigma$-алгебру $\sigma(Y)$.

\begin{sol}
Содержит 4 элемента, порождается разбиением.
\end{sol}
\end{problem}


\begin{problem}
Сейчас либо солнечно, либо дождь, либо пасмурно без дождя. 
Соответственно, множество $\Omega$ состоит из трёх исходов, $\Omega=\{\text{солнечно},\text{дождь},\text{пасмурно}\}$. 
Джеймс Бонд пойман и привязан к стулу с завязанными глазами, но он может на слух отличать, идёт ли дождь.
\begin{enumerate}
\item Как выглядит $\sigma$-алгебра событий, которые различает агент 007?
\item Как выглядит минимальная $\sigma$-алгебра, содержащая событие $A=\{не видно солнце\}$?
\item Сколько различных $\sigma$-алгебр можно придумать для данного $\Omega$?
\end{enumerate}


\begin{sol}
$\cF=\{\emptyset, \Omega, \{\text{дождь}\}, \{\text{солнечно}, \text{пасмурно}\}$. Всего есть $1+1+3=5$ $\sigma$-алгебр.
\end{sol}
\end{problem}

\begin{problem}
Возможно четыре исхода, $\Omega=\{a,b,c,d\}$. Величины $X$ и $Y$ принимают значения $0$ и $1$:

\begin{tabular}{c|cc}
 & $Y=0$ & $Y=1$ \\
\hline
$X=0$ & $a$ & $b$ \\
$X=1$ & $c$ & $d$ \\
\end{tabular}

Найдите $\sigma$-алгебры $\sigma(X)$, $\sigma(Y)$, $\sigma(X,Y)$, $\sigma(X+Y)$, $\sigma(X-Y)$



\begin{sol}

\end{sol}
\end{problem}



\section{Модель в дискретном времени, дерево}

\begin{problem}
 Consider a stock which price today is \$50. Suppose that over the next
  year, the stock price can either go up by 10\%, or down by 3\%, so the
  stock price at the end of the year is either \$55 or \$48.50. The
  interest rate on a \$1 bond is 6\%. If there also exists a call on the
  stock with an exercise price of \$50, then what is the price of the
  call option? Also, what is the replicating portfolio?

\begin{sol}

\end{sol}
\end{problem}

\begin{problem}
 A stock price is currently \$50.  It is known that at the end of 6
  months, it will either be \$60 or \$42.  The risk-free rate of
  interest with continuous compounding on a \$1 bond is 12\% per annum.
  Calculate the value of a 6-month European call option on the stock
  with strike price \$48 and find the replicating portfolio.

\begin{sol}

\end{sol}
\end{problem}

\begin{problem}
 (Dividend-paying stock) \\
Consider the binomial asset pricing model
with the modification described below. Let
\[
Y_{n+1}(\om_1\dots\om_n\om_{n+1})=
\begin{cases}
u,&\text{if }\om_{n+1}=H,\\
d,&\text{if }\om_{n+1}=T.
\end{cases}
\]
\ni $Y_{n+1}$ depends only on the $(n+1)$-th coin toss and $Y_{n+1}S_n$ equaled to the
stock price at time $n+1$. For the \emph{dividend-paying model} consider another random
variable $A_{n+1}(\om_1\dots\om_n\om_{n+1})$ taking values in $(0,1)$ and define
the \emph{dividend paid} at time $n+1$ as the amount $A_{n+1}Y_{n+1}S_n.$ After the dividend
is paid, the stock price at time $n+1$ is
\[
S_{n+1}=(1-A_{n+1})Y_{n+1}S_n.
\]
\ni An agent who begins with initial capital $X_0$ and at each time takes a position of
$\Dt_n$ shares of stock, where $\Dt_n$ depends only on the first $n$ coin tosses, has
a portfolio value governed by the wealth equation:
\begin{align}
X_{n+1}&=\Dt_n S_{n+1}+(1+r) (X_n-\Dt_n S_n) +\Dt_n A_{n+1} Y_{n+1} S_n \\
&=\Dt_n Y_{n+1} S_n+(1+r)(X_n-\Dt_n S_n).
\end{align}
\begin{enumerate}
\item[(i)] Show that such a discounted wealth process is a martingale under the
risk-neutral measure, which is defined as in the case of usual binomial asset pricing model
(prove it as well!!!).

\item[(ii)] Show that the risk-neutral pricing formulastill applies.

\item[(iii)] Show that the discounted stock price is not a martingale under the
risk-neutral measure. However, if $A_{n+1}$ is a constant $a\in(0,1)$, regardless of the value of
$n$  and the outcome of the coin tossing $\om_1\dots\om_{n+1},$ then $\frac{S_n}{(1-a)^n(1+r)^n}$ is
a martingale under the risk-neutral measure.
\end{enumerate}

\begin{sol}

\end{sol}
\end{problem}

\begin{problem}
 Chooser option \\
 Let $1\le m\le N-1$ and $K>0$ be given
A \emph{chooser option} is a contract sold at time zero that confers on its owner
the right to receive either a call or a put at time $m$. The owner of the option may
wait until time $m$ before choosing. The call or put chosen expires at time $N$ with
the strike price $K$. Show that the time-zero price of a chooser option is the sum of
the time-zero price of a put, expiring at time N and having strike price $K$, and a call,
expiring at time $m$ and having strike price $\frac{K}{(1+r)^{N-m}}.$

\begin{sol}

\end{sol}
\end{problem}

\begin{problem}
Asian option. Consider an $N$-period binomial model.
An \emph{Asian option} has a payoff based on the average stock price, i.e.,
\[
V_N=f\left(\frac{1}{N+1}\sum\lims_{n=0}^{N} S_n \right),
\]
\ni where the function $f$ is determined by the contractual details of the option.
\begin{enumerate}
\item[(i)] Let $Y_n=\sum\lims_{k=0}^{n} S_k$. Show that the two-dimensional process
$(S_n,Y_n),\, n=0,1,\dots,N$ is Markov.

\item[(ii)] Show that the price $V_n$ of the Asian option at time $n$
is some function $v_n$ of $S_n$ and $Y_n$; i.e.
\[
V_n=v_n(S_n,Y_n),\, n=0,1,\dots,N.
\]
\ni Give a formula for $v_N(s,y)$ and provide an algorithm for backward computing
$v_n(s,y)$ in terms of $V_{N+1}.$
\end{enumerate}


\begin{sol}

\end{sol}
\end{problem}

\begin{problem}
 Asian option continued) Consider an $N$-period binomial model,
and let $M$ be a fixed number between $0$ and $N-1$. Consider an Asian option with the payoff
at time $N$ defined as
\[
V_N=f\left(\frac{1}{N_M}\sum\lims_{n=M+1}^{N} S_n \right),
\]
\ni where again the function $f$ is determined by the contractual details of the option.

Define
\[
Y_n=
\begin{cases}
0,&\text{if }0\le n\le M\\
\sum\lims_{k=M+1}^{n} S_k,&\text{if }M+1\le n\le N.
\end{cases}
\]
\ni Show that the two-dimensional process $(S_n,Y_n)$ is Markov
(under the risk-neutral measure $\tilde{\P}$).

As we already know the price $V_n$ of the Asian option at time $n$
is some function $v_n$ of $S_n$ and $Y_n$; i.e.
\[
V_n=v_n(S_n,Y_n),\, n=0,1,\dots,N.
\]
\ni Of course, when $n\le M,$ $Y_n$ is not random and does not need to be included
in this function. Thus, for such $n$ we should seek a function $v_n$ of only $S_n$.
\ni Give a formula for $v_N(s,y)$ and provide an algorithm for backward computing
$v_n(s,y)$ in terms of $V_{N+1}.$ Note that the algorithm is different for $n<M$
and $n>M,$ and there is a separate transition formula for $v_M(s)$ in terms of
$v_{M+1}(\cd,\cd).$

\begin{sol}

\end{sol}
\end{problem}

\begin{problem}
 Докажите, что адаптированные случайные процессы $\left\{ \frac{S_n}{(1+r)^n}\right\}$ и
  $\left\{ \frac{X_n}{(1+r)^n}\right\}$ являются мартингалами относительно риск-нейтральной
  меры $(\ti p, \ti q)$. Условие $0<d<1+r<u$ является естественным \emph{необходимым} условием
  отсутствия арбитража. Доказав, что $\left\{ \frac{X_n}{(1+r)^n}\right\}$ мартингал, вы покажете,
  что это условие также является и \emph{достаточным}.

\begin{sol}

\end{sol}
\end{problem}

\begin{problem}
Рассмотрим 3-х периодную модель. Параметры: $S_{0}=8$, $u=1.5$, $d=0.5$, $r=0.1$ \\
Найдите стоимость следущих опционов: \\
\begin{enumerate}
\item американский колл с ценой страйк $K=10$
\item американский пут с ценой страйк $K=10$
\item реализуемый в любой момент времени $t=0,1,2,3$ и приносящий (в случае реализации) платеж $\bar{S}-10$, где $\bar{S}$ — средняя цена акций за время от $t=0$ до момента реализации
\end{enumerate}

\begin{sol}

\end{sol}
\end{problem}

\begin{problem}
 Consider a knockout call option with strike price $K=80$. The main difference with a standard call option is that the option vanishes as soon as the underlying stock goes below a knockout threshold $H=70$. In this event the option pays a \$2 consolation payoff. The parameters are given by $r=0.1$, $u=1.5$, $d=0.5$. The tree covers two periods or three dates, initial stock price is $S_{0}=120$. \\
\begin{enumerate}
\item calculate standard European call price \\
\item calculate knockout European call price
\end{enumerate}

\begin{sol}
\end{sol}
\end{problem}



\subsection{Арбитраж}

\begin{problem}
Имеется три опциона колл с одинаковым сроком исполнения и разными страйк-ценами: $K_{1}<K_{2}<K_{3}$, причем $K_{2}-K_{1}=K_{3}-K_{2}$. Обозначим цены этих опционов как $C_{1}$, $C_{2}$ и $C_{3}$. Сравните $C_{2}$ и $\frac{C_{1}+C_{3}}{2}$

\begin{sol}

\end{sol}
\end{problem}

\begin{problem}
 Пусть $S_{T}$ — цена акции в момент $T$. Экзотический опцион выплачивает сумму $g(S_{T})=min\{max\{S_{T},100\},120\}$. Выразите цену этого опциона через цены двух подходящих опционов колл и процентную ставку.

\begin{sol}

\end{sol}
\end{problem}

\begin{problem}
Докажите, что при отсутствии арбитража, должно выполняться неравенство:
\[
S_{t}-D-K\le C_{t}-P_{t}\le S_{t}-Ke^{-r(T-t)}
\]
здесь:\\

$S$ — цена акции \\
$C$ — цена американского опциона колл со страйк ценой $K$ и датой исполнения $T$\\
$P$ — цена американского опциона пут со страйк ценой $K$ и датой исполнения $T$\\
$D$ — приведенная к моменту $t$ стоимость дивидендов

\begin{sol}
\end{sol}
\end{problem}



\section{Мера}

\begin{problem}
Пусть $X=Y$ (a.e.) и $Y=Z$ (a.e.) Верно ли, что $X=Z$ (a.e.)?

\begin{sol}
да
\end{sol}
\end{problem}

\begin{problem}
Случайные величины $X$ и $Y$ независимы, $f$ и $g$ — измеримые
функции. Докажите, что случайные величины $f(X)$ и $g(Y)$ независимы.

\begin{sol}

\end{sol}
\end{problem}

\begin{problem}
Докажите, что функция распределения имеет не более, чем счетное
число разрывов. Исходя из этого факта, установите, что для любой
случайной величины $X$ можно выбрать бесконечное (более, чем
счетное) количество таких чисел $x$, что $\P(X=x)=0$.

\begin{sol}

\end{sol}
\end{problem}

\begin{problem}
Пусть $\Omega=\{q\in[0;1]|q\in\mathbb{Q}\}$, и $\mathcal{F}_{1}$ -
класс подмножеств множества $\Omega$, имеющих вид $[a;b)$,
$(a;b]$, $[a,b]$ или $(a,b)$, где $a$ и $b$ — рациональные числа.
Обозначим через $\mathcal{F}_{2}$ класс всех конечных сумм попарно
непересекающихся множеств из $\mathcal{F}_{1}$. Определим
функцию $\P$ следующим образом: \\
$\P(A)=b-a$, если $A\in \mathcal{F}_{1}$ \\
$\P(B)=\sum_{i=1}^{n}\P(A_{i})$, если $B\in
\mathcal{F}_{2}$ и $B=\cup_{i=1}^{n}A_{i}$, где $A_{i}$ попарно не
пересекаются.
\begin{enumerate}
\item Верно ли, что $\Omega$ — алгебра? $\sigma$-алгебра?
\item Найдите $\P(\Omega)$.
\item Верно ли, что $\P$ является аддитивной? $\sigma$-аддитивной?
\end{enumerate}

\begin{sol}

\end{sol}
\end{problem}

\begin{problem}
Пусть $\Omega=\{a,b,c,d\}$, и
$\mathcal{A}=\{\{a,b\},\{c,d\},\{a,c\},\{b,d\}\}$. \\
Пусть $\mathcal{F}$ — множество всех подмножеств $\Omega$. \\
Определим вероятности $\P_{1}$ и $\P_{2}$ исходя
из: \\
$\P_{1}(\{a\})=\P_{1}(\{d\})=\frac{1}{6}$,
$\P_{1}(\{b\})=\P_{1}(\{c\})=\frac{1}{3}$, \\
$\P_{2}(\{a\})=\P_{2}(\{d\})=\frac{1}{3}$,
$\P_{2}(\{b\})=\P_{2}(\{c\})=\frac{1}{6}$, \\
a) Верно ли, что $\sigma(\mathcal{A})=\mathcal{F}$? \\
б) Верно ли, что $\P_{1}$ и $\P_{2}$ совпадают на
$\mathcal{A}$? \\
в) Почему не применима теорема Каратеодори о единственности
продолжения меры на порождаемую $\sigma$-алгебру?

\begin{sol}

\end{sol}
\end{problem}

\begin{problem}
Пусть $(\Omega,\mathcal{F},P)$ — вероятностное пространство, $B$ -
событие с $\P(B)>0$. Для $A\in \mathcal{F}$ определим
$P_{B}(A)=\P(A|B)$. Докажите, что $(\Omega,\mathcal{F},P_{B})$ -
вероятностное пространство.

\begin{sol}

\end{sol}
\end{problem}

\begin{problem}
Докажите, что
$\P(\cap_{n=1}^{+\infty}A_{n})=\lim_{T}\P(\cap_{n=1}^{T}A_{n})$

\begin{sol}

\end{sol}
\end{problem}

\begin{problem}
Пусть $f$ и $g$ обычные действительные функции.
\begin{enumerate}
\item Приведите пример таких $f$ и $g$, что $f$ — непрерывна, $g$ -
непрерывна почти наверное (относительно
классической меры Лебега $\lambda$), $f=g$ почти наверное.
\item Приведите примеры таких $f$ и $g$, что $f=g$ почти наверное,
$f$ — непрерывна, $g$ не является непрерывной даже почти наверное.
\item Приведите пример такой $f$, непрерывной почти наверное, что ни
одна $g$ почти наверное равная $f$, не является непрерывной.
\end{enumerate}

\begin{sol}

\end{sol}
\end{problem}

\begin{problem}
Две $\sigma$-алгебры $\mathcal{F}_{1}$ и $\mathcal{F}_{2}$
независимы. Пусть $\mathcal{F}_{1}^{'}\subseteq \mathcal{F}_{1}$ и
$\mathcal{F}_{2}^{'}\subseteq \mathcal{F}_{2}$. Докажите, что
$\sigma$-алгебры $\mathcal{F}_{1}^{'}$ и $\mathcal{F}_{2}^{'}$
независимы.

\begin{sol}

\end{sol}
\end{problem}

\begin{problem}
Про независимость в лимсуп/лиминф 1.8.16, 1.8.32

\begin{sol}

\end{sol}
\end{problem}

\begin{problem}
 (надо проверить, решабельно ли это до интеграла лебега) \\
Докажите неравенства Фату для вероятности: \\
$\P(\liminf_{n}A_{n})\le \liminf_{n}\P(A_{n})$ и
$\limsup_{n}\P(A_{n})\le \P(\limsup_{n}A_{n})$

\begin{sol}

\end{sol}
\end{problem}

\begin{problem}
 А что такое метрика? \\
Докажите, что на плоскости $(x,y)$ обе функции:
1. $\rho((x_{1},x_{2}),(y_{1},y_{2}))=\sqrt{(x_{1}-y_{1})^{2}+(x_{2}-y_{2})^{2}}$ \\
2. $\rho((x_{1},x_{2}),(y_{1},y_{2}))=|x_{1}-y_{1}|+|x_{2}-y_{2}|$ \\
являются метриками

\begin{sol}

\end{sol}
\end{problem}

\begin{problem}
Пусть $X$ и $Y$ одинаково распределены, и $f(t)$ — борелевская
функция. \\
Верно ли, что $f(X)$ и $f(Y)$ одинаково распределены?

\begin{sol}

\end{sol}
\end{problem}

\begin{problem}
Известно, что $ \P(X>c \Delta Y>c)=0 $ для любой константы $ c $. Верно ли, что $X=Y$ почти наверное?

\begin{sol}
Да. Рассмотрим множество $ X-Y>c $. Его можно представить в виде $ \{X-Y>c\}=\cup_{q} \{ X>q \cap Y<q-c\} $. Далее замечаем, что $ \P(X>q \cap Y<q-c)=\P(Y>q \cap Y<q-c)=0 $ для $ c>0 $. Т.е. $ \P(X-Y>c)=0 $ для $ c>0 $. В силу непрерывности вероятности получаем $ \P(X-Y>0)=0 $. Далее используем симметрию и получаем $ \P(Y-X>0)=0 $. Значит $ \P(X=Y)=1 $
\end{sol}
\end{problem}

\begin{problem}
Имеется последовательность событий $A_1$, $A_2$, \ldots, такая что вероятность любого $A_i$ равна $\P(A_i)=1$. Найдите $\P(\cup A_i)$ и $\P(\cap A_i)$

\begin{sol}
Обе равны 1
\end{sol}
\end{problem}

\begin{problem}
Имеется последовательность событий $A_1$, $A_2$, \ldots, такая что $\lim \P(A_i)=1$. Найдите $\lim \P(B\cap A_i)$

\begin{sol}
$\P(B)$
\end{sol}
\end{problem}

\begin{problem}
Существует ли последовательность событий $A_1$, $A_2$, \ldots, такая что $\lim \P(A_i)=1$, но $\P(\cap_{i=k}^{\infty} A_i)=0$ для любого числа $k$?

\begin{sol}
Да
\end{sol}
\end{problem}


\section{Виды сходимостей}

\begin{problem}
Известно, что $X_{n} \ge 0$. Верно ли, что
$\E(\sum_{n=1}^{+\infty}X_{n})=\sum_{n=1}^{+\infty}\E(X_{n})$?

\begin{sol}

\end{sol}
\end{problem}

\begin{problem}
$[$Grimmett, ex. 7.11.1$]$ \\
Пусть $X_{n}$ заданы на одном вероятностном пространстве, и их
функции плотности имеют вид
$p_{n}(t)=\frac{n}{\pi(1+n^{2}t^{2})}$ \\
К чему сходится $X_{n}$ и в каком смысле (as, $L^{1}$, in
probability, in distribution)?

\begin{sol}

in probability (ok), in distribution (ok)
\end{sol}
\end{problem}

\begin{problem}
Пусть $X_{n}$ и $Y_{n}$ заданы на одном вероятностном пространстве
и $X_{n}\rightarrow X$ (in distribution), $Y_{n}\rightarrow Y$ (in
distribution). Верно ли, что $X_{n}+Y_{n}\rightarrow X+Y$ (in
distribution)?

\begin{sol}
no
\end{sol}
\end{problem}

\begin{problem}
Пусть $X_{n}$ — имеет распределение $Bin(n,0.5)$. И $Y_{n}$ — остаток от деления $X_{n}$ на 5 (можно и не на пять). К чему сходится $Y_{n}$ по распределению? \\
решение (через характеристические функции?) \\
см. aops, t=174723

\begin{sol}

\end{sol}
\end{problem}

\begin{problem}
Converges AS\\
Show that $X_{n}$ converges to $X$ a.s. iff for all $\varepsilon$, there exists $n$ s.t. for every random integer $N(w)>n$ (for all $w$), we have $\P(|X_N(w) (w) — X (w)| > \varepsilon) <\varepsilon$.\\
решение — ?, aops, t=174817

\begin{sol}

\end{sol}
\end{problem}

\begin{problem}
Пусть $X$ — интегрируемая и $A_{n}$ попарно не пересекаются.
Обозначим $A=\cup A_{n}$. Докажите, что
$\sum_{n=1}^{+\infty}\E(X\cdot 1_{A_{n}})=\E(A\cdot 1_{A})$.

\begin{sol}

\end{sol}
\end{problem}

\begin{problem}
Если $f$ — непрерывна и $X_{n}\rightarrow X$ a.e., то
$f(X_{n})\rightarrow f(X)$ a.e.

\begin{sol}

\end{sol}
\end{problem}

\begin{problem}
Докажите или опровергните утверждения:
\begin{enumerate}
\item Если $Y_{n}\uparrow Y$, то $1_{Y_{n}>a}\uparrow 1_{Y>a}$
\item Если $Y_{n}\uparrow Y$, то $1_{Y_{n}\ge a}\uparrow 1_{Y\ge a}$ \\
\item Если $1_{Y_{n}>a}\uparrow 1_{Y>a}$, то $\P(Y_{n}>a)\uparrow
\P(Y>a)$ \\
\item Если $Y_{n}\rightarrow Y$ (as), то
$\lim_{n\to\infty}1_{Y_{n}>x}\ge 1_{Y>x}$ (as) \\
\item Если $Y_{n}\rightarrow Y$ (as), то
$\lim_{n\to\infty}1_{Y_{n}\ge x}\le 1_{Y\ge x}$ (as)
\end{enumerate}

\begin{sol}

b) easy counterexample by Sasha Serova \\
$Y_{n}=1-1/n$, $Y_{n}\uparrow 1$, take $a=1$, $1_{Y_{n}\ge 1}=0$,
$1_{Y\ge 1}=1$
\end{sol}
\end{problem}

\begin{problem}
Пусть $X\sim U(0;1)$ и $X_{n}=(\cos(\frac{2\pi}{X}))^{n}$. Верно
ли, что
$X_{n}\rightarrow 0$ (a.s)?

\begin{sol}

\end{sol}
\end{problem}

\begin{problem}
Пусть $X\in L^{1}$, и последовательность событий $A_{n}$ такова,
что $\P(A_{n})\rightarrow 0$.
\begin{enumerate}
\item Докажите, что $|X| 1_{A_{n}} \rightarrow 0$ (in probability)
\item Используя DCT и тот факт, что в последовательности, сходящейся
по вероятности, можно выделить подпоследовательность, сходящуюся
почти наверное, покажите, что $E\left(|X| 1_{A_{n}}\right)
\rightarrow 0$
\item Для $\varepsilon>0$ существует такое $\delta>0$,
что из $\P(A)<\delta$ следует $\E(|X|1_{A})<\varepsilon$
\item Докажите, что $\lim_{T\to\infty}\P(|X|>T)=0$
\end{enumerate}

\begin{sol}

\end{sol}
\end{problem}

\begin{problem}
Пусть $X_{n}\rightarrow X$ (in $L^{1}$). Докажите, что
$\E(X_{n}1_{A})\rightarrow \E(X 1_{A})$.

\begin{sol}

$|\E(X_{n}1_{A})-\E(X 1_{A})|\le \E(|X_{n}1_{A}-X 1_{A}|)\le
\E(|X_{n}-X|)\rightarrow 0$
\end{sol}
\end{problem}

\begin{problem}
Пусть $X_{n}\le Y_{n}$ (as) и $X_{n}\rightarrow X$ (in $L^{1}$), а
$Y_{n}\rightarrow Y$ (in $L^{1}$). Верно ли, что $X\le Y$ (as)?

\begin{sol}

Верно. Рассмотрим любое $\varepsilon>0$. \\
$E|X-X_{n}|+E|Y_{n}-Y|\ge E|X-Y+Y_{n}-X_{n}|\ge E\left(
|X-Y+Y_{n}-X_{n}| 1_{X-Y>\varepsilon}1_{Y_{n}\ge X_{n}}\right)\ge
\varepsilon \P(X-Y>\varepsilon)$ \\
Следовательно, для любого $\varepsilon$ вероятность
$\P(X-Y>\varepsilon)=0$ (т.к. левая часть стремится к 0 при
$n\rightarrow +\infty$)
\end{sol}
\end{problem}

\begin{problem}
$[$Williams, E4.7$]$ \\
Ваш выигрыш в $n$-ой партии равен $X_{n}=
\begin{cases}
  n^{2}-1, & p=n^{-2} \\
  -1, & p=1-n^{-2} \\
\end{cases}$ и $X_{n}$ — независимы. \\
\begin{enumerate}
\item Найдите $\E(X_{n})$.
\item Пусть $A_{t}=\{$не будет ни одного выигрыша начиная с времени
$t\}$. Найдите $\P(A_{t})$ и $\P(\cup_{t} A_{t})$.
\item В каком смысле существует $\lim_{n}\bar{X}_{n}$ и чему он
равен?
\item Можно ли назвать эту игру справедливой?
\end{enumerate}

\begin{sol}


$\P(A_{t})=\frac{t-1}{t}$ и $\P(\cup_{t} A_{t})=1$ \\
То есть с вероятностью 1 будет лишь конечное число выигрышей. Другими
словами $\bar{X}_{n} \rightarrow -1$ (as)
\end{sol}
\end{problem}

\begin{problem}
Пусть $X\sim U[0;1]$ и $X_{n}=\sqrt{n}(1-X)^{n}$. \\
Куда и в каком смысле сходится $X_{n}$?

\begin{sol}

\end{sol}
\end{problem}

\begin{problem}
Пусть $U_{n}\sim U[0;1]$ — независимые случайные величины. Пусть
$X_{n}=\min\{U_{1},\ldots,U_{n}\}$
\begin{enumerate}
\item К чему и в каком смысле сходится $X_{n}$?
\item При каких $\theta$ последовательность $Y_{n}=n^{\theta}X_{n}$
будет сходиться по распределению? К чему она будет
сходиться?
\item К чему и в каком смысле сходится $Z_{n}=(U_{1}\cdot\ldots\cdot U_{n})^{1/n}$?
\end{enumerate}

\begin{sol}

\end{sol}
\end{problem}

\begin{problem}
Пусть $X_{0}\sim U[0;1]$ и $X_{n}=6+\sqrt{X_{n-1}}$. \\
К чему и в каком смысле сходится $X_{n}$?

\begin{sol}

\end{sol}
\end{problem}

\begin{problem}
Вероятностное пространство $(\Omega,\mathcal{F},P)$ называется
дискретным, если $\Omega$ конечно или счетно (чаще всего при этом
полагают, что $\mathcal{F}$ содержит все подмножества $\Omega$).
Верно ли, что для дискретного вероятностного пространства
сходимость почти наверное и сходимость по вероятности — это одно и
то же?

\begin{sol}

\end{sol}
\end{problem}

\begin{problem}
Пусть величины $X_{1}$, $X_{2}$, \ldots~имеют нулевое среднее, ограниченную общей константой дисперсию и некоррелированы. Пусть $S_{n}=X_{1}+\ldots+X_{n}$ и $\alpha>1/2$.\\
Сходится ли $\frac{S_{n}}{n^{\alpha}}$ в $L^{2}$?

\begin{sol}

\end{sol}
\end{problem}

\begin{problem}
 breaking a stick \\
A stick of length 1 is broken at a random point that is uniformly distributed. The right hand part is thrown away, and the left part is broken similarly (the breaking point being uniformly distributed). Let $X_{n}$ be the length of the remaining part after $n$ steps. We define $Y_{n}=X_{n}^{1/n}$. \\
In what sense does the $Y_{n}$ converge? What is the limit?

\begin{sol}

limit is $1/e$ almost surely \\
hint?: It's the law of large numbers. Show that the logarithm of a uniform 0-1 has expected value (-1). Its negative is an exponential distribution with parameter 1.
\end{sol}
\end{problem}

\begin{problem}
Рассмотрим вероятностное пространство
$(\Omega,\mathcal{B}[0;1],P)$, где $P=\lambda$ — классическая
мера Лебега.\\
Пусть $A_{n}=[0;\frac{1}{n}]$.
\begin{enumerate}
\item Найдите $\P(\limsup A_{n})$.
\item Найдите $\sum_{n=1}^{\infty}\P(A_{n})$.
\item Противоречит ли этот пример лемме Бореля-Кантелли?
\end{enumerate}

\begin{sol}

\end{sol}
\end{problem}

\begin{problem}
Приведите пример последовательности $X_{n}$ и предела $X$, таких что: $\lim E|X_{n}-X|=0$, но $\E(|X_{n}|)=\infty$

\begin{sol}
 В качестве $X_{n}$ и $ X $ берем любую случайную величину с бесконечным мат. ожиданием
\end{sol}
\end{problem}

\begin{problem}
Let $X_{n}=X\cdot 1_{X\leq n}$. In what sense (as., in $L^{2}$, in probability, in law) does the sequence $ \{X_{n}\} $ converge? What is the limit? Consider two cases:
\begin{enumerate}
\item $X\sim N(0;1)$ (standard normal distribution).
\item $X =2^{T}$ where $T$ is the time of the first head in the infinite sequence of coin tosses.
\end{enumerate}


\begin{sol}
 $ X_{n}\to X $. a) as, P, law, $ L^{2} $; b) as, P, law
\end{sol}
\end{problem}

\begin{problem}
Случайная величина $X_1$ равномерна на отрезке $[0;1]$ и $X_n=X_1^n$. В каких смыслах и к чему сходится $X_n$?

\begin{sol}
К нулю, кроме как поточечно.
\end{sol}
\end{problem}

\begin{problem}
Последовательность случайных величин $X_n$ такова, что $\P(X_n\to 0)=1$. Возможно ли, что $\lim \E(X_n)\neq 0$?

\begin{sol}

\end{sol}
\end{problem}

\begin{problem}
Известно, что $X_n \xrightarrow{\P} 0$. Верно ли, что $\Var(X_n)\to 0$?

\begin{sol}
Нет.
\end{sol}
\end{problem}

\begin{problem}
Известно, что $X_n \xrightarrow{\P} X$. Верно ли, что $Y\cdot X_n \xrightarrow{\P} YX$?

\begin{sol}
Да
\end{sol}
\end{problem}

\begin{problem}
Известно, что $X_n \xrightarrow{\P} X$ и $Y_n \xrightarrow{\P} Y$. Верно ли, что $X_nY_n \xrightarrow{\P} XY$?

\begin{sol}
Да
\end{sol}
\end{problem}

\begin{problem}
Известно, что $\varepsilon>0$. Сравните величины $\P(X\leq t)$ и $\P(Y\leq t+\varepsilon)+\P(|X-Y|>\varepsilon)$.

\begin{sol}
$\P(X\leq t)$ меньше
\end{sol}
\end{problem}

\begin{problem}
Какое условие является более сильным?
\begin{enumerate}
\item Последовательность $X_n$ сходится к $X$ по вероятности
\item \[
\E\left(\frac{|X_n-X|}{1+|X_n-X|}\right)\to 0
\]
\end{enumerate}

\begin{sol}
Условия равносильны
\end{sol}
\end{problem}

\begin{problem}
Последовательность случайных величин $X_n$ поточечно сходится к случайной величине $X$. Кроме того, $\lim \E(X_n)=0$.
\begin{enumerate}
\item Верно ли, что $\E(X)$ обязательно существует?
\item Если дополнительно известно, что $\E(X)$ существует, то верно ли, что $\E(X)=0$?
\end{enumerate}

\begin{sol}

\end{sol}
\end{problem}

\begin{problem}
Последовательность случайных величин $X_n$ сходится к $a$ по вероятности. Функция $f$ — борелевская и имеет производную в точке $x=a$. Представим $f(X_n)$ в виде
\begin{equation}
f(X_n)=f(a)+f'(a)(X_n-a)+Y_n(X_n-a)
\end{equation}
Сходится ли $Y_n$ по вероятности? Если да, то к чему?

\begin{sol}
Сходится по вероятности к нулю
\end{sol}
\end{problem}

\begin{problem}
Известно, что $X_n$ сходится по вероятности к нулю и $\lim_{t\to\infty} \sup_n \P(|Y_n|>t)=0$. Верно ли, что $X_n Y_n$ сходится по вероятности к нулю?

\begin{sol}
да,
\begin{multline}
\P(|X_nY_n|>\varepsilon)=\P(|X_nY_n|>\varepsilon, |Y_n|>t)+\P(|X_nY_n|>\varepsilon, |Y_n|\leq t)\leq \\
\leq \underbrace{\P(|X_n|>\varepsilon/t)}_{\to 0}+\underbrace{\P(|Y_n|>t)}_{\to 0}
\end{multline}
\end{sol}
\end{problem}

\begin{problem}
Хвостовая $\sigma$-алгебра. Случайные величины $X_1$, $X_2$, \ldots независимы, $Y_n=X_1\cdot X_2\cdot\ldots\cdot X_n$. Каждая $X_i$ равновероятно принимает значения $0.5$ и $1.5$.
\begin{enumerate}
\item Найдите $\P(Y_n \to 0)$
\item Как изменится ответ, если $X_i$ равновероятно принимает значения $n/(n+1)$ и $(n+2)/(n+1)$?
\item Как изменится ответ, если $X_i$ равновероятно принимает значения $1-1/\sqrt{n+1}$ и $1+1/\sqrt{n+1}$?
\end{enumerate}

\begin{sol}
Событие $Y_n\to 0$ лежит в хвостовой $\sigma$-алгебре, поэтому $\P(Y_n\to 0)$ может равняться только нулю или единице.
\end{sol}
\end{problem}

\section{Интеграл Лебега}

\begin{problem}
Cлучайные величины $X$ и $Y$ являются интегрируемыми и независимыми
(то есть независимы минимальные $\sigma$-алгебры, ими
порождаемые). Докажите, что $\E(XY)=\E(X)\E(Y)$. \\
Воспользуйтесь стандартной техникой: индикатор — простая -
неотрицательная — произвольная

\begin{sol}

\end{sol}
\end{problem}

\begin{problem}
Пусть $(\Omega,\mathcal{F},P)$ — вероятностное пространство, $X$ -
случайная величина, причем $X\ge 0$ и $0<\E(X)< + \infty$. Для
произвольного события $A\in\mathcal{F}$ определим
$P_{X}(A)=\frac{\E(X\cdot 1_{A})}{\E(X)}$.
\begin{enumerate}
\item Докажите, что $(\Omega,\mathcal{F},P_{X})$ — вероятностное пространство.
\item Обозначим $E_{X}(Y)=\int Y \cdot dP_{X}$. Докажите, что
если $Y\ge 0$, то $E_{X}(Y)=\frac{\E(XY)}{\E(X)}$.
\end{enumerate}

\begin{sol}

\end{sol}
\end{problem}

\begin{problem}
Пусть $X$ — дискретная случайная величина, со счетным множеством
значений $\{a_{1},a_{2},\ldots\}$, где $a_{i} \ge 0$. Докажите, что
$\E(X)=\sum_{i=1}^{+\infty} a_{i}\cdot \P(X=a_{i})$.

\begin{sol}

\end{sol}
\end{problem}

\begin{problem}
Рассмотрим вероятностное пространство
$([0;1],\mathcal{B},\lambda)$. \\
Пусть $q(x)=
\begin{cases}
  1, & x\in \mathbb{Q} \\
  0, & x\notin \mathbb{Q} \\
\end{cases}$ — функция Дирихле.
\begin{enumerate}
\item Является ли $q$ измеримой?
\item Найдите $\int_{\Omega}q \cdot d\lambda$ (интеграл Лебега).
\item Докажите, что $\int_{0}^{1} q \cdot dx$ (интеграл Римана) не
существует.
\end{enumerate}

\begin{sol}

\end{sol}
\end{problem}

\begin{problem}
Докажите, что условия (1) и (2) равносильны: \\
(1) $\E(X 1_{F})\le \E(Y 1_{F})$ для любого события $F$ \\
(2) $X\le Y$ (as)

\begin{sol}


(2) $\rightarrow$ (1): $X\le Y$ (as) $\Rightarrow$ $X 1_{F}\le Y
1_{F}$ (as) $\Rightarrow$ $\E(X 1_{F})\le \E(Y 1_{F})$ \\
(1) $\rightarrow$ (2): для $F=\{X-Y>\varepsilon\}$ получаем
$\E((X-Y) 1_{F})\le
 0$, но $\E((X-Y) 1_{F})\ge \E(\varepsilon 1_{F})=\varepsilon \P(F)$.
Значит $\P(X-Y>\varepsilon)=0$
\end{sol}
\end{problem}

\begin{problem}
Пусть $X\in L^{1}$, $Y\in L^{1}$. Верно ли, что $\max\{X,Y\}\in
L^{1}$?

\begin{sol}

\end{sol}
\end{problem}

\begin{problem}
\label{L beskonechnost} Обозначим
$||X||_{p}=(\E(|X|^{p}))^{\frac{1}{p}}$ и
$||X||_{\infty}=\inf\{t|\P(|X|\le t)=1\}$.
\begin{enumerate}
\item Докажите, что $||X||_{p}\le ||X||_{\infty}$ при $1\le p\le
\infty$.
\item Докажите, что $\lim_{p\to\infty}||X||_{p}=||X||_{\infty}$.
\end{enumerate}

\begin{sol}
\begin{enumerate}
\item  Обнулим величину $X$ там, где она больше, чем $||X||_{\infty}$.
\item Обозначим $a=||X||_{\infty}$. Тогда $\P(X>a-\varepsilon)>0$. И
$X\ge (a-\varepsilon)1_{X>a-\varepsilon}$. Значит $||X||_{p}\ge
(a-\varepsilon)||1_{X>a-\varepsilon}||_{p}$.
\end{enumerate}
\end{sol}
\end{problem}

\begin{problem}
\label{X2Y2XY} Пусть $X\in L^{2}$, $Y\in L^{2}$.
\begin{enumerate}
\item Верно ли, что $X^{2}\in L^{1}$?
\item Верно ли, что $XY\in L^{1}$?
\end{enumerate}

\begin{sol}
\begin{enumerate}
\item
\item $\E(X^{2}+Y^{2}-2XY)\ge 0$
\end{enumerate}
\end{sol}
\end{problem}

\begin{problem}
Two random variables X and Y have the same distiribution and that X<=Y almost surely.
Under this condition I have to show that X=Y almost surely.

\begin{sol}

We know that X>Y with probability zero. Assume that X<Y with probability greater than zero. Then, the expectation of X would have to be less than the expectation of Y. And the distributions would be different — hence contradiction. So thereofre X=Y with probability 1 (almost surely)
\end{sol}
\end{problem}

\begin{problem}
Известно, что $\Var(X)=0$. Существует ли константа $a$ такая, что $0<\P(X=a)<1$?

\begin{sol}
Нет, $X$ — почти наверное константа.
\end{sol}
\end{problem}

\begin{problem}
Известно, что $\P(0<X<1)=1$. Сравните $\Var(X)$ и $\E(X)$.

\begin{sol}
$\Var(X)<\E(X)$
\end{sol}
\end{problem}

\begin{problem}
Задана последовательность функций $f_n(x)=\frac{n^2 x e^{-n^2 x^2}}{1+x^2}$. Сходится ли числовая последовательность $a_n=\int_{[0;+\infty)}f_n d\lambda$, где $\lambda$ — классическая мера Лебега?

\begin{sol}

\end{sol}
\end{problem}

\begin{problem}
Функции $f_n$ заданы на отрезке $[0;1]$ и интегрируемы. Последовательность $f_n$ сходится равномерно к функции $f$ по классической мере Лебега $\lambda$. Верно ли, что $\int_{[0;1]} f\,d\lambda$ существует? Верно ли, что $\int_{[0;1]} f_n\,d\lambda\to \int_{[0;1]} f\,d\lambda$?

\begin{sol}
Да
\end{sol}
\end{problem}

\begin{problem}
Функции $f_n$ заданы на множестве $[0;\infty)$ формулой $f_n(x)=2nxe^{-nx^2}$. Существует ли  интегрируемая функция $f$, такая что $f\geq f_n$ для всех $n$?

\begin{sol}
\end{sol}
\end{problem}


\section{Характеристическая функция}

\begin{problem}
$[$Williams, 16.2$]$ \\
Пусть $U\sim U[-1;1]$.
\begin{enumerate}
\item Найдите характеристическую функцию $\phi_{U}(t)$.
\item Докажите, что $U$ не представима в виде $X-Y$, где $X$ и $Y$ независимы и одинаково распределены.
\end{enumerate}

\begin{sol}


\begin{enumerate}
\item $\phi_{U}=(t)=\sin(t)/t$
\item $\phi(t)\phi(-t)=|\phi(t)|^{2}=\sin(t)/t$, но правая часть
бывает отрицательной, противоречие.
\end{enumerate}
\end{sol}
\end{problem}

\begin{problem}
Пусть величины $X$ и $Y$ имеют совместную функцию плотности: \\
$f(x,y)=\frac{1}{4}(1+xy(x^{2}-y^{2})$, при $|x|<1$, $|y|<1$.
\begin{enumerate}
\item  Верно ли, что $X$ и $Y$ независимы?
\item Верно ли, что $\phi_{X+Y}(t)=\phi_{X}(t)\phi_{Y}(t)$?
\end{enumerate}

\begin{sol}

\end{sol}
\end{problem}

\begin{problem}
Пусть $X$ имеет функцию плотности Коши,
$p(t)=\frac{1}{\pi(1+t^{2})}$. \\
Верно ли, что $\phi_{2X}(t)=\phi_{X}^{2}(t)$?

\begin{sol}

\end{sol}
\end{problem}

\begin{problem}
Пусть $X\sim N(0;1)$ и $W=X^{4}$ \\
Докажите, что:
\begin{enumerate}
\item существует $\varphi(t)=\E(e^{itW})$
\item не существует $f(t)=\E(e^{tW})$
\end{enumerate}

\begin{sol}

\end{sol}
\end{problem}

\begin{problem}
Пусть $X\sim N(0;1)$.
\begin{enumerate}
\item Чему равно $\E(X^{2n+1})$?
\item Докажите, что $\E(X^{2n})=1\cdot 3\cdot 5\cdot\ldots\cdot(2n-1)$.
\item Пусть $Y\sim N(0;\sigma^{2})$. Найдите $\E(Y^{2n})$.
\end{enumerate}

\begin{sol}

\end{sol}
\end{problem}

\begin{problem}
Случайная величина $X$ задана табличкой: \\
$\begin{tabular}{cccc}
\toprule
  $x_{i}$ & 0 & 1 & 2 \\
  $\P(X=x_{i})$ & 0.1 & 0.3 & 0.6 \\
\bottomrule
\end{tabular}$

\begin{enumerate}
\item Найдите функцию $g(t)=\E(t^{X})$ и постройте ее график.
\item Чему равно $g(1)$?
\item Верно ли, что $g(0)=\P(X=0)$?
\item Верно ли, что $g(1)=\P(X=1)$?
\item Найдите $g'(1)$ и $\E(X)$.
\item Найдите $g''(1)$ и $\E(X(X-1))$.
\item Выразите $\Var(X)$ через $g'(1)$ и $g''(1)$.
\end{enumerate}

\begin{sol}

\end{sol}
\end{problem}

\begin{problem}
  Случайной величине $X$, чей закон распределения задан таблицей: \\
\begin{tabular}{cccc}
\toprule
% after \\: \hline or \cline{col1-col2} \cline{col3-col4} \ldots
  $x_{i}$ & 0 & 1 & 2 \\
  $(X=x_{i})$ & 0.3 & 0.2 & 0.5 \\
\bottomrule
\end{tabular}

сопоставим «производящую» функцию
$f(t)=0.3t^{0}+0.2t^{1}+0.5t^{2}$ \\
\begin{enumerate}
\item Какой вероятностный смысл имеют величины $f'(1)$,
$f''(1)+f'(1)-(f'(1))^{2}$?
\item Как связаны $\P(X=x_{i})$ и $f^{(i)}(0)$?
Примечание: $f^{(i)}(0)$ — это $i$-ая производная.
\end{enumerate}

\begin{sol}

\end{sol}
\end{problem}

\begin{problem}
Случайная величина $Y$ задана табличкой: \\
$\begin{tabular}{cccc}
\toprule
  $y_{i}$ & 0 & 1 & -1 \\
  $\P(Y=y_{i})$ & 0.3 & 0.2 & 0.5 \\
\bottomrule
\end{tabular}$

\begin{enumerate}
\item Найдите функцию $g(t)=\E(t^{X})$ и постройте ее график.
\item Чему равно $g(1)$? \\
\item Верно ли, что $g(0)=\P(X=0)$?
\item Верно ли, что $g(1)=\P(X=1)$?
\item Найдите $g'(1)$ и $\E(X)$.
\item Найдите $g''(1)$ и $\E(X(X-1))$.
\item Выразите $\Var(X)$ через $g'(1)$ и $g''(1)$.
\end{enumerate}


\begin{sol}

\end{sol}
\end{problem}

\begin{problem}
Пусть $S_{n}$ — симметричное случайное блуждание.
\begin{enumerate}
\item Найдите производящую функцию для $S_{n}$.
\item Найдите $\E(S_{n}^{4})$.
\end{enumerate}

\begin{sol}

\end{sol}
\end{problem}

\begin{problem}
Пусть $X$ и $Y$ независимы и их сумма нормальна. \\
Верно ли, что $X$ и $Y$ нормальны?

\begin{sol}
 берем любые две характеристические функции, чтобы получалось в итоге нормальное распределение.
\end{sol}
\end{problem}

\begin{problem}
Пусть $g(t)$ — производящая функция целочисленной неотрицательной случайной величины $X$. Известно, что $g(0)=0$ и $g(\frac{1}{3})=\frac{1}{3}$. Что можно сказать о законе распределения $X$?

\begin{sol}

Во-первых, $p_{0}=0$ \\
Во-вторых, $g(\frac{1}{3})=p_{1}\frac{1}{3}+\sum_{k=2}^{\infty}p_{k}\left(\frac{1}{3}\right)^{k}\le p_{1}\frac{1}{3}+\sum_{k=2}^{\infty}(1-p_{1})\left(\frac{1}{3}\right)^{k}=p_{1}\frac{1}{3}+(1-p_{1})\frac{1}{6}=\frac{1+p_{1}}{6}$ \\
Единственно возможный вариант это $p_{1}=1$. \\
Т.е. $X$ — это константа единица.
\end{sol}
\end{problem}

\begin{problem}
Вася хватается за точку $t$ на отрезке $[0;1]$. Затем на отрезке делается $n$ равномерно выбираемых случайных разрезов.
Васе достается тот кусок, за который он держится. Какова средняя длина куска, достающегося Васе?

\begin{sol}

Если не считать Васю, то $n$ случайных точек разрезают отрезок на куски со средней длиной $\frac{1}{n+1}$.
Слева от Васи оказывается $L+1$ кусок, где $L$ — число разрезов, попавших слева от Васи. Значит, средняя длина от Васи до ближайшей точки налево равна
$\E(\frac{t}{L+1})$, где $L$ — $Bin(n,t)$. Аналогично, от Васи до ближайшего разреза справа — $\E(\frac{t}{R+1})$, где $R$ — $Bin(n,1-t)$.

Пусть $L$ — $Bin(n,t)$. Считаем $\E(\frac{1}{L+1})$. Для начала находим $\E(x^{L})=(tx+1-t)^{n}$.
Затем замечаем, что $\frac{1}{L+1}=\int_{0}^{1}x^{L}$. Значит $\E(\frac{1}{L+1})=\int_{0}^{1}\E(x^{L})=\ldots=\frac{1-(1-t)^{n+1}}{t(n+1)}$.
Получаем ответ: $\E(\frac{t}{L+1})+\E(\frac{t}{R+1})=\frac{2-(1-t)^{n+1}-t^{n+1}}{n+1}$.
Чтобы кусок получался длиннее, лучше хвататься по центру!
\end{sol}
\end{problem}




\section{Условное ожидание}

\begin{problem}
 Borel-Kolmogorov paradox \\
По статье из Wikipedia \\
Пусть совместная функция плотности $X$ и $Y$ имеет вид: \\
$p_{X,Y}(x,y)=
\begin{cases}
  1, & 0<y<1, -y<x<1-y \\
  0, & \text{ иначе } \\
\end{cases}$ \\
Пусть $U=\frac{X}{Y}+1$.

\begin{enumerate}
\item Найдите функцию плотности $p_{Y|X}(y|x)$.
\item Найдите функции плотности  $p_{U,Y}(u,v)$ и $p_{Y|U}(y,u)$.
\item Верно ли, что $X=0 \Leftrightarrow U=1$?
\item Верно ли, что $p_{Y|X}(t|0)=p_{Y|U}(t,1)$?
\item Мораль?
\end{enumerate}

\begin{sol}

\end{sol}
\end{problem}

\begin{problem}
Пусть $X$ — равномерна на отрезке $[0;1]$. В шляпе лежат две свернутые бумажки. На одной бумажке написано $X$, на другой $X^{2}$. Вы тяните одну бумажку наугад. Пусть $Z$ — число, написанное на вытянутой Вами бумажке, а $W$ — число на другой бумажке. Увидев число, Вы решаете, оставить себе эту бумажку, или отказаться от этой и забрать оставшуюся. Ваш выигрыш — число на оставшейся у Вас бумажке. \\

\begin{enumerate}
\item Найдите $\E(W|Z)$.
\item Максимально подробно (кубическое уравнение там будет суровое, не решайте его) опишите стратегию, максимизирующую Ваш выигрыш.
\item Как изменится результат, если на одной бумажке написано значение $X$, а на второй — значение случайной величины, имеющей такое же распределение, как и $X^{2}$, но независимой от $X$?
\end{enumerate}



\begin{sol}

$\E(W|Z)=\frac{2Z^{2}+1}{2+\frac{1}{\sqrt{Z}}}$ \\
Будет пороговое $b\approx 0.32$ (кубическое уравнение с плохими корнями). Если $Z<b$, то брать текущую, если $Z>b$, то менять выбор. \\
Source: aops, t=173650
\end{sol}
\end{problem}

\begin{problem}
Пусть $X$ и $Y$ независимы и экспоненциально распределены с параметром $\lambda$. \\
Найдите $\E(X|X+Y)$, $\E(X+Y|X-Y)$, $\E(X|X-Y)$, $\E(X||X-Y|)$.

\begin{sol}

 $\E(X+Y|X-Y)=|X-Y|+\frac{1}{\lambda}$
\end{sol}
\end{problem}

\begin{problem}
Пусть $X\sim U[0;100]$, $Y\sim U[-1;1]$ и независимы. Найдите $\E(X|X+Y)$.

\begin{sol}

\end{sol}
\end{problem}

\begin{problem}
Внутри круга радиуса 1 равномерно выбирается точка. Пусть $X$ и $Y$ — ее абсцисса и ордината. Найдите совместную функцию плотности $p(x,y)$, частную функцию плотности $p(x)$, условную функцию плотности $p(x|y)$, $\E(X|Y)$, $\E(X^{2}|Y)$, $Cov(X,Y)$. \\
Являются ли $X$ и $Y$ независимыми?

\begin{sol}

\end{sol}
\end{problem}

\begin{problem}
Suppose (X,Y) is a random vector with a distribution given by $\P(X=i \cap Y=j)=1/10$  for $1\le i\le j\le 4$. \\
Find $\E(Y|X)$

\begin{sol}
Составляем таблицу $4 \times 4$. Находим $\E(Y|X=1)=2.5$, $\E(Y|X=2)=3$, $\E(Y|X=3)=3.5$, $\E(Y|X=4)=4$.

И, $\E(Y|X)=2+0.5X$.
\end{sol}
\end{problem}

\begin{problem}
Пусть $X\sim U[0;1]$, $Y$ равновероятно равно 0 и 1, $X$ и $Y$ независимы, $Z=X^{Y}$.
Найдите $\E(Z|Y)$, $\Var(Z|Y)$, $\E(Z|X)$, $\Var(Z|X)$

\begin{sol}

\end{sol}
\end{problem}

\begin{problem}
 Правильный кубик подбрасывается один раз. $X$ — число очков, выпавшее на кубике. $Y$ — индикатор того, выпала ли четная грань. $Z$ — индикатор того, выпало ли число больше 2-х. \\
Найдите закон распределения (проще говоря, заполните табличку) для случайных величин $\E(XY|XZ)$, $\E(Z|X)$, $\E(X|Z)$ \\
Табличка для заполнения: \\
$\begin{tabular}{ccccccc}
\toprule
$\Omega$ & $w_{1}$ & $w_{2}$ & $w_{3}$ & $w_{4}$ & $w_{5}$ & $w_{6}$ \\
$\E(XY|XZ)$ & & & & & & \\
$\E(Z|X)$ & & & & & & \\
$\E(X|Z)$ & & & & & & \\
\bottomrule
\end{tabular}$

\begin{sol}
\[
\E(XY|XZ) = \begin{cases}
1, XZ = 0 \\
0, XZ = 3 \\
4, XZ = 4 \\
0, XZ = 5 \\
6, XZ = 6 \\
\end{cases}
\]
\end{sol}
\end{problem}

\begin{problem}
Докажите, что $|EX|\le E|\E(X|\mathcal{H})|\le E|X|$.

\begin{sol}

\end{sol}
\end{problem}

\begin{problem}
Пусть $A$ — событие, и $X\in L^{1}$. Как связаны $\E(X|A)$,
$\E(X|A^{c})$ и $\E(X|1_{A})$?

\begin{sol}

\end{sol}
\end{problem}

\begin{problem}
Пусть $X$ и $Y$ независимы и одинаково распределены, $Z=X+Y$. Найдите $p_{X|Z}(x|z)$, $\E(X|Z)$, $\Var(X|Z)$ для случаев:
\begin{enumerate}
\item  $X$ — экспоненциально распределено с $\lambda$
\item $X$ — равномерно распределено на $[0;1]$
\end{enumerate}

\begin{sol}

\end{sol}
\end{problem}

\begin{problem}
Пусть $X$ имеет экспоненциальное распределение с параметром $\lambda$ \\
Найдите $\E(X|X^{2})$

\begin{sol}

\end{sol}
\end{problem}

\begin{problem}
Два игрока одновременно выбирают действительные числа  $x_{1} $  и
$x_{2} $, соответственно. Платежные функции имеют вид
 $\left(\begin{array}{l} {u_{1} } \\
 {u_{2} } \end{array}\right)=
 \left(\begin{array}{l} {-x_{1}^{2} +2x_{2} x_{1} +\theta _{1} x_{1} } \\
 {-x_{2}^{2} +4x_{1} x_{2} +2\theta _{2} x_{2} } \end{array}\right)$ ,
 где  $\theta _{1} \sim U\left[0;2\right]$ ,  $\theta _{2} \sim U\left[1;2\right]$
Значение  $\theta _{1} $  известно первому игроку, а значение
$\theta _{2} $  — второму. Найдите равновесие по Нэшу в чистых
стратегиях;

\begin{sol}

\end{sol}
\end{problem}

\begin{problem}
 Пусть $X$ и $Y$ — интегрируемые случайные величины, такие что
$\E(X|Y)=Y$ и $\E(Y|X)=X$. Верно ли, что $X=Y$ (as)?

\begin{sol}
Да, для $X$ и $Y$ из $L^2$ доказательство простое: $\E(X^2)=\E(Y^2)=\E(XY)$ и $\E((X-Y)^2)=0$. В общем случае нужно спускаться до событий и определения (?).
\url{http://math.stackexchange.com/questions/34101/}

\end{sol}
\end{problem}

\begin{problem}
Пусть $X$ и $Y$ — независимые случайные величины, равные 1 с
вероятностью $p\in(0;1)$ или 0. Пусть $Z=1_{X+Y=0}$. Найдите
$\E(X|Z)$.

\begin{sol}

\end{sol}
\end{problem}

\begin{problem}
 $[$важно!$]$ \\
 Пусть $X\in L^{1}$, $Y\in L^{1}$ и $\E(X)=\E(Y)$.
Рассмотрим все множества $A$, для которых $\E(X 1_{A})=\E(Y 1_{A})$.
Верно ли, что
они образуют $\sigma$-алгебру? \\
Почему для нахождения $\E(X|Y)$ достаточно проверять СЕ2
множествами вида $A=1_{Y\le t}$?

\begin{sol}

\end{sol}
\end{problem}

\begin{problem}
Пусть $f:\mathbb{R}\rightarrow [0;1]$ и $Y\sim
U[0;1]$. Найдите $\P(Y\le f(X)|X)$

\begin{sol}

\end{sol}
\end{problem}

\begin{problem}
 Pythagoras is back!

\begin{enumerate}
\item  Назовем условной дисперсией величину
$\Var(X|\mathcal{H})=\E(X^{2}|\mathcal{H})-\E(X|\mathcal{H})^{2}$ \\
Докажите, что
$\Var(X)=\E(\Var(X|\mathcal{H}))+\Var(\E(X|\mathcal{H}))$.
\item Пусть $S$ — сумма случайного количества слагаемых,
$S=\sum_{i=1}^{N}X_{i}$, где $N$ и $X_{i}$ независимы, $X_{i}$ — независимы и одинаково распределены.
Докажите, что $\E(S)=\E(N)\E(X_{i})$ и
$\Var(S)=\E(N)\Var(X_{i})+[\E(X_{i})]^{2}\Var(N)$.

Комментарий: чтобы прочувствовать последнюю формулу, представьте,
во что она превращается, если $X_{i}$ или $N$ — константа.
\end{enumerate}


\begin{sol}

\end{sol}
\end{problem}

\begin{problem}
Пусть $\mathcal{H}=\{\emptyset,\Omega,A,A^{c}\}$ и $0<\P(A)<1$. Что
представляет собой $\P(A|\mathcal{H})$ и $\P(B|\mathcal{H})$?

\begin{sol}

\end{sol}
\end{problem}

\begin{problem}
Пусть $\Omega=\{a,b,c\}$. Приведите пример, в котором
$\E(\E(X|\mathcal{H}_{1})|\mathcal{H}_{2})\neq
\E(\E(X|\mathcal{H}_{2})|\mathcal{H}_{1})$.

\begin{sol}

\end{sol}
\end{problem}

\begin{problem}
Пусть $\mathcal{H}_{1} \subseteq \mathcal{H}_{2}$. Докажите, что
условие $\E(X|\mathcal{H}_{1})=\E(X|\mathcal{H}_{2})$ равносильно
условию $\E(X|\mathcal{H}_{2})\in \mathcal{H}_{1}$.

\begin{sol}

\end{sol}
\end{problem}

\begin{problem}
Докажите условную :) формулу условной вероятности: \\
Если $A\in \mathcal{H}$ и $\P(B)>0$, то
$\P(A|B)=\frac{\E(1_{A}\P(B|\mathcal{H}))}{\E(\P(B|\mathcal{H}))}$.

\begin{sol}

\end{sol}
\end{problem}

\begin{problem}
Пусть $Z=\E(X|Y)$. Верно ли, что $\E(X|Y^{2})=\E(Z|Y^{2})$?

\begin{sol}

\end{sol}
\end{problem}

\begin{problem}
Пусть $A$ — событие (не обязательно из $\mathcal{H}$). Введем
обозначение $\P(A|\mathcal{H})=\E(1_A|\mathcal{H})$ по аналогии
c безусловным $\P(A)=\E(1_{A})$.
\begin{enumerate}
\item Найдите $\P(A|\mathcal{H})$ для случая $A\in\mathcal{H}$.
\item Пусть $A$ и $B$ независимы, $\P(A)=p$, $\P(B)=q$ и $Z=1_{A\cup B}$. Найдите $\P(A|Z)$.
\item Пусть $X$ и $Y$ — результаты двух подбрасываний правильного
кубика. Найдите $\P(X>Y|Y)$.
\end{enumerate}

\begin{sol}

\end{sol}
\end{problem}

\begin{problem}
Пусть $f_{X,Y}(x,y)=
\begin{cases}
  x+y & x,y \in [0;1] \\
  0 & \text{ иначе } \\
\end{cases}$.
\begin{enumerate}
\item Найдите $\E(X|Y)$.
\item Найдите $\hat{X}=aY+b$ так, чтобы $E[(X-\hat{X})^{2}]$ была
минимальной.
\end{enumerate}

\begin{sol}

\end{sol}
\end{problem}

\begin{problem}
\todo[inline]{какая-то фигня}
Величина $U\sim U[0;1]$, $X$ имеет распределения Рэлея: \\
$f_{X}(x)=
\begin{cases}
  \frac{x}{\sigma^{2}}e^{-x^{2}/2\sigma^{2}} & x\ge 0 \\
  0 & otherwise \\
\end{cases}$.
\begin{enumerate}
\item Найдите $\E(X|Y)$.
\item Найдите $\hat{X}=aY+b$ так, чтобы $E[(X-\hat{X})^{2}]$ была
минимальной.
\end{enumerate}

\begin{sol}

\end{sol}
\end{problem}

\begin{problem}
Пусть $X$ и $Y$ имеют совместное нормальное распределение.
Известны математические ожидания и ковариационная матрица.
\begin{enumerate}
\item  Найдите $\hat{X}=aY+b$ так, чтобы $E[(X-\hat{X})^{2}]$ была
минимальной.
\item Докажите, что $\E(X|Y)=\hat{X}$.
\end{enumerate}

\begin{sol}


b) $Cov(X-\hat{X},Y)=0 \Rightarrow $ независимость $X-\hat{X}$ и
$Y$ \\
$A\in\sigma{Y} \Rightarrow
E[(X-\hat{X})1_{A}]=\E(X-\hat{X})\E(1_{A})=0$
\end{sol}
\end{problem}

\begin{problem}
Пусть $X\in L^{2}$ и $Y\in L^{2}$, $X$ и $Y$ iid. \\
Найдите $\E(X|X+Y)$, $\E(X-Y|X+Y)$, $\E(X^{2}-Y^{2}|X+Y)$

\begin{sol}

$\E(X|X+Y)=0.5(X+Y)$, $\E(X-Y|X+Y)=0$, $\E(X^{2}-Y^{2}|X+Y)=0$
\end{sol}
\end{problem}

\begin{problem}
	Пусть $X_{i}$ — независимы и равномерны $U[0;1]$, $X_{min} = \min \{X_1, X_2, \ldots, X_n\}$, и $X_{max} = \max\{X_1, \ldots, X_n\}$.
\begin{enumerate}
	\item Найдите $\E(X_{1}| X_{min}$ и $\E(X_{1}| X_{max})$
\item Найдите $\E(X_1 | X_{min}, X_{max})$.
\item Найдите $\E(X_1^2 | X_{min})$.
\end{enumerate}


\begin{sol}

$p(x_{1}|\min)=\frac{n-1}{n(1-\min)}$ при $x_{1}>\min$ \\
$\E(X_{1}|\min\{X_{1},\ldots,X_{n}\})=\frac{(1+\min)(n-1)}{2n}$
\end{sol}
\end{problem}

\begin{problem}
Известно, что $Y=X^2$. Найдите $\E(X|Y)$ если
\begin{enumerate}
	\item $X\sim \cN(0; 1)$;
	\item $X\sim \cN(1; 1)$;
\end{enumerate}

\begin{sol}
0, 
\[
	\E(X|Y) = -\sqrt{Y} \frac{f(-\sqrt{Y})}{f(-\sqrt{y}) + f(\sqrt{Y})} + \sqrt{Y} \frac{f(\sqrt{Y})}{f(-\sqrt{y}) + f(\sqrt{Y})} = \sqrt{Y} \frac{\exp(2\sqrt{Y}) - 1}{\exp(2\sqrt{Y})+1} 
\]
\end{sol}
\end{problem}

\begin{problem}
Пусть $X$ и $Y$ имеют функцию плотности
$p(x,y)=\frac{1}{2\pi}exp\left\{-\frac{1}{2}\left(x^{2}+y^{2}\right)\right\}$.\\
Пусть $Z=\sqrt{X+Y}$. Найдите $\E(X|Z)$.

\begin{sol}

Рассмотрим величину $F$ такую, что $X=Z\cos(F)$, $Y=Z\sin(F)$.
\end{sol}
\end{problem}

\begin{problem}
Автобусы приходят на остановку через случайные промежутки времени и представляют собой пуассоновский поток с параметром $\lambda$. В первый день Вася приходит на остановку и замеряет время до первого автобуса. Пусть это время $X$. На следующий день Вася приходит на остановку и считает, сколько автобусов придет в течении времени $X$. Он получает количество автобусов $N$.
\begin{enumerate}
\item Найдите $\E(N)$ и $\Var(N)$.
\item Найдите $\P(N>0)$.
\end{enumerate}


\begin{sol}
$\E(N)=1$, $\Var(N)=2$, $\P(N>0)=0.5$
\end{sol}
\end{problem}

\begin{problem}
Let $X_1, X_2, \ldots X_n$ be indepdently and identically distribuetd random variables, strictly positive. \\
Show that $\E(S_{m}|S_{n})= m/n$ for $m\le n$ , where $S_{m} = X_{1} + X_{2} + \ldots + X_{m}$

\begin{sol}

\end{sol}
\end{problem}

\begin{problem}
Монетка выпадает орлом с вероятностью $p$. Эксперимент состоит из двух этапов. На первом этапе монетку подкидывают 100 раз и записывают число орлов. На втором этапе монетку подбрасывают до тех пор, пока не выпадет столько орлов, сколько выпало на первом этапе. Обозначим число подбрасываний монетки на втором этапе буквой $X$. \\
Найдите $\E(X)$, $\Var(X)$

\begin{sol}

\end{sol}
\end{problem}

\begin{problem}
Рассмотрим вероятностное пространство
$(\Omega, \mathcal{B}[0;1], \lambda)$, где $\lambda$ — классическая
мера Лебега.\\
На этом пространстве введем величины: \\
$X(\omega)=1_{\omega\in[0;0.5)}$,
$Y(\omega)=1_{\omega\in[0;0.75)}$,
$Z(\omega)=1_{\omega\in[0.25;0.75)}$.
\begin{enumerate}
\item  Верно ли, что $X$ и $Z$ независимы?
\item Найдите $\E(X|Y)$ и $\E(X|Y,Z)$.
\end{enumerate}

\begin{sol}

\end{sol}
\end{problem}

\begin{problem}
Пусть $X$ и $Z$ независимы и стандартно нормально распределены;
$Y=\frac{Z}{1+X^{2}}$.
\begin{enumerate}
\item  Верно ли, что $X$ и $Y$ независимы?
\item Верно ли, что $\E(Y|X)=\E(Y)$?
\end{enumerate}

\begin{sol}

\end{sol}
\end{problem}

\begin{problem}
 Let $X$ be an exponential random variable, so that $\P[X > t] = e^{-\lambda t}$
	for $t > 0$, with $\lambda > 0$. 	Let $T$ be a fixed value, and then compute
	$E[ X | \max(X,T) ]$ and $E[X | \min(X,T)]$.

\begin{sol}

\end{sol}
\end{problem}

\begin{problem}
 Пусть совместное распределение $X$ и $Y$ задано таблицей: \\
\begin{tabular}{ccc}
  \toprule
  % after \\: \hline or \cline{col1-col2} \cline{col3-col4} \ldots
   & $X=-1$ & $X=1$ \\
  $Y=-1$ & 1/8 & 4/8 \\
  $Y=2$ & 2/8 & 1/8 \\
  \bottomrule
\end{tabular}
\begin{enumerate}
\item Найдите $\E(X|Y)$, представьте ответ в виде $\E(X|Y)=a+bY$.
\item Убедитесь, что $\E(\E(X|Y))=\E(X)$.
\item Найдите $\E(XY|Y)$ и представьте ответ в виде $f(Y)$.
\end{enumerate}

\begin{sol}

\end{sol}
\end{problem}

\begin{problem}
\begin{enumerate}
\item  Что больше $\Var(X)$ или $\Var(\E(X|\mathcal{H}))$?
\item Всегда ли $\Var(X)$ больше $\Var(X|\mathcal{H})$?
\end{enumerate}

\begin{sol}

 а) $\Var(X)$ больше (можно разложить $\Var(X)$)
\end{sol}
\end{problem}

\begin{problem}
Пусть $X$ равномерно на $[10;20]$, а $Y$ распределено экспоненциально с параметром $\lambda=X$.
\begin{enumerate}
\item Найдите $\E(Y)$, $\Var(Y)$.
\item Сравните с $\E(Y)$ и $\Var(Y)$ для экспоненциального распределения с $\lambda=15$.
\end{enumerate}

\begin{sol}

\end{sol}
\end{problem}

\begin{problem}
Пусть клиенты приходят пуассоновским потоком с интенсивностью $\lambda_{in}$. А время обслуживания одного клиента экспоненциально с параметром $\lambda_{service}$. \\
Пусть $X$ — число клиентов, пришедших за время обслуживания Васи Сидорова. \\
Как распределено $X$?

\begin{sol}

$\P(X=k)=\E(1_{X=k})=\E(\E(1_{X=k}|T))=\int_{0}^{\infty}e^{-lambda_{in}t}\frac{(\lambda_{in}t)^{k}}{k!}\lambda_{service}e^{-\lambda_{service}t}dt=\ldots$ \\
Ответ: геометрически, $\alpha=\frac{\lambda_{service}}{\lambda_{in}}$, $\P(X=k)=\frac{\alpha}{(1+\alpha)^{k+1}}$
\end{sol}
\end{problem}

\begin{problem}
Пусть $X$ распределено экспоненциально с параметром $\lambda$, $Y$ — распределено экспоненциально с параметром $X$. Найдите $\E(X)$, $\E(Y)$, $\E(Y|X)$, $\E(X|Y)$

\begin{sol}
 $p(y)=\frac{\lambda}{(\lambda+y)^{2}}$, $\E(Y|X)=1/X$, $\E(X|Y)=\frac{2}{\lambda+Y}$
\end{sol}
\end{problem}

\begin{problem}
Верно ли, что $Cov(X,\E(Y|X))=Cov(X,Y)$?

\begin{sol}

\end{sol}
\end{problem}

\begin{problem}
Приведите пример:
\begin{enumerate}
\item $X$ и $Y$ зависимы, но $\E(XY)=\E(X)\E(Y)$
\item $X$ и $Y$ зависимы, но $\E(Y|X)=\E(Y)$ (as)
\item $\E(XY)=\E(X)\E(Y)$, но $\E(Y|X)\neq \E(Y)$ (as)
\end{enumerate}

\begin{sol}

\end{sol}
\end{problem}

\begin{problem}
Докажите, что если $\E(Y|X)=\E(Y)$ (as), то $\E(XY)=\E(X)\E(Y)$

\begin{sol}

\end{sol}
\end{problem}

\begin{problem}
Известно, что $\E(X|Y)=Y$, $\Var(X|Y)=1$, $\E(Y)=3$, $\Var(Y)=10$. \\
Найдите: $\E(X)$, $\E(X^{2}|Y)$, $\Var(X)$, $Cov(X,Y)$

\begin{sol}

\end{sol}
\end{problem}

\begin{problem}
Find conditional expectation $\E(X|Y)$ intuitively (without formal proof)  \\
$Z\sim U[0;1]$, $A=\{Z>0.5\}$, $X=2Z^{2}$, $Y=2Z-1_{A}$.

\begin{sol}

\end{sol}
\end{problem}

\begin{problem}
We toss a fair coin 3 times. Let $X$ be the total number of heads and $Y$ — the total number of tails. $A$ — is the event that there is at least one head and at least one tail.
\begin{enumerate}
\item Find explicitely $\sigma(A)$, $\sigma(X\cdot Y)$.
\item How many elements there in $\sigma(X)$?
\item Find explicitely $\E(X|\sigma(Y))$, $\E(X|\sigma(XY))$, $\E(X|\sigma(A))$.
\item Find explicitely $\E(Y|\sigma(XY))$, $\E(1_{A}|\sigma(XY))$, $\E(1_{A}|\sigma(A))$.
\end{enumerate}


\begin{sol}

\end{sol}
\end{problem}

\begin{problem}
Верны ли следующие утверждения?
\begin{enumerate}
  \item Если величины $X$ и $Y$ независимы, то $\E(Y|X)=\E(Y)$.
  \item Если $\E(Y|X)=\E(Y)$, то величины $X$ и $Y$ независимы.
  \item Если величины $X$ и $Y$ некоррелированы, то $\E(Y|X)=\E(Y)$.
  \item Если $\E(Y|X)=\E(Y)$, то величины $X$ и $Y$ некоррелированы.
\end{enumerate}

Источник: Алексей Суздальцев

\begin{sol}
 первое и четвертое верны, а второе и третье — нет.
\end{sol}
\end{problem}

\begin{problem}
 Вася случайно выбирает между 0 и 1 число $X_1$, затем случайно выбирает между 0 и $X_1$ число $X_2$, затем $X_3$ между 0 и $X_{2}$, и так до бесконечности.
\begin{enumerate}
  \item Найдите $\E(X_n)$, $\Var(X_n)$;
  \item Найдите функцию плотности распределения $X_n$;
  \item Найдите $\E(X_2|X_1,X_3)$;
\item К какой случайной величине стремится $X_n$ и в каких смыслах?
\end{enumerate}

Источник: Алексей Суздальцев

\begin{sol}

$\E(X_n)=\frac{1}{2^n}$; $V(X_n)=\frac{1}{3^n}-\frac{1}{4^n}$. $f_{X_n}(x)=\frac{(-\ln{x})^{n-1}}{(n-1)!}\cdot 1_{[0;1]}.$\\
$\E(X_2|X_1,X_3)=\frac{X_1-X_3}{\ln{X_1}-\ln{X_3}}.$\\
При любом $\omega$ последовательность $X_n(\omega)$ монотонно невозрастающая и ограниченная (нулем). Значит, она сходится. При этом
\[
 \P(\forall i \mbox{ } X_i>\epsilon)<\P(X_k>\epsilon)<\frac{\E(X_k)}{\epsilon}=\frac{1}{2^k\epsilon}
 \]
 для любого $k$. Значит, $\P(\forall i \; X_i>\epsilon)$ меньше любого положительного числа, то есть $\P(\forall i \; X_i>\epsilon)=0$. Значит, с вероятностью 1 $X_n(\omega)$ сходится именно к нулю, то есть $X_n\to 0$ почти наверное. Отсюда следуют сходимости по вероятности и по распределению. Поскольку $\lim\limits_{n\to \infty}\E(X_n)=0$, а $0<\E(X_n^p)<\E(X_n)$ для $\forall p>1$, то имеет место также и сходимость в $L^p$. Итак, данная последовательность стремится к нулю во всех разумных смыслах.
\end{sol}
\end{problem}

\begin{problem}
Маша собрала $n$ грибов в лесу наугад. В лесу есть рыжики, мухоморы и лисички. Рыжики попадаются с вероятностью $r>0$, лисички "--- с вероятностью $l>0$,  мухоморы "--- с вероятностью $m>0$, $r+m+l=1$. Пусть $R$ "--- количество собранных рыжиков, $L$ "--- лисичек, а $M$ "--- мухоморов. Найдите:
\begin{enumerate}
\item $\E(R+L|M)$, $\E(M|R+L)$
\item $\E(R|L)$
\item $\Var(R|L)$
\item $\E(R+L|L+M)$
\item $\E(R|R-L)$ (? похоже, не решается в явном виде)
\item $\Var(R|L)$
\item $\P(\E(R|L)=0)$
\item $\P(R=0 | L)$
\item $\E \left( \left( \frac{m}{r+m} \right)^{100-L} \right)$
\end{enumerate}

\begin{sol}
$\E(R+L|M)=n-M$, $\E(M|R+L)=n-R-L$, $\E(R|L)=(n-L)\frac{r}{1-l}$, $\Var(R|L)=(n-L)\frac{r}{1-l}\frac{1-r-l}{1-l}$, $\P(\E(R|L)=0)=l^{n}$, $\P(R=0 | L)=\left(\frac{r}{r+m}\right)^{n-L}$
\end{sol}
\end{problem}

\begin{problem}
 Пусть $X$ и $Y$ — независимые случайные величины, $X$ — $Bin(n_{1},p)$, $Y$ — $Bin(n_{2},p)$. Найдите $\E(X|X+Y)$.

\begin{sol}
$\E(X|X+Y)=\frac{n_{1}}{n_{1}+n_{2}}(X+Y)$
\end{sol}
\end{problem}

\begin{problem}
 Пусть $X$ и $Y$ — независимые пуассоновские случайные величины с параметрами $\lambda_{1}$ и $\lambda_{2}$. Найдите $\E(X|X+Y)$.

\begin{sol}
$\E(X|X+Y)=\frac{\lambda_{1}}{\lambda_{1}+\lambda_{2}}(X+Y)$
\end{sol}
\end{problem}

\begin{problem}
Вася, Петя и Коля играют в карточного «дурака» втроём. Вася проигрывает с вероятностью $p_1$, Петя — с вероятностью $p_2$, Коля — с вероятностью $p_3$. Естественно, $p_1+p_2+p_3=1$. Всего они сыграли $n$ партий. Обозначим количества проигранных ими партий $X_1$, $X_2$ и $X_3$, соответственно. Найдите $\E(X_1|X_1+X_2)$. Может получится и $\Var(X_1|X_1+X_2)$?

\begin{sol}
Раскладываем $X_1$ в сумму $n$ индикаторов, получаем $\E(X_1|X_1+X_2)=\ldots=\frac{p_1}{p_1+p_2}(X_1+X_2)$
\end{sol}
\end{problem}

\begin{problem}
Пусть $X_{1}\sim U[0;1]$, $X_{2}\sim U[-1;1]$, $X_{3}\sim U[-1;2]$. И $Y_{i}=X_{i}^{2}$.

Найдите $ \E(X_{1}|Y_{1}) $, $ \E(X_{2}|Y_{2}) $ и $ \E(X_{3}|Y_{3}) $.

\begin{sol}
 $Y_{1}$, $0$, $Y_{3}1_{Y_{3}>1}$
\end{sol}
\end{problem}

\begin{problem}
Let $ X $ be uniform on $[0;1]$. We define $ A:=\{X>0.1\} $ and $ Y:=X^{2} $. Find $ \E(Y|\sigma(1_{A})) $, $\E(1_{A}|\sigma(Y))$ and $\E(1_{A}+Y|Y-1_{A})$.

\begin{sol}

\end{sol}
\end{problem}

\begin{problem}
Какая $\sigma$-алгебра в общем случае больше, $\sigma(X)$ или $\sigma(X^2)$? Приведите пример для случая $\sigma(X)=\sigma(X^2)$.

\begin{sol}
$\sigma(X) \subset \sigma(X^2)$, величина $X$, принимающая неотрицательные значения.
\end{sol}
\end{problem}

\begin{problem}
Пусть $ X_{1} $, \ldots, $ X_{100} $ независимы и равномерны на $ [0;1] $. Пусть $ L=\max\{X_{1},X_{2},\ldots, X_{80}\} $ а $ R=\max\{X_{81},X_{82},\ldots,X_{100}\} $ и $ M=\max\{X_{1},\ldots,X_{100}\} $

Найдите
\begin{enumerate}
\item $ \P(L>R|L)$ и $ \P(L>R|R) $ и $ \P(L>R|M)$, $\P(L>R|L,M) $
\item $\E( X_1 | L )$, $\E( X_1 | \min \{ X_1, \ldots, X_{100} \})$
\item $\E( \min \{ X_1, \ldots, X_{100} \} | \max \{ X_1, \ldots, X_{100} \})$
\item $\E(\min \{ X_1, \ldots, X_{100} \} | X_1)$
\end{enumerate}


\begin{sol}
$ \P(L>R|L)=L^{20}$,  $ \P(L>R|R)=1-R^{80} $, $ \P(L>R|M)=\frac{80}{100}$, $\P(L>R|L,M)=1-1_{L>M} $

$\E(\min \{ X_1, \ldots, X_{100} \} | X_1)$ сводится к подсчету $E(1/(N+1))$ для биномиальной случайной величины.
\end{sol}
\end{problem}

\begin{problem}
Пусть $ X_{1} $, \ldots, $ X_{n} $ — независимы и равномерны на $ [0;1] $, и $ M=max\{X_{1},\ldots,X_{n}\} $. Найдите $\E(M^{k})$ в уме без вычислений. Подсказка: $ M=\P(M>X_{n+1}|M) $


\begin{sol}
Используя подсказку получаем: $ M^{k}=\P(M>X_{n+1} \cap M>X_{n+2} \cap \ldots \cap M>X_{n+k}|M) $. Значит $ \E(M^{k})=\P($максимум из $ (n+k) $ равномерных величин лежит среди $ n $ первых$)=\frac{n}{n+k} $
\end{sol}
\end{problem}

\begin{problem}
Известно, что $ Y $ и $ X=\E(Y|\mathcal{H}) $ одинаково распределены. Верно ли, что $ Y=X $ почти наверное? Hint: start by comparing sets $ Y>0 $ and $ X>0 $

\begin{sol}
Во-первых, интуитивно: наше условное мат. ожидание не сделало $ Y $ более «грубым», значит не должно было совсем изменить, то есть ответ «верно». Во-вторых, формально. Т.к. $ X $ и $ Y $ одинаково распределены:

\[ \E(1_{Y>0}Y)=\E(\max\{Y,0\})=\E(\max\{X,0\})=\E(1_{X>0}X) \]

По определению условного математического ожидания:

\[ \E(1_{X>0}X)=\E(1_{X>0}Y)=\E(1_{X>0\cap Y>0} Y)+\E(1_{X>0\backslash Y>0} Y)\leq \E(1_{ Y>0} Y)+\E(1_{X>0\backslash Y>0} Y)\]

Получаем, что $ \E(1_{Y>0}Y) \leq \E(1_{ Y>0} Y)+\E(1_{X>0\backslash Y>0} Y) $. Это возможно только если $ \E(1_{X>0\backslash Y>0} Y)\geq 0 $. Но $ 1_{X>0\backslash Y>0} Y \leq 0$. Значит $ \P(X>0 \backslash Y>0)=0 $.

Рассмотрев дополнительно $ X<0 $ и $ Y<0 $ мы получаем, что $ \P(X>0 \Delta Y>0)=0 $.

Сравнив аналогичным образом $ (X-c) $ и $ (Y-c) $ мы получаем, что $ \P(X>c \Delta Y>c)=0 $.


\end{sol}
\end{problem}

\begin{problem}
Верно ли, что $\E(X|Y^{2})=\E(X|\,|Y|)$?

\begin{sol}
Да, так как $\sigma(Y^{2})=\sigma(|Y|)$
\end{sol}
\end{problem}

\begin{problem}
Величины $X_1$, $X_2$, \ldots независимы и одинаково распределены. Мы складываем случайное количество $N$ слагаемых. Обозначим сумму буквой $S=\sum_{i=1}^{N}X_i$. Найдите $\E(S)$, $\Var(S)$ и $Cov(S,N)$

\begin{sol}
 $Cov(S,N)=\E(SN)-\E(S)\E(N)=\E(\E(SN|N))-\E(\E(S|N))\E(N)=\ldots$
\end{sol}
\end{problem}

\begin{problem}
Неправильный кубик выпадает с вероятностью $0{,}5$ шестеркой вверх. Остальные пять граней выпадают равновероятно. Случайная величина $X$ -- остаток от деления номера грани на два, $Y$ -- остаток от деления номера грани на три. Найдите
\begin{enumerate}
\item Закон распределения $\E(X\mid Y)$, $\E(Y\mid X)$
\item Выразите $\E(Y\mid X)$ через $X$, а $\E(X\mid Y)$ через $Y$
\item Найдите $Cov(\E(Y\mid X),\E(X\mid Y))$, $Cov(\E(Y\mid X),X)$, $Cov(Y,X)$
\end{enumerate}

\begin{sol}

\end{sol}
\end{problem}

\begin{problem}
Случайная величина $X$ принимает значения в диапазоне $[0;1]$. Всегда ли существует случайная величина $Y$, принимающая всего два значения, такая, что $E(Y|X)=X$? Такая, что $E(X|Y)=Y$?

\begin{sol}
\begin{enumerate}
\item рассмотрите величину $Y$, принимающую значения 0 и 1
\item нет, если $X$ принимает больше значений, чем $Y$
\end{enumerate}
\end{sol}
\end{problem}

\begin{problem}
Цена литра молока, $X$, распределена равномерно на отрезке $[1;2]$. Количество молока, которое дает корова Мурка, $Y$, распределено экспоненциально с $\lambda=1$. Надои не зависят от цены. Величина $Z$ — выручка кота Матроскина от продажи всего объема молока.

Найдите
\begin{enumerate}
\item $E(Z|X)$, $\Var(Z|X)$, корреляцию $Z$ и $X$
\item Закон распределения $E(Z|X)$
\item Функцию плотности величины $\Var(Z|X)$
\end{enumerate}

\begin{sol}
 $E(X)=1$, $\Var(X)=1$, $E(X^2)=\Var(X)+E^2(X)=2$;

$E(Y)=3/2$, $\Var(Y)=1/12$, $E(Y^2)=\Var(Y)+E^2(Y)=7/3$;

\begin{equation}
E(Z|X)=E(XY|X)=XE(Y|X)=XE(Y)=1.5X
\end{equation}

\begin{equation}
\Var(Z|X)=\Var(XY|X)=X^2\Var(Y|X)=X^2\Var(Y)=\frac{X^2}{12}
\end{equation}

\begin{multline}
Cov(X,Z)=Cov(X,XY)=E(X^2Y)-E(X)E(XY)=\\
=E(X^2)E(Y)-E(X)E(X)E(Y)=E(Y)\Var(X)=\frac{3}{2}
\end{multline}

\begin{equation}
\Var(Z)=E(X^2Y^2)-E(X)^2E(Y)^2=2\cdot \frac{7}{3}-1^2\left(\frac{3}{2}\right)^2=\frac{29}{12}
\end{equation}

\begin{equation}
Corr(X,Z)=\frac{3}{2}\cdot \frac{1}{\sqrt{1\cdot \frac{29}{12}}}=\frac{3\sqrt{3}}{\sqrt{29}}
\end{equation}

Величина $W=E(Z|X)$ равномерна на $[1.5;3]$

Заметим, что $V=X^{2}/12$ и $X=2\sqrt{3}\sqrt{V}$.
\begin{equation}
\frac{dx}{dv}=\frac{\sqrt{3}}{\sqrt{v}}
\end{equation}

Функция плотности
\begin{equation}
p(v)=p_{X}\left(2\sqrt{3v}\right)\cdot \frac{\sqrt{3}}{\sqrt{v}}=\exp\left(-2\sqrt{3v}\right)\frac{\sqrt{3}}{\sqrt{v}}
\end{equation}
\end{sol}
\end{problem}

\begin{problem}
Кубик подбрасывают бесконечное количество раз. Величина $X$ — номер подбрасывания, когда впервые выпала единица, а $Y$ — номер подбрасывания, когда впервые выпала шестерка. Найдите $\E(Y|X)$.

\begin{sol}

\end{sol}
\end{problem}

\begin{problem}
Верно ли, что $\E(XY)=\E(\E(X\mid Z)\cdot \E(Y\mid X))$, если левая и правая части определены?

\begin{sol}
Нет, например, возьмем независимые одинаково распределенные $X$ и $Y$ и $Z=(X+Y)/2$.
\end{sol}
\end{problem}

\begin{problem}
The random variables $X_1$, $X_2$, \ldots are independent uniformly distributed on $[0;1]$. I~am summing them until the first $X_i$ greater than $0.5$ is added. After this term I stop. Let's denote by $S$ the total sum and by $N$ — the number of terms added. Find $\E(S|N)$, $\Var(S|N)$, $\E(S)$, $\Var(S)$

Котёнок с улицы Лизюкова ловит карасей до тех пор, пока не поймает карася длиной более полуметра. Длины карасей независимы и равномерны от 0 до 1 метра. Обозначим буквой $N$ количество пойманных карасей, а буквой $S$ — их суммарную длину. Найдите $\E(S|N)$, $\Var(S|N)$, $\E(S)$, $\Var(S)$.

\begin{sol}
\[
\E(S|N)= \frac{0.5}{2} (N-1) + \frac{1.5}{2} = \frac{N}{4}+\frac{1}{2}
\]

Случайные величины $X_i$ условно независимы при фиксированном $N$, поэтому
\[
\Var(S|N) = (N-1)\frac{0.5^2}{12} + \frac{0.5^2}{12} = N/48
\]

Обратим внимание, что $N$ имеет геометрическое распределение:
\[
\E(S)=\E(\E(S|N))=\frac{1}{4}\E(N)+\frac{1}{2}=1
\]

Переходим к дисперсии:
\[
\Var(S)=\Var(\E(S|N)) + \E(\Var(S|N)) = 1/6
\]


\end{sol}
\end{problem}

\begin{problem}
 Правильная монетка подбрасывается 3 раза. Случайная величина $X$ — количество выпавших орлов. Событие $A$ состоит в том, что в первый и третий раз результаты подбрасывания совпали. Событие $B$ — при последнем броске выпала решка.
\begin{enumerate}
\item Найдите $\sigma(A)$.
\item Сколько элементов в $\sigma(X)$?
\item Найдите $\E(X\mid A)$, $\E(X\mid B)$.
\item Найдите $\E(X|\sigma(A))$, $\E(X|\sigma(B))$ и $\E(1_A \mid X)$.
\end{enumerate}

\begin{sol}
В $\sigma(A)$ четыре события, $\sigma(A)=\{\Omega, \emptyset, A, A^c\}$. В $\sigma(X)$ 16 событий. Оказывается, что $\E(X\mid A)=\E(X\mid \sigma(A))=1.5$.
\end{sol}
\end{problem}

\begin{problem}
Найдите $\E(Y|X)$ и $\E(Y^3|X)$ если $X$ и $Y$ стандартные нормальные в совокупности с $\Corr(X,Y)=\rho$.

\begin{sol}

\end{sol}
\end{problem}

\begin{problem}
Известно, что $Y=X\cos(X)+Z$, где $X\sim N(0,10)$, $Z\sim U[0;30]$ и $X$ и $Z$ независимы. Найдите $\E(Y|X)$, $\E(Y|Z)$

\begin{sol}
$\E(Y|X)=X\cos(X)+15$, $\E(Y|Z)=Z$
\end{sol}
\end{problem}

\begin{problem}
Известно, что $X=(Z_1+Z_2)^2+Z_3$ и $Y=(Z_1+Z_2)^3+Z_3$, величины $Z_i$ независимы, $Z_1\sim U[0;2]$, $Z_2\sim N(1,4)$, $Z_3\sim N(-2,9)$. Найдите
\begin{enumerate}
\item $\E(X \mid Z_1)$, $\E(X \mid Z_2)$, $\E(X \mid Z_3)$
\item $\E(Y \mid Z_1)$, $\E(Y \mid Z_2)$, $\E(Y \mid Z_3)$
\end{enumerate}


\begin{sol}
\end{sol}
\end{problem}

\begin{problem}
Величины $X$ и $Y$ имеют совместную функцию плотности
\[
f(x, y) = \begin{cases}
x+y, \text{ если } x, y \in [0; 1], \\
0, \text{ иначе } \\
\end{cases}.
\]
Найдите $f(x)$, $f(y|x)$, $\E(Y|X)$.
\begin{sol}
  $f(x)=x+0.5$ при $x\in[0; 1]$, $f(y|x) = \frac{x+y}{x+0.5}$, $\E(Y|X) = \int_0^1 y \cdot f(y|X) \, dy = \frac{3X+2}{6X+3}$.
\end{sol}
\end{problem}


\begin{problem}
  Если $X$ имеет нормальное распределение $\cN(\mu;\sigma^2)$, а фукнция $g(t)$ не слишком быстро растёт при $t\to\infty$, то $\E((X-\mu)g(X))=\sigma^2\E(g'(X))$. Доказывать это равенство не требуется, но если интересно, то оно доказывается путём интегрирования по частям левой части. Исходя из данного равенства докажите, что для довольно произвольных $Y_1$, $Y_2$, \ldots, $Y_n$ и $X\sim \cN(\mu;\sigma^2)$ и функции $f$ выполнено равенство:
  \[
   \E((X-\mu)f(X, Y_1, \ldots, Y_n))=\sigma^2 \E(f'_x(X, Y_1, \ldots, Y_n)).
  \]
\begin{sol}
\end{sol}
\end{problem}


\begin{problem}
Храбрая исследовательница Василиса подбрасывает правильную монетку до выпадений двух орлов подряд. Пусть $X$ — это количество орлов, $Y$ — решек, а $Z$ — всего подбрасываний. 
\begin{enumerate}
\item Найдите $\E(X|Y=3)$;
\item Найдите $\E(X|Y)$, $\E(Y|X)$, $\E(X|Z)$, $\E(Z|X)$, $\E(Y|Z)$ и $\E(Z|Y)$;
\item Хм, а $\E(X^2|Y)$?
\end{enumerate}
\begin{sol}
\[
	\E(X|Y) = 2 + \E(T_1 + T_2 + \ldots + T_Y | Y),
\]
где $T_i$ равна 1, если непосредственно до $i$-ой решки был орёл. Все $\E(T_i |Y)$ равны. Находим $\E(T_1 |Y)$. Каждому варианту исходу, начинающемуся с Р соответствует ровно один исход начинающийся с ОР, но в два раза менее вероятный. Отсюда $\E(T_1 | Y) = 1/3$ и $\E(X|Y) = 2 + Y/3$.
\end{sol}
\end{problem}

\section{Разложение в сумму, первый шаг, цепи Маркова}

% Разложение в сумму, Первый шаг, Принцип отражения и случайное блуждание
% Берем из элементарного



\begin{problem}
  Information on buy-backs is adapted from investorwords.com.
  This problem suggests how results on biased random walks
  can be worked into more realistic models.

  Consider a naive model for a stock that has a support level
  of \$20/share because of a corporate buy-back program.
  (This means the company will buy back stock if shares dip
  below \$20 per share.   In the case of stocks, this reduces
  the number of shares outstanding, giving each remaining
  shareholder a larger percentage ownership of the
  company. This is usually considered a sign that the
  company's management is optimistic about the future and
  believes that the current share price is
  undervalued. Reasons for buy-backs include putting unused
  cash to use, raising earnings per share, increasing internal
  control of the company, and obtaining stock for employee
  stock option plans or pension plans.)   Suppose also that the
  stock price moves randomly with a downward bias when the
  price is above \$20, and randomly with an upward bias when
  the price is below \$20.  To make the problem concrete, we
  let $Y_n$ denote the stock price at time $n$, and we express
  our stock support hypothesis by the assumptions that
  \begin{eqnarray*}
   \P[ Y_{n+1} = 21 | Y_{n} = 20] &=& 9/10 \\
   \P[ Y_{n+1} = 19 | Y_{n} = 20] &=& 1/10
  \end{eqnarray*}

  We then reflect the downward bias at price levels above
  \$20 by requiring that for $k > 20$:
  \begin{eqnarray*}
   \P[ Y_{n+1} = k+1 | Y_{n} = k ] &=& 1/3 \\
   \P[ Y_{n+1} = k-1 | Y_{n} = k ] &=& 2/3.
  \end{eqnarray*}

  We then reflect the upward bias at price levels below \$20
  by requiring that for $k < 20$:
  \begin{eqnarray*}
   \P[ Y_{n+1} = k+1 | Y_{n} = k ] &=& 2/3 \\
   \P[ Y_{n+1} = k-1 | Y_{n} = k ] &=& 1/3
  \end{eqnarray*}

  Using the methods of ``single-step analysis'' calculate the
  expected time for the stock to fall from \$25 through the
  support level all the way down to \$18.
  (I don't believe that there is any way to solve this problem
  using formulas.  Instead you will have to go back to basic
  principles of single-step or first-step analysis to solve
  the problem.)

\begin{sol}

The first thing to notice is that there is no natural upper
boundary for this problem. In effect, the stock situation is
like playing against an infinitely rich adversary. Therefore,
we will set a temporary
artificial boundary at $25+M$, where $M$ is some
positive integer, and then at the end of the problem, we will
let $M$ go to infinity. Let $D_z$ be the duration of the game,
that is, the expected number of moves until the stock goes
from price $25$ to price $18$. Second, we note that the
conditioning by expectation, or one-step analysis, give the
following set of equations:
\begin{eqnarray*}
 D_{25} &=& (1/3) D_{26} + (2/3) D_{24} + 1  \\
 D_{24} &=& (1/3) D_{25} + (2/3) D_{23} + 1  \\
 D_{23} &=& (1/3) D_{24} + (2/3) D_{22} + 1  \\
 D_{22} &=& (1/3) D_{23} + (2/3) D_{21} + 1  \\
 D_{21} &=& (1/3) D_{22} + (2/3) D_{20} + 1  \\
 D_{20} &=& (1/10) D_{19} + (9/10) D_{21} + 1  \\
 D_{19} &=& (2/3) D_{20} + (1/3) D_{18} + 1  \\
 D_{18} &=& 0
\end{eqnarray*}
Еще одно уравнение! (Или пара)
\end{sol}
\end{problem}

\begin{problem}
Пусть $S_{n}$ — симметричное случайное блуждание с началом в нуле
и $\tau$ — момент первого достижения суммы в 1 рубль. Найдите
$\E(S_{n})$, $\E(S_{n\wedge \tau})$ и $\E(S_{\tau})$.

\begin{sol}

\end{sol}
\end{problem}

\begin{problem}
Допустим, что игрок играет либо до достижения некоторой суммы либо
до полного проигрыша. Обозначим $p$ вероятность выигрыша в отдельной партии. \\
Прокомментируйте следующее утверждение: \\
Ставку выгодно удвоить, если $p<\frac{1}{2}$ и выгодно снизить в
два раза, если $p>\frac{1}{2}$.

\begin{sol}

\end{sol}
\end{problem}

\begin{problem}
Сколько существует маршрутов движения частицы, начинающихся в нуле
($S_{0}=0$), заканчивающихся в нуле при $t=10$ ($S_{10}=0$) и не
посещающих ноль между этими моментами времени?

\begin{sol}

\end{sol}
\end{problem}

\begin{problem}
Пусть $S_{n}$ — случайное блуждание, $S_{n}-S_{n-1}=
\begin{cases}
  1, & p \\
  -1, & 1-p \\
\end{cases}$. \\
Пусть $n$ — четное число и $t\le n$. Зависит ли
$\P(S_{t}=k|S_{n}=0)$ от $p$?

\begin{sol}

\end{sol}
\end{problem}

\begin{problem}
 The following example is a classic in elementary probability theory called
the \emph{Ballot Problem.}  In an election, candidate A receives $n$
votes and candidate B receives $m$ votes, where $m > n$, so A is the
winner.  Assuming that all orderings of vote countings are equally
likely, prove that the probability that A is always ahead in the count of votes
is $(n-m)/(n+m)$

\begin{sol}

\end{sol}
\end{problem}

\begin{problem}
   Consider a coin-flipping game, where the probability of a head is
  $p$ and the probability of a tail is $1-p$.  What is the probability
  that for the first time on flip $2n$ after beginning, the total number of
  heads is the same as the total number of tails?  Since the number of
  heads is the same as the number of tails, the total number of coin-flips
  is double the number of heads, hence an even number.  Note that we
  do ask that heads be ahead of tails until flip $2n$ or that tails be
  ahead of heads, only that they first equalize at flip $2n$

  Equivalently, consider a gambler starting from an initial stake $X_0$.
  The gambler wins a dollar when the coin flip is heads with
  probability $p$.  The gambler loses a dollar when the coin flip is
  tails with probability $1-p$. Let the outcome of the $i$th flip be
  \[
      X =
      \begin{cases}
	1  &  \text{with probability $p$} \\
        -1 &  \text{with probability $1-p$}
      \end{cases}
   \]
   Then the gambler's fortune at stage $n$ is $X_0 + \sum_{i-1}^n
   X_i.$ Define the probability that the gambler's fortune is again $X_0$
   for the first time at step $2n$?  We do not care if the gambler has been
   ahead or behind, only that he gambler's fortune again is his starting value
   at flip $2n$.

\begin{sol}

\end{sol}
\end{problem}

\begin{problem}
 In a random walk starting at the origin find the probability
    that the point $a > 0$ will be reached before the point $-b < 0$.

\begin{sol}

Let the probability that $X_i = +1$ be $p$, that is, a step to the
right occurs with probability $p$.
The stated random walk problem is equivalent to the
ruin probability for a gambler starting
with initial fortune $b$ who succeeds by reaching increasing
his fortune by $a$ to reach the level
$a + b$ before being ruined by losing $b$ reaching 0.  This is the
complementary probability to the ruin probability and so may
be expressed as
\[
  p_b = 1 - q_b
      = 1 - \frac{ (q/p)^{a+b} - (q/p)^{b}}{(q/p)^{a+b} - 1}
\]
This can be simplified to
\[
   \frac{ (q/p)^b - 1}{ (q/p)^{a+b} - 1}.
\]
Alternatively, we can view this as the ruin of the gambler's
adversary, and using the idea in the first corollary, express
this as:
\[
( (p/q)^{a+b} - (p/q)^{a} )/( (p/q)^{a+b} - 1 ).
\]
One can verify with a little algebra that these expressions
are equivalent.
\end{sol}
\end{problem}

\begin{problem}
 Show that in a random walk starting at the origin the
    probability to reach  $a>0$ before returning to the origin
    equals $p(1-q_1)$

\begin{sol}

If the walker starts at the origin and
goes to $-1$ at the first step, then the walk must return to
the origin again before possibly reaching $a>0$.  Hence we
need only consider the possibility of the walk starting from
the point $1$ at the first step, and then reaching the value
$a>0$ before returning to the origin.  The probability of
going to $1$ is $p$, and then from $1$ the subsequent
independent probability of reaching $a$ before returning to
the origin is the same as the probability of the gambler
achieving success at $a$ before reaching the origin, which is
$1-q_{1}$.  Therefore the joint probability of the two
independent events
in succession is $p(1-q_{1})$.
Consider a special case just to check the results.  Consider $p =  1/2 = q$.
Then it is easy to compute that $p(1-q_1) = (1/2)(1 - (1 - 1/a)) = 1/(2a)$.
Now take $a = 2$, and start from the origin.  It is easy to see that the only to go from
the origin to the point $a=2$ before returning to the origin is
to go from $0$ to $1$, and then $1$ to $2$ in sequence.
Anything whatsoever can then happen, so long as the walk
ultimately returns to the origin, which it must.  So from direct
computation, the probability of such a path is $1/4$.  The
formula gives the probability $1/(2a) = 1/(2 \cdot 2) = 1/4$, so the
formula agrees with direct calculation in this special case,
and we are reassured!
\end{sol}
\end{problem}

\begin{problem}
  \begin{enumerate}
    \item Draw a sample path of a random walk (with $p = 1/2 = q$)
      starting from the
      origin where the walk visits the position $5$ twice before
      returning to the origin.
    \item Using the result from questions 2, it can be shown
      with careful but elementary reasoning that the number of times
      $N$ that a random
      walk ($p = 1/2 = q$) reaches the value $a$ a total of
      $n$ times before
      returning to the origin is a random variable with
      probability
      \[
           \P[N=0] = 1 - 1/(2a)
      \]
      and
      \[
          \P[N=n] = \left( \frac{1}{4a^2} \right) \left( 1 -
          \frac{1}{2a} \right)^{n-1}.
      \]
      for $n \ge 1$.
      Compute the
      expected number of visits $\E[N]$ to level $a$.
    \item Compare the expected number of visits of a random walk
      ($p= 1/2 = q$) to the value ``1 million'' before returning to
      the origin and to the level $10$ before returning to the origin.
  \end{enumerate}

\begin{sol}

for part b:
\begin{quote}
  If $q \ge p$, conclude from the preceding problem:  In a random
  walk starting at the origin, the number of visits to the point
  $a > 0$ that take place before the first return to the
  geometric distribution with ratio $1 - q q_{a-1}$.
  (Explain why the condition $q \ge p$ is necessary)
\end{quote}

This problem is conceptually and computationally simpler if $p
= q = 1/2$ so I will work only that case.
It is convenient to first calculate and record
some probabilities, then solve the problem.
Starting from the origin, the probability to reach $a>0$ before
returning to the origin is $p (1-q_1) = p ( 1 - (1 - 1/a)) =
p (1/a) = 1/(2a)$.
Starting at $a$, the probability to reach the origin before
returning to the starting point is $q q_{a-1} = q ( 1 - (a-1)/a)
= q (1/a) = 1/(2a)$.  This makes sense, since this situation is
symmetric with the previous situation and so should have the
same probability.
The ratio for the geometric probability is supposed to be $1 -
q q_{a-1} = 1 - q (1 - (a-1)/a) = 1 - q (1/a) = 1 - 1/(2a) =
(2a-1)/(2a)$.
The probability of $0$ visits to $a$ before hitting the
origin is the complement of the probability of one (or more)
visits to $a$ before hitting the origin.  Since the random
walk is recurrent (that is, will hit every point eventually
from any starting point, this is where the necessity of $p = q
=1/2$ comes in!) the probability of hitting the origin again
from $a$ (perhaps after more visits to $a$) is certain, so the
probability of one (or more) visits to $a$ before hitting the
origin is the same as the probability of a first visit to $a$
before returning to the origin,
namely $p (1-q_1) = 1/(2 a)$.  Then the desired probability of $0$
visits is $(1 - 1/(2a)) = (2a-1)/(2*a)$.
The probability of exactly one visit to $a$ before returning to
the origin is the
probability of passing from the origin to $a$ without first
returning to the origin, followed by passing from $a$ to the
origin without first returning to $a$.  This is
$p (1-q_1) q q_{a-1} = ( 1/(2 a) ) ( 1/(2 a) ) = 1/(4 a^2)$.
The probability of exactly two visits to $a$ before
returning to the origin is the
probability of passing from the origin to $a$ without first
returning to the origin, followed by a bridge from to $a$ to $a$
without touching the origin, followed by passing from $a$ to the
origin without first returning to $a$.  The middle probability
is the same as the already computed probability of passing
exactly $0$ visits to $a$ from the origin, by the symmetry of the
situation.  This is
\[
  p (1-q_1) ( (2 a-1)/(2 a) )   q q_{a-1} = ( 1/(2 a) ) (
  1/(2 a) ) = 1/(4 a^2) (2 a-1)/(2 a).
\]
Now the pattern is clear, and we see that the number of visits
to $a$ with out first returning to the origin is a deficient geometric
random variable with ratio $(2a-1)/(2a)$.

That is, the probability distribution is
\begin{eqnarray*}
  \P[N = 0] &=& (2a-1)/(2a) \\
  \P[N = 1] &=& 1/(4a^2) \\
  \P[N = 2] &=& ( 1/(4 a^2)) (2a-1)/(2a) \\
  \P[N = 3] &=& ( 1/(4 a^2)) ((2a-1)/(2a))^2 \\
\end{eqnarray*}
and so on the expected value of $N$ is then
\[
    \E[N] = \sum_{i=1}^{\infty} i \frac{1}{4a^2} \left(
    \frac{2a-1}{2a} \right)^{i-1} = 1
\]
This is astonishing, the expected number of visits to $a$
before returning to the origin, regardless of the value of $a$
is 1, the same for 1 million as for $10$!.
\end{sol}
\end{problem}



% Мартингалы (и равномерная интегрируемость пока сюда)
% сюда же моменты остановки и пр.
% отличие от п 9 состоит в том, что в задаче требуется понимание, что такое мартингал/
% /момент остановки
\section{Мартингалы}

\begin{problem}
Придумайте мартингал $X_{n}$, такой, что одновременно выполнены 3 условия: \\
\begin{enumerate}
\item $\sup X_n<\infty$ \\
\item Существует число $a$, такое что $\P(X_{n}=a i.o.)=1$
\item Существует число $b>a$, такое что $\P(X_{n}=b i.o.)=1$
\end{enumerate}

\begin{sol}

\end{sol}
\end{problem}

\begin{problem}
Приведите пример субмартингала $X_{n}$ такого, что $X_{n}^{2}$ -
супермартингал.

\begin{sol}

\end{sol}
\end{problem}

\begin{problem}
Докажите утверждения: \\
Если $X_{n}$ — субмартингал, то $(X_{n}-a)^{+}$ — субмартингал. \\
Если $X_{n}$ — супермартингал, то $X_{n} \wedge a$ -
супермартингал.

\begin{sol}

\end{sol}
\end{problem}

\begin{problem}
Если $X_{n}$ — мартингал по отношению к некоторой фильтрации
$\mathcal{F}_{n}$, то $X_{n}$ — мартингал по отношению к своей
естественной фильтрации.

\begin{sol}

\end{sol}
\end{problem}

\begin{problem}
Пусть $S_{n}$ — симметричное случайное блуждание, $S_{0}=0$. \\
Верно ли, что $Z_{n}=(-1)^{n}\cos(\pi S_{n})$ — мартингал? \\
Source: Zastawniak, ex. 3.6

\begin{sol}

\end{sol}
\end{problem}

\begin{problem}
\label{strict Doob} Пусть $M_{n}$ — неотрицательный субмартингал,
$\lambda>0$ и
$M_{n}^{*}=\sup_{i\le n}M_{i}$. \\
Верен ли следующий вариант неравенства Дуба: $\lambda
\P(M_{n}^{*}>\lambda)\le \E(M_{n}1_{M_{n}^{*}>\lambda})\le
\E(M_{n})$? \\
\ref{strict Doob}

\begin{sol}

Да. Доказательство не меняется при замене неравенства. Например,
Steele, p. 19-20.
\end{sol}
\end{problem}

\begin{problem}
$[$Williams, E10.1$]$ \\
В начальный момент времени в урне лежит 1 черный и 1 белый шар. В
каждый момент времени из урны наугад извлекается один шар. Вместо
одного извлеченного шара в урну кладут два шара такого же цвета.
Пусть $C_{n}$ — количество, а $M_{n}$ — доля черных шаров после
$n$-го шага. Рассмотрим также величину $B_{n}$, показывающую,
сколько раз из урны доставали черный шар к моменту времени $n$
включительно.
\begin{enumerate}
\item Как связаны $B_{n}$ и $C_{n}$?
\item Докажите, что $M_{n}$ — мартингал.
\item Найдите $\P(C_{n}=k)$.
\item Как распределен $\lim M_{n}$?
\item Докажите, что для $0<\theta<1$ мартингалом будет последовательность
\[
N_{n}=(n+1)C_{n}^{B_{n}}\theta^{B_{n}}(1-\theta)^{n-B_{n}}.
\]
\end{enumerate}

\begin{sol}
 г) \\
Процедуру можно представить себе так: \\
Положим на прямую красный шар. Затем по одному будем класть белые шары (миллион штук). Причем класть их будем равновероятно на любое место между уже положенными шарами, или с любого края полоски. \\
В результате положение красного шара (отделяющего шары разных урн) распределено равновероятно. \\
То есть в пределе доля распределена равномерно на $[0;1]$
\end{sol}
\end{problem}

\begin{problem}
В фирме работает два менеджера, $A$ и $B$. Изначально у каждого менеджера было по одному клиенту. Обозначим $A_{n}$ и $B_{n}$ — количество клиентов, обратившихся к менеджерам $A$ и $B$ к моменту времени $n$ включительно. Обозначим $M_{n}=\frac{A_{n}}{A_{n}+B_{n}}$, долю клиентов, предпочедших менеджера $A$. \\
В каждый момент времени $n$ в фирму приходит один новый клиент. Он обращается к менеджеру $A$ с вероятностью $M_{n-1}$, и к менеджеру $B$ с вероятностью $(1-M_{n-1})$.
\begin{enumerate}
\item Верно ли, что $M_{n}$ — мартингал?
\item Верно ли, что $M_{n}$ сходится (as)?
\item К чему сходится (as) $M_{n}$?
\end{enumerate}

\begin{sol}

\end{sol}
\end{problem}

\begin{problem}
Вася стреляет в мишень. Вероятность попасть в следующий раз равна доле имеющихся попаданий. В первый раз Вася попал, во второй — нет.

\begin{sol}

\end{sol}
\end{problem}

\begin{problem}
Изначально курс акций $S_{0}=100$. Безрисковая процентная ставка
равна $r=0.1$. В каждый момент времени курс акций может либо
домножиться на $u=1.2$ c вероятностью $p$, либо на $d=0.9$ c
вероятностью $1-p$. При каком $p$ дисконтированный курс акций
будет мартингалом?

\begin{sol}

\end{sol}
\end{problem}

\begin{problem}
Каждый индивид живет один момент времени. Число потомков каждого
индивида — случайная величина, имеющая закон распределения $X$
(одинаковый для всех индивидов). Пусть $Y_{n}$ — численность
$n$-го поколения, $Y_{0}=1$.
\begin{enumerate}
\item Найдите константу $\theta$, при которой
$G_{n}=\frac{Y_{n}}{\theta^{n}}$ будет мартингалом.
\item Пусть $A$ — событие, состоящее в том, что рано или поздно
популяция погибнет. Пусть $\P(A)=\alpha$. Найдите
$\P(A|\mathcal{F}_{n})$ как функцию от $Y_{n}$. Верно ли, что это
мартингал?
\item Найдите уравнение на $\alpha$.
\end{enumerate}

\begin{sol}

\end{sol}
\end{problem}

\begin{problem}
$[$Williams, E10.9$]$ \\
Пусть $X_{n}$ — супермартингал, $T$ — момент остановки. Докажите,
что
\begin{enumerate}
\item $\E(X_{T}1_{T<\infty})\le \E(X_{0})$
\item $c\P(\sup_{n} X_{n} \ge c)\le \E(X_{0})$
\end{enumerate}

\begin{sol}

\end{sol}
\end{problem}

\begin{problem}
$[$Steele, 2.5.$]$ \\
Пусть $M_{n}$ — субмартингал, а $\tau\le\nu$ ограниченные моменты
остановки. Докажите, что $\E(M_{\tau})\le \E(M_{\nu})$

\begin{sol}

\end{sol}
\end{problem}

\begin{problem}
Некоррелированность приращений. \\
Пусть $X_{n}$ — мартингал, $X_{n}\in L^{2}$, и четыре момента
времени $n_{1}<n_{2}\le n_{3}<n_{4}$. \\
Найдите $Cov(X_{n_{2}}-X_{n_{1}},X_{n_{4}}-X_{n_{3}})$

\begin{sol}

\end{sol}
\end{problem}

\begin{problem}
$[$Williams, E10.2$]$ Принцип Беллмана \\
Выигрыш $X_{n}$ в $n$-ой партии может равняться 1 или -1 c
вероятностями $p$ и $q=1-p$, соответственно. $Z_{n}$ -
благосостояние игрока после $n$ партий. Пусть $1/2<p<1$ и
$Z_{0}>0$. Игрок решил играть $N$ партий. Цель игрока -
максимизировать ожидаемую «процентную ставку»
$\E(\ln\frac{Z_{N}}{Z_{0}})$. К моменту времени $n$ игрок владеет
информацией $\mathcal{F}_{n}=\sigma(X_{1},\ldots,X_{n})$. На $n+1$-ую
партию игрок может поставить любую сумму в диапазоне $[0;Z_{n}]$.
\\
Обозначим «энтропию» $\alpha=p\ln p+q\ln q+\ln 2$.
\begin{enumerate}
\item Докажите, что для любой (предсказуемой) стратегии величина $\ln
Z_{n}-n\alpha$ будет супермартингалом.
\item Найдите наилучшую стратегию, т.е. такую, что $\ln
Z_{n}-n\alpha$ будет мартингалом

Примечание: величина $\ln\frac{Z_{N}}{Z_{0}}$ называется
процентной ставкой в силу того, что $\ln(1+r)\approx r$
\end{enumerate}

\begin{sol}

Обозначим $s$ размер ставки в $n+1$-ой партии. Тогда
$Z_{n+1}=Z_{n}+X_{n+1}sZ_{n}$. \\
Задача сводится к максимизации $E[(\ln Z_{n+1}-(n+1)\alpha) — (\ln
Z_{n}-n\alpha)|\mathcal{F}_{n}]$ по $s$. \\
Получаем $s=2p-1=p-q$.
\end{sol}
\end{problem}

\begin{problem}
A propos
{\it Энтропией} величины  $X$  называется число $E\left(\log _{2}
\frac{1}{p\left(X\right)} \right)$ , где
$p\left(t\right)=\P\left(X=t\right)$  для дискретных величин, а для
непрерывных величин  $p\left(t\right)$  — функция плотности. Энтропия
измеряет количество информации, получаемой при наблюдении величины
$X$. При расчете энтропии важны не сами значения величины $X$, а
их вероятности. \\
Логарифм можно брать по другому основанию, когда логарифм берется
по основанию 2, то единица измерения энтропии — бит. \\
Пусть имеется монетка, выпадающая гербом с вероятностью  $\theta$.

Величина  $X$  равна единице, если монетка выпадает гербом, и нулю
в противном случае. При каком  $\theta$  энтропия будет
максимальной? Чему она при этом будет равна?

\begin{sol}

\end{sol}
\end{problem}

\begin{problem}
Докажите утверждения: \\
Если $X_{n}$ — субмартингал, то $(X_{n}-a)^{+}$ — субмартингал. \\
Если $X_{n}$ — супермартингал, то $X_{n} \wedge a$ -
супермартингал.

\begin{sol}

\end{sol}
\end{problem}

\begin{problem}
Известно, что $X_{n}$ — субмартингал по отношению к фильтрации $\mathcal{F}_{n}$. Также известно, что $\E(X_{n})=const$. Верно ли, что $X_{n}$ — мартингал?

\begin{sol}
да
\end{sol}
\end{problem}

\begin{problem}
Приведите пример адаптированного случайного процесса, $X_t$, такого что: $\E(X_{t+1})=\E(X_t)$, но процесс $X_t$ — не мартингал.

\begin{sol}
$X_t=tU$, где $U$ равновероятно принимает значения $(+1)$ или $(-1)$.
\end{sol}
\end{problem}

\begin{problem}
Приведите пример случайного процесса $X_t$ и двух фильтраций $\mathcal{F}_t$ и $\mathcal{H}_t$, таких что $X_t$ — мартингал относительно $\mathcal{F}_t$, но не мартингал относительно $\mathcal{H}_t$, хотя $X_t$ адаптирован к обеим фильтрациям. Будет ли в построенном примере $X_t$ мартингалом относительно $\mathcal{K}_t=\mathcal{F}_t \cap \mathcal{H}_t$?


\begin{sol}

$X_t$ — случайное блуждание, $\mathcal{F}_t$ — естественная фильтрация, $\mathcal{H}_t$ — колдунская фильтрация. Да, мартингал относительно $\mathcal{K}_t=\mathcal{F}_t \cap \mathcal{H}_t$.
\end{sol}
\end{problem}


\begin{problem}
Величины $Z_1$, $Z_2$, \ldots независимы и равновероятно принимают значения плюс и минус один. Процесс $(X_n)$ задаётся соотношением $X_n=Z_1+ \ldots + Z_n$. Фильтрация $(\cF_n)$ — естественная фильтрация для процесса $(Z_n)$.

\begin{enumerate}
  \item Является ли процесс $(Z_n)$ мартингалом?
  \item Является ли процесс $(X_n)$ мартингалом?
  \item Является ли процесс $Y_n=2^{X_n}$ мартингалом?
  \item Подберите константу $\alpha$, чтобы процесс $R_n = X_n - \alpha n$ был мартингалом.
  \item Подберите константу $\beta$, чтобы процесс $S_n = \beta^{X_n}$ был мартингалом.
\end{enumerate}


\begin{sol}
\end{sol}
\end{problem}


\subsection{Равномерная интегрируемость}

\begin{problem}
Конечный набор интегрируемых случайных величин равномерно
интегрируем.

\begin{sol}
 По DCT.
\end{sol}
\end{problem}

\begin{problem}
Если $|X_{\alpha}|<Y$ и $\E(Y)<+\infty$, то набор $\{X_{\alpha}\}$
равномерно интегрируем.

\begin{sol}

\end{sol}
\end{problem}

\begin{problem}
Если набор случайных величин ограничен в смысле $L_{p}$,
$p>1$, то он равномерно интегрируем. \\
Ограничен в смысле $L^{p}$:
$\sup_{\alpha}E\left(|X_{\alpha}|^{p}\right)<+\infty$.

\begin{sol}

\end{sol}
\end{problem}

\begin{problem}
Равномерно интегрируемый набор ограничен в $L_{1}$. [Aad, 1.16]

\begin{sol}

Существует $N$, такое что $\sup_{Z\in \mathcal{H}}E
\left(|Z|1_{Z>N} \right)<1$. \\
$\sup_{Z\in \mathcal{H}}E |Z|=\sup_{Z\in \mathcal{H}}E
\left(|Z|1_{Z>N} \right)+\sup_{Z\in \mathcal{H}}E \left(|Z|1_{Z\le
N} \right)\le
1+N$
\end{sol}
\end{problem}

\begin{problem}
Сходящаяся в $L_{1}$ последовательность является равномерно
интегрируемой. [Aad, 1.17]

\begin{sol}

(better solution is welcome). \\
$|X_{k}|1_{|X_{k}|>M}\le |X-X_{k}|1_{|X_{k}|>M}+|X|1_{|X_{k}|>M}$
\\
Заметим, что $1_{|X_{k}|>M}\le 1_{|X|>M/2}+1_{|X-X_{k}|>M/2}$ \\
$|X_{k}|1_{|X_{k}|>M}\le
|X-X_{k}|+|X|1_{|X|>M/2}+|X|1_{|X-X_{k}|>M/2}$ \\
Можно выбрать $K$ и $M$ так, что для любого $k\ge K$ правая часть
будет меньше, чем $3\varepsilon$. А в силу конечности $K$ можно
выбрать $M$ так, чтобы $|X_{k}|1_{|X_{k}|>M}\le 3\varepsilon$ для
$k<K$.
\end{sol}
\end{problem}

\begin{problem}
Пусть наборы случайных величин $\mathcal{H}_{1}$ и
$\mathcal{H}_{2}$ равномерно интегрируемы.
Тогда равномерно интегрируемо их объединение.

\begin{sol}

\end{sol}
\end{problem}

\begin{problem}
 $[$Grimmett, 7.10.1$]$\\
 Пусть $\{X_{n}\}$ и $\{Y_{n}\}$ равномерно
интегрируемы. Верно ли, что последовательность $\{X_{n}+Y_{n}\}$
равномерно интегрируема?

\begin{sol}

\end{sol}
\end{problem}

\begin{problem}
 Show that a convex function of a martingale is a submartingale.
In other words, let $M_0,M_1,\dots,M_N$ be a martingale and let $\phi$ be a convex function.
Show that $\phi(M_0),\phi(M_1),\dots,\phi(M_N)$ is a submartingale.

\begin{sol}

\end{sol}
\end{problem}

\begin{problem}
Problem 1.3 (Discrete time stochastic integral) Suppose $M_0,M_1,\dots,M_N$
is a martingale and let $\Dt_0,\Dt_1,\dots,\Dt_{N-1}$ is an adapted process. Define the
\emph{discrete-time stochastic integral} (sometimes called a \emph{martingale transform})
$I_0,I_1,\dots,I_N$ by setting $I_0=0$ and
\[
I_n=\sum\lims_{j=0}^{n-1} \Dt_j (M_{j+1}-M_j),\quad n=1,2,\dots,N.
\]
\ni Show that $I_0,I_1,\dots,I_N$ is a martingale.

\begin{sol}

\end{sol}
\end{problem}

\begin{problem}
 Рассмотрим серию испытаний Бернулли с подбрасыванием монеты. Положим $X_j=1,$
  если при $j$-ом подбрасывании монета оказалась верхом стороной \emph{head}, и
  $X_j=-1$ иначе. Определим \emph{дискретное случайное блуждание} как
  случайный процесс $M_0,M_1,M_2,\dots,$ где $M_0=0$ и
\[
  M_n=\sum\lims_{j=1}^{n} X_j,\quad n\ge 1.
\]
  \ni Докажите, что $M_0,M_1,\dots$ матингал. Пусть $\sigma>0.$
  Положим $S_n=e^{\sigma M_n}\cdot\left(\frac{2}{e^{\sigma}+e^{-\sigma}} \right)^n$
  \ni Докажите, что процесс $S_0,S_1,S_2,\dots$ также мартингал.

\begin{sol}

\end{sol}
\end{problem}

\begin{problem}
 Дискретный стохастический интеграл. Пусть процесс $M_0,M_1,\dots,M_N$ является
  мартингалом, а процесс $\Dt_0,\Dt_1,\dots,\Dt_{N-1}$ адаптирован. Определим новый процесс
  $I_0,I_1,\dots,I_N$ как
 \[
  I_0=0,\quad I_n=\sum\lims_{k=0}^{n-1} \Dt_j\cd (M_{j+1}-M_j),\, n=1,\dots,N.
\]
  \ni Докажите, что $I_0,I_1,\dots,I_N$ также мартингал.

\begin{sol}

\end{sol}
\end{problem}

\begin{problem}
 Toss a symmetric coin repeatedly. Let $X_j=1$ if the $j$-th toss
results in a head and $X_j=-1$ if the $j$-th toss results in a tail. Consider the
stochastic process $M_0,M_1,M_2,\dots$ defined by $M_0=0$ and
\[
M_n=\sum\lims_{j=1}^{n} X_j,\quad n\ge 1.
\]
\ni This is called a \emph{symmetric random walk}; with each head it steps up one,
and with each tail, it steps down one.

\begin{enumerate}
\item[(i)] Show that $M_0,M_1,\dots$ is a martingale

\item[(ii)] Let $\sigma$ be a positive constant and, for $n\ge 0$, define
\[
S_n=e^{\sigma M_n}\cdot\left(\frac{2}{e^{\sigma}+e^{-\sigma}} \right)^n
\]
\ni Show that $S_0,S_1,S_2,\dots$ is a martingale. Note that even though the symmetric random
walk $M_n$ has no tendency to grow, the \emph{geometric symmetric random walk}, which is
$e^{\sigma M_n}$ does have a tendency to grow. This is the result of putting a martingale into
the convex exponential function. In order to again have a martingale, we must «discount» the
geometric symmetric random walk, using the term  $\frac{2}{e^{\sigma}+e^{-\sigma}}$ as the discount rate.
\end{enumerate}

\begin{sol}

\end{sol}
\end{problem}

\begin{problem}
 [Steele, 2.1.]
$S_{n}=S_{n-1}+X_{n}$,
$X_{n}=
\begin{cases}
	-1, p=0,52 \\
	1, p=0,45 \\
	2, p=0,03 \\
\end{cases} $
Найдите $x$ чтобы процесс $x^{S_{n}}$ был мартингалом.

\begin{sol}

\end{sol}
\end{problem}

\begin{problem}
 /Тема/. \\
Известно, что $A\in\mathcal{F}_{n}$ и $\omega\in A \Rightarrow
N(\omega)>n$. Верно ли, что $A \in \mathcal{F}_{n\wedge N}$?

\begin{sol}

\end{sol}
\end{problem}

\begin{problem}
Пусть $X_{i}\ge 0$, $S_{n}=S_{n-1}+X_{n}$, $N_{t}=sup(n|S_{n}\le
t)$. Докажите, что $N_{t}+1$ — момент остановки.

\begin{sol}

\end{sol}
\end{problem}

\begin{problem}
 Обезьяны и Шекспир \\
Кто-то давным-давно заметил, что обезьяна, посаженная за
пишущую машинку, напечатает рано или поздно полное собрание
сочинений Шекспира. Вопрос в том, сколько ей в
среднем потребуется нажатий на клавиши (предположим, что их 26 как
букв в
английском языке) чтобы напечатать слово ABRACADABRA? \\
Трюк: Организуем казино! В каждый момент времени в казино приходит
новый игрок с начальным капиталом в 1 рубль. Каждый входящий игрок
действует по одной и той же схеме: ставит все имеющиеся деньги на
очередную букву слова ABRACADABRA (т.е. сначала на А, потом на B и
так далее, когда слово кончается все деньги снова ставятся на A\ldots).
Если обезьяна напечатала нужную букву, то игрок получает свою
ставку, увеличенную в 26 раз, если нет — то игрок покидает казино.
Рассмотрите мартингал, связанный с суммарным благосостоянием всех
игроков. \\
Трюк исполнен профессиональными каскадерами, не пытайтесь
повторить его самостоятельно (Williams, Proba with Martingales).

\begin{sol}

\end{sol}
\end{problem}

\begin{problem}
 Гладиаторы-вампиры \\
Есть две команды гладиаторов-вампиров. У каждого из них своя сила.
Если вампир силы $a$ убивает вампира силы $b$ (что происходит с
вероятностью $\frac{a}{a+b}$), то его сила увеличивается на $b$.
Соревнование между командами построено в форме отдельных
поединков. От каждой команды выдвигается по одному участнику,
победитель возвращается в свою команду и начинается следующий
поединок. Допускается неоднократное участие одного вампира в
поединках. Пусть в первой команде $n_{x}$ участников с суммарной
силой $X_{0}$, а во второй — $n_{y}$ участников с суммарной силой
$Y_{0}$; $X_{n}$ — суммарная сила первой команды после $n$
поединков.
\begin{enumerate}
\item  Докажите, что $X_{n}$ — мартингал.
\item Верно ли, что игра закончится к моменту $n_{x}+n_{y}$?
\item Найдите вероятность выигрыша первой команды.
\end{enumerate}

\begin{sol}

\end{sol}
\end{problem}

\begin{problem}
 Пуассоновский процесс. \\
Случайный процесс $N(t)$ (в непрерывном времени) называется
Пуассоновским
процессом с параметром $\lambda$, если выполнены: \\
P1. При $0=t_{0}<t_{1}<t_{2}<\ldots<t_{n}$ величины
$\triangle_{k}=N(t_{k})-N(t_{k-1})$ являются независимыми. \\
P2. Для $t>s$ величина $N(t)-N(s)$ имеет распределение Пуассона с
параметром $\lambda \cdot (t-s)$ \\
По смыслу $N(t)$ показывает, сколько событий произошло к моменту
времени $t$. \\
Кстати, кто не знал и забыл: \\
Распределение Пуассона с параметром $\lambda$:
$\P(X=n)=e^{-\lambda}\frac{\lambda^{n}}{n!}$ \\
Пусть $t_{n}$ — фиксированные моменты времени и
$X_{n}=N(t_{n})-\lambda\cdot t_{n}$. \\
Верно ли, что $X_{n}$ — мартингал?

\begin{sol}

\end{sol}
\end{problem}

\begin{problem}
По мотивам Rock-Scissors-Paper and survival of the weakest \\
Есть $N$ клеток, в каждой из которых живет скорпион. Скорпионы
бывают трех видов. \\
Яд скорпиона 1-го вида убивает скорпиона 2-го вида с вероятностью
$p_{1}$. Яд скорпиона 2-го вида убивает скорпиона 3-го вида с
вероятностью $p_{2}$. Яд скорпиона 3-го вида убивает скорпиона
1-го вида с вероятностью $p_{3}$. В остальных случаях яд
безвреден. \\
В каждый момент времени случайным образом выбираются 2 клетки.
Происходит сражение скорпионов. Если один из скорпионов погиб, то
опустевшая клетка проигравшего скорпиона заселяется новым
скорпионом победившего вида. \\
Численности скорпионов в момент времени $t$ обозначим $N_{1}(t)$,
$N_{2}(t)$ и $N_{3}(t)$.
\begin{enumerate}
\item Чему равно $\E(N_{1}(t+1)-N_{1}(t)|\mathcal{F}_{t})$?
\item При каких условиях на $N_{1}(0)$, $N_{2}(0)$ и $N_{3}(0)$
средняя численность скорпионов каждого вида будет неизменной?
\item Верно ли, что при этих условиях численность скорпионов каждого
вида будет мартингалом?
\item Как зависят от $p_{1}$, $p_{2}$ и $p_{3}$ равновесные
численности скорпионов?
\end{enumerate}

\begin{sol}

\end{sol}
\end{problem}

\begin{problem}
Перед Машей колода в 52 карты. Маша открывает карты одну за одной.
В любой момент времени Маша может сделать заявление "Следующая
карта будет красной!". Маша выигрывает, если ее
предсказание сбудется.
\begin{enumerate}
\item Найдите оптимальную стратегию, используя элементарные методы.
\item Определите мартингал, на который делает ставку Маша.
\end{enumerate}

\begin{sol}
Маше безразлично, делать ли прогноз на следующую карту или на
последнюю, так как о них в любой момент времени имеется одинаковая
информация. Исходная вероятность того, что последняя карта будет
красной равна $\frac{1}{2}$. Следовательно, любая Машина стратегия
приводит к такой вероятности выигрыша. Маша не может ее ни
увеличить, ни снизить!
\end{sol}
\end{problem}

\begin{problem}
 «La grande martingale» \\
Пусть $Y_{i}$ — iid, результат $i$-го подбрасывания правильной
монетки, т.е. $ Y_{i}=
\begin{cases}
1, p=1/2 \\
0, p=1/2 \\
\end{cases}$ \\
Обозначим $X_{n+1}=X_{n}+Y_{n+1}$ \\
Допустим, что игрок использует «стратегию удвоения» (la grande
martingale). На первый результат подбрасывания он ставит
рубль. Если он проигрывает, то удваивает ставку. Если он
выигрывает, то снова начинает со ставки размером в рубль. \\
Опишите, как выглядит соответствующее мартингальное преобразование
$(C\cdot X)_{n}$. \\
Верно ли, что $C_{i}$ является предсказуемым процессом? \\
Чему равно ожидаемое благосостояние игрока после $n$ подбрасываний
монетки?

\begin{sol}

\end{sol}
\end{problem}

\begin{problem}
 Намек на сходимость мартингалов. \\
Пусть последовательность действительных чисел $\{x_{n}\}$
сходится. Зафиксируем два числа $a<b$. Верно ли, что число
пересечений снизу-вверх отрезка $[a;b]$ конечно? \\
Пусть последовательность действительных чисел $\{x_{n}\}$
ограничена. Число пересечений снизу-вверх любого отрезка $[a;b]$,
где $a<b$ конечно. Верно ли, что $\{x_{n}\}$ сходится?

\begin{sol}

\end{sol}
\end{problem}

\begin{problem}
 Let $M_0,M_1,M_2\dots$ ne the symmetric random walk. Define $I_0=0$ and
\[
I_n=\sum\lims_{j=0}^{n-1} M_j(M_{j+1}-M_j),\quad n=1,2,\dots
\]
\ni Show that
\[
I_n=\frac{1}{2}M_n^2-\frac{n}{2}.
\]

\begin{sol}

\end{sol}
\end{problem}

\begin{problem}
Докажите, что
\begin{enumerate}
\item  Если $T$ — момент остановки, то множества $\{T<n\}$, $\{T\le
n\}$, $\{T=n\}$, $\{T\ge n\}$ и $\{T>n\}$ лежат в
$\mathcal{F}_{n}$.
\item Случайная величина $T\in \{0\}\cup \mathbb{N}\cup \{+\infty\}$
является моментом остановки если и только если $\{T=n\}\in
\mathcal{F}_{n}$ для любого $n$.
\item Если $T_{1}$ и $T_{2}$ моменты остановки, то $T_{1}\vee T_{2}$
и $T_{1}\wedge T_{2}$ — моменты остановки.
\end{enumerate}

\begin{sol}

\end{sol}
\end{problem}

\begin{problem}
Пусть $T$, $S$ — моменты остановки.
\begin{enumerate}
\item Проверьте то, что $\mathcal{F}_{T}$ — действительно
$\sigma$-алгебра.
\item Можно ли в определении $\mathcal{F}_{T}$ заменить $A\cap (T\le
n)\in \mathcal{F}_{n}$ на $A\cap (T=n)\in
\mathcal{F}_{n}$?
\item Пусть $T\equiv n$. Верно ли, что
$\mathcal{F}_{T}=\mathcal{F}_{n}$?
\item Пусть $T\le S$. Как связаны между собой $\mathcal{F}_{T}$ и
$\mathcal{F}_{S}$?
\item Верно ли, что $\mathcal{F}_{T\wedge S}=\mathcal{F}_{T}\cap
\mathcal{F}_{S}$?
\item Пусть $T<\infty$ и $X_{n}$ — адаптированный процесс. Верно
ли, что $X_{T}\in \mathcal{F}_{T}$?

Комментарий: попробуйте дать сначала интуитивный ответ на вопросы
в-е, а затем уже возиться с $\sigma$-алгебрами.
\end{enumerate}

\begin{sol}

\end{sol}
\end{problem}

\begin{problem}
Равномерная интегрируемость\ldots Докажите (или опровергните)\ldots
\begin{enumerate}
\item  Конечный набор интегрируемых величин равномерно
интегрируем.
\item Если существует $Y\in L^{1}$, такое что для любого $X\in
\mathcal{A}$ выполнено $|X|\le Y$, то набор $\mathcal{A}$ -
равномерно интегрируем.
\item Если $\sup_{X\in \mathcal{A}}\E(|X|^{p})<\infty$, где $p>1$, то
набор $\mathcal{A}$ -
равномерно интегрируем.
\item Для любых случайных величин $X$ и $Y$ выполнено неравенство:
$\E(|X|1_{|Y|>M})\le \E(|X|1_{|X|>N})+\frac{N}{M}\E(|Y|1_{|Y|>M})$.
\item Если $\mathcal{A}=\{\E(X|\mathcal{H}):\mathcal{H}\subseteq
\mathcal{F}\}$, т.е. набор $\mathcal{A}$ — это условные ожидания
от $X$ по всевозможным $\sigma$-алгебрам, то $\mathcal{A}$ -
равномерно интегрируем.
\end{enumerate}

\begin{sol}

\end{sol}
\end{problem}

\begin{problem}
Пусть $S_{n}$ — симметричное случайное блуждание с началом в
нуле и $\tau$ — момент первого достижения суммы в 1 рубль. Найдите
$\E(S_{n})$, $\E(S_{n\wedge \tau})$ и $\E(S_{\tau})$. Почему
$\lim_{n}
\E(S_{n\wedge \tau})\neq \E(S_{\tau})$?

\begin{sol}

\end{sol}
\end{problem}

\begin{problem}
Приведите примеры процесса в дискретном времени, который:
\begin{enumerate}
\item является и мартингалом, и цепью Маркова
\item не является ни мартингалом, ни цепью Маркова
\item является мартингалом, но не цепью Маркова
\item является цепью Маркова, но не мартингалом
\end{enumerate}

\begin{sol}
\end{sol}
\end{problem}

\subsection{Задачи на тему остановки мартингала}

\begin{problem}
Пусть $W_{t}$ — стандартный винеровский процесс, $a$ и $b$ больше 0, $\tau=min\{t|W_{t}=a \cup W_{t}=-b\}$. \\
Найдите $\P(W_{\tau}=a)$, $\E(\tau)$, $\E(\tau W_{\tau})$

\begin{sol}

Hint: Мартингалы $W_{t}$, $W_{t}^2-t$, $W_{t}^{3}-3tW_{t}$ вам в помощь
\end{sol}
\end{problem}

\begin{problem}
Пусть $W_{t}$ — стандартный винеровский процесс, $a$ и $b$ больше 0, $X_{t}=W_{t}+\mu t$, $\tau=min\{t|X_{t}=a \cup X_{t}=-b\}$. \\
Найдите $\P(X_{\tau}=a)$, $\E(\tau)$

\begin{sol}

Hint: Мартингалы $a^{X_{t}}$ и $X_{t}-\mu t$ вам в помощь
\end{sol}
\end{problem}

\begin{problem}
Карты из колоды в 52 карты извлекаются до появления 2-го туза. Какова вероятность того, что следующая карта будет тузом?

\begin{sol}

Hint: Мартингал $X_{t}$ — доля тузов среди неоткрытых карт вам в помощь
\end{sol}
\end{problem}

\begin{problem}
Двое играют в шахматы друг с другом до перевеса в 5 очков. Победитель партии получает 1 очко, проигравший — 0 очков, в случае ничьей каждый получает по пол-очка. Вася играет лучше Пети, поэтому: вероятность ничьей — одна четверть, вероятность того, что победит Вася равна половине, вероятность победы Пети — одна четверть.
\begin{enumerate}
\item Какова вероятность того, что в результате выиграет Вася?
\item Сколько партий они в среднем сыграют?
\end{enumerate}


\begin{sol}

Hint:  Если $X_{t}$ — разница в очках, то $Y_t=a^{X_{t}}$ и $M_t=X_{t}+bt$ — мартингалы.

$X_t=X_{t-1}+\e_t$,

\begin{tabular}{cccc}
\toprule
$\e_t$ & $0$ & $1$ & $-1$ \\
$\P$ & $0.25$ & $0.5$ & $0.25$ \\
\bottomrule
\end{tabular}

$\E(\e_t)=1/4$.

$M_t=X_t-t/4$

Из условия $\E(Y_{t+1}|\cF_t)=Y_t$ получаем $a=1/2$ и $Y_t=\left( \frac{1}{2} \right)^{X_t}$ — мартингал. Применяя теорему Дуба, находим, что $p=32/33$ и $\E(\tau)=18.8$.


\end{sol}
\end{problem}

\begin{problem}
У Маши — неправильная монетка (орел выпадет с вероятностью 0.6), у Светы — правильная. Маша и Света одновременно подкидывают свои монетки до тех пор, пока у кого-то из них орлов не накопится на два больше, чем у подружки. Побеждает в этой игре набравшая больше орлов. Какова вероятность того, что Маша выиграет?

Hints: Пусть $B_{t}=M_{t}-S_{t}$ — баланс числа орлов, т.е. преимущество Маши над Светой в момент времени $t$.
\begin{enumerate}
\item Какие значения и с какими вероятностями принимает $ \Delta B_{t} $?
\item При каком $ a $ случайный процесс $ a^{B_{t}} $ будет мартингалом?
\end{enumerate}


\begin{sol}
$ \left(\frac{2}{3}\right)^{B_{t}} $ — мартингал

$\e_t=\Delta B_t$

\begin{tabular}{cccc}
$\e_t$ & $0$ & $1$ & $-1$ \\
\hline
$\P$ & $0.5$ & $0.2$ & $0.3$ \\
\end{tabular}

Кстати, используя другой мартингал, можно установить, что
\[
\E\left( \left( \frac{10}{6} \right)^{M_{\tau}} \right) = \E\left( \left( \frac{10}{5} \right)^{S_{\tau}} \right)
\]

\end{sol}
\end{problem}

\begin{problem}
Пусть $S_{t}$ — симметрическое случайное блуждание с началом в нуле. $a$ и $b$ больше 0, $\tau=min\{t|S_{t}=a \cup S_{t}=-b\}$. \\
Найдите $\P(S_{\tau}=a)$, $\E(\tau)$

\begin{sol}

Hint: Мартингалы $S_{t}$, $S_{t}^2-t$ вам в помощь
% аналог с $\E(\tau S_{tau})$ ?
\end{sol}
\end{problem}

\begin{problem}
Пусть $S_{t}$ — несимметрическое случайное блуждание с началом в нуле, $\P(S_{t}-S_{t-1}=1)=p$, $a$ и $b$ больше 0, $\tau=min\{t|S_{t}=a \cup S_{t}=-b\}$. \\
Найдите $\P(S_{\tau}=a)$, $\E(\tau)$

\begin{sol}

Hint: Мартингалы $a^{S_{t}}$, $S_{t}+bt$ вам в помощь
\end{sol}
\end{problem}

\begin{problem}
Игральный кубик подбрасывается бесконечное количество раз. Сколько подбрасываний в среднем пройдет до появления последовательности 1-2-3-1-2?

\begin{sol}

Hint: самое сложное тут — подходящий мартингал
\end{sol}
\end{problem}

\begin{problem}
В коробке лежат 30 зеленых и 50 красных шаров. Мы извлекаем два наугад, один берем в левую руку, другой — в правую. Шар в левой руке красим в цвет шара, находящегося в правой руке. Затем возвращаем шары в коробку. Снова извлекаем два наугад и так далее. До тех пор, пока в коробке все шары не окрасятся в один цвет.
\begin{enumerate}
\item  Какова вероятность того, что шары будут окрашены в красный цвет?
\item  Сколько в среднем пар шаров будет извлечено?
\end{enumerate}


\begin{sol}

Hint: $K_{t}$ — число красных шаров — мартингал, $p=5/8$.
\end{sol}
\end{problem}

%Пусть $S_{n}$ — симметричное случайное блуждание, $S_{0}=0$, $K$ -
%натуральное число, $\tau=\min\{t|S_{t}=K\}$ \\
%a) Верно ли, что $Z_{n}=(-1)^{n}\cos(\pi(S_{n}+K))$ — мартингал? \\
%б) Используя Optional Stopping Theorem найдите $\E((-1)^{\tau})$ \\
%Source: Zastawniak, ex. 3.12 }
%\solution{ что-то там нужно косинус вообще убрать? проверить!}
% Задача не очень, т.к. $ Z_{n} $ — детерминистический процесс




\begin{problem}
Safes and keys \\
source: Taiwan NMO 2006\\
There are 94 safes and 94 keys. Each key can open only one safe, and each safe can be opened by only one key. We place randomly one key into each safe. 92 safes are then randomly chosen, and then locked. What is the probability that we can open all the safes with the two keys in the two remaining safes?\\
(Once a safe is opened, the key inside the safe can be used to open another safe.)

\begin{sol}

hint: Consider martingale $X_{n}$ the proportion of keys from the first two safes in safes which are closed at time $t$ \\
sol (one of many possible) \\
let $\tau$ be a stopping time — the time when we cannot open more safes (all safes are open or we have no more keys). $T=\tau \wedge (n-1)$ is also a stopping time. The «last safe» is the safe not opened at $T$ with minimal possible number. Let $M_{t}$  be the proportion of not opened safes at time $t$  containing keys from the first $k$ safes.  is a martingale. $T$ — bounded stopping time. $\E(M_{T}) = \E(M_{0}) = \frac {k}{n}$ by the optional stopping time theorem. And $\E(M_{T})$ is the probability we have been searching (the probability of opening all safes is exactly the probability that the last safe contains the key from the first  safes). \\
aops, 1310759
\end{sol}
\end{problem}

% Тождество Вольда, Wald's identity
\begin{problem}
Величины $X_i$ независимы, одинаково распределены и строго положительны, $S_n=\sum_{i=1}^n X_i$. Число $x>0$. Момент остановки $T_x=\min\{n|S_n\geq x\}$. Докажите неравенство $x\leq \E(\min\{X_1,x\})\E(T_x)\leq 2x$

\begin{sol}
По тождеству Вольда
\begin{equation}
 \E(\min\{X_1,x\})\E(T_x)=\E\left(\sum_{i=1}^{T_x}\min\{X_i,x\} \right)
\end{equation}
Остается доказать, что $x\leq \sum_{i=1}^{T_x}\min\{X_i,x\}\leq 2x $
% источник m.se.com, 76233
\end{sol}
\end{problem}


\begin{problem}
Вася прыгает на один метр вперед с вероятностью $p$ и на два метра вперед с вероятностью $1-p$. Как только он пересечет дистанцию в 100 метров он останавливается. Получается, что он может остановиться либо на отметке в 100 метров, либо на отметке в 101 метр.
\begin{enumerate}
\item Какова вероятность того, что он остановится ровно на отметке в 100 метров?
\item Численно с помощью компьютера найдите, при каком $p$ эта вероятность будет минимальной и чему она будет равна?
\end{enumerate}

% копия в методе первого шага
\begin{sol}
Обозначим $P_n$ вероятность остановиться ровно на $n$ метрах. Мы ищем $P_{100}$.

Решение 1. По методу первого шага:  $P_n=pP_{n-1}+(1-p)P_{n-2}$.

Решение 2. Попасть ровно в $n$ можно двумя способами: перелетев $n-1$ или попав в $n-1$ и сделав шаг в один метр. Значит $P_n=(1-P_{n-1})+pP_{n-1}$.

Решение 3. Обозначим Васину координату в момент времени $t$ как $X_t$. Можно найти $a$ так чтобы процесс $Y_t=a^{X_t}$ был мартингалом, $a=1/(p-1)$. Момент остановки $T=\min\{t|X_t\geq n\}$. Мартингал $Y_{t\wedge T}$ ограничен, теорема Дуба применима. $\E(Y_T)=\E(Y_0)=1$. Получаем
\[
P=\frac{1+(1-p)^{101}}{2-p}
\]

%«»=
%f <- function(p) {
%  ans <- (1+(1-p)^101)/(2-p)
%  return(ans)
%}
%x <- seq(0,1,by=0.001)
%qplot(x,f(x),geom="line")
%@
\end{sol}
\end{problem}


\begin{problem}
Храбрая исследовательница Мишель подбрасывает правильную монетку до появления последовательности ОРРОР. Сколько в среднем подбрасываний ей потребуется? Какова дисперсия количества подбрасываний?


\begin{sol}
\end{sol}
\end{problem}

\section{Броуновское движение}

\begin{problem}
Пусть $W_{t}$ — стандартное броуновское движение и $a>0$. \\
Являются ли следующие процессы броуновским движением? мартингалом? \\
\begin{enumerate}
\item $X_{t}=-W_{t}$;
\item $X_{t}=W_{a+t}-W_{a}$;
\item $X_{t}=\frac{1}{a}W_{a^{2}t}$
\item $X_{t}=W_{t}^{3}$
\end{enumerate}
\begin{sol}

We have to systematically check each of the defining properties
of Brownian motion in turn for each of the transformed processes.
\begin{enumerate}
  \item
    \[
      W_1(t) = cW(t/c^2)
    \]
    \begin{enumerate}
       \item
       The increments
       \[
         W_1(t) - W_1(s) = cW( (t)/c^2 ) - cW(s/c^2) = c
         ( W( t/c^2  ) - W( s/c^2 ))
       \]
       are clearly normally distributed as a multiple of a
       normally distributed random variable.  Since the
       increment $ W( t/c^2 ) - W( s/c^2 ) $ has
       mean zero, then
       \[
         W_1(t+s) - W_1(s) = c ( W( t/c^2 ) - W( s/c^2
         ))
       \]
       must have mean zero.  The variance is
       \begin{eqnarray*}
         \E[(W_1(t)-W(s))^2] &=& \E[(c W((t)/c^2) - c W(s/c^2))^2]
         \\
         &=& c^2 \E[ (W( t/c^2 ) - W(s/c^2))^2] \\
         &=& c^2 (t/c^2 - s/c^2) = t - s.
       \end{eqnarray*}

       \item
       Note that if $ t_1 < t_2 \le t_3 < t_4 $, then $ t_1/c^2
       < t_2/c^2 \le t_3/c^2 < t_4/c^2 $, and the corresponding
       increments $ W(t_4/c^2) - W(t_3/c^2) $ and $ W(t_2/c^2)
       - W(t_1/c^2) $ are independent.  Then the multiples of
       each by $ c $ are independent and so $ W_1(t_4) - W_1(t_3)
       $ and $ W_1(t_2) - W_1(t_1) $ are independent.

       \item
       $ W_1(0) = c W(0/c^2) = c W(0) = 0 $.

       \item
       As the composition of continuous functions, $ W_1 $ is
       continuous.
    \end{enumerate}

  \item
  To show that
  \[
    W_2(t) = t W(1/t)
  \]
  is a Brownian motion by the four definitional properties
  seems to be difficult.  Instead the usual proof is to show
  that any Gaussian process $X(t)$ with mean $0$ and $\cov[X(s),
  X(t)] = \min(s,t)$ must be Brownian motion. See the reference
  and outside links for more information.
  \begin{enumerate}
    \item
    \[
      W_2(t) - W_2(s) = t W(1/t) - s W(1/s)
    \]
    which will be the difference of normally distributed
    random variables each with mean $ 0 $, so the difference
    will be normal with mean $ 0 $.  It remains to check
    that the normal random variable has the correct
    variance.
    \begin{eqnarray*}
      \E[ (W_2(t)-W_2(s))^2] &=& \E[ (s W(1/s) - t W(1/
      t))^2 ] \\
      &=& \E[ ( s W(1/s) - s W(1/t) + s W(1/t) - t W(1/t)
       - (s -t) W(0) )^2 ] \\
      &=& s^2 \E[ (W(1/s) - W(1/t))^2 ] + (s-t)^2 \E[
      (W(1/t) - W(0))^2
      ] \\
      &=& s^2 ( 1/s - 1/t ) + (s-t)^2 ( 1/t ) \\
      &=& t-s
    \end{eqnarray*}
    Note the use of independence of $ W(1/s) - W(1/t) $
    from $ W(1/t) - W(0) $ at the third equality.

    \item
    It seems to be hard to show the independence of
    increments directly.  Instead rely on the fact that a
    Gaussian process with mean $0$ and covariance function
    $\min(s,t)$ is a Brownian motion, and thus prove it indirectly.
    Note that
    \[
      \cov[ W_2(s), W_2(t)] = st \min(1/s, 1/t) =
      \min(s,t).
    \]

    \item
    By definition, $ W_2(0) = 0 $.

    \item
    The argument that $ \lim_{t to 0} W_2(t) = 0 $ is
    equivalent to showing that $ \lim_{t \to \infty} W(t)/t
    = 0 $.  To show this requires use of Kolmogorov's
    inequality for Brownian motion and clever use of the
    Borel-Cantelli lemma and is beyond the scope of this
    course.  Use the translation property in the third
    statement of this theorem to prove continuity at every
    value of $ t $.
  \end{enumerate}

  \item
  \[
      W_3(t) = W(t+h) - W(h)
  \]
  \begin{enumerate}
    \item
    The increment
    \[
        W_3(t+s) - W_3(s) = [ W(t+s+h) - W(h) ] - [ W(s+h) -
        W(h) ] = W(t+s+h) - W(s+h)
    \]
    which is by definition normally distributed with mean $
    0 $ and variance $ t $.

    \item
    The increment
    \[
        W_3(t_4) - W_3(t_3) = W(t_4+h) - W(t_3+h)
    \]
    is independent from
    \[
        W(t_2+h) - W(t_1+h) = W_3(t_2) - W_3(t_1),
    \]
    by the property of independence of disjoint increments
    of $ W(t) $.

    \item
    \[
        W_3(0) = W(0+h) - W(h) = 0.
    \]

    \item
    As the composition and difference of continuous
    functions, $ W_3 $ is continuous.
    \end{enumerate}
  \end{enumerate}

\end{sol}
\end{problem}

\begin{problem}
Пусть $Z$ — стандартная нормальная случайная величина. Определим случайный процесс $X(t)=\sqrt{t}Z$.
\begin{enumerate}
\item  Найдите $\E(X(t))$
\item Найдите $\Var(X(t))$
\item Верно ли, что у процесса $X(t)$ независимые приращения?
\item Как выглядит типичная реализация $X(t)$?
\item Будет ли $X(t)$ броуновским движением?
\end{enumerate}

\begin{sol}
\end{sol}
\end{problem}

%Пусть $W_t $
% — стандартный Винеровский процесс (aka броуновское движение). $W_t $
% означает положение некоей частицы движущейся на прямой в момент времени $t$
%. Броуновское движение, в частности, обладает свойствами:
%\\
%1. $W_{0}=0$ \\
%2. Если $t > s$ , то $\left( {W_t  - W_s } \right) \sim N\left(
%{0;\left( {t - s} \right)} \right)$ . \\
%3. Если отрезки $\left[ {t_1 ;t_2 } \right]$
% и $\left[ {t_2 ;t_3 } \right]$
% не пересекаются, то с.в. $\left( {W_{t_3 }  - W_{t_2 } } \right)$
% и $\left( {W_{t_2 }  - W_{t_1 } } \right)$
% независимы. \\
%4. $\P(W_{t}\text{ непрерывно })=1$ \\



\begin{problem}
Найдите $\lim_{\Delta t \to 0} E\left[ {\left( {W_{t + \Delta t} -
W_t } \right)^2 } \right]$ , $\lim_{\Delta t \to 0} \frac{{E\left[
{\left( {W_{t + \Delta t} - W_t } \right)^2 } \right]}} {{\Delta
t}}$ , $\lim_{\Delta t \to 0} \frac{{E\left[ {\left( {W_{t +
\Delta t} - W_t } \right)^2 } \right]}} {{\left( {\Delta t}
\right)^2 }}$. \\
Найдите $E\left( {\left| {W_1  - W_0 } \right|} \right)$ ,
$E\left( {\left| {W_1  - W_{0,5} } \right| + \left| {W_{0,5}  -
W_0 } \right|} \right)$ , $\lim_{\Delta t \to 0} \frac{{E\left(
{\left| {W_{t + \Delta t} - W_t } \right|} \right)}} {{\Delta t}}$
. Какое расстояние (имеется ввиду полная длина пути, а не
перемещение) успевает в среднем пройти частица за единицу времени?

\begin{sol}

\end{sol}
\end{problem}

\begin{problem}
Пусть отрезок времени $[0;T]$ разделен на $n$ равных частей. \\
Для краткости обозначим $W_{i}=W(i\cdot \frac{T}{n})$, $\Delta W_{i}=W_{i}-W_{i-1}$.
\begin{enumerate}
\item Найдите $\E(\Delta W_{i})$, $\Var(\Delta W_{i})$, $\E((\Delta W_{i})^{2})$, $\E((\Delta W_{i})^{3})$, $\E((\Delta W_{i})^{4})$.
\item Пусть $A_{n}=\sum_{i=1}^{n}(\Delta W_{i})^{2}$. Найдите $\E(A_{n})$, $\Var(A_{n})$. Найдите к чему стремится $A_{n}$ в смысле $L^{2}$.
\item Пусть $B_{n}=\sum_{i=1}^{n}\Delta W_{i}$. Найдите $\E(B_{n})$, $\Var(B_{n})$. Найдите к чему стремится $B_{n}$ в смысле $L^{2}$.
\item Пусть $C_{n}=W_{n}^{2}-\sum_{i=1}^{n}W_{i}\Delta W_{i}$ \\
Выразите $C_{n}$ через $W_{n}$ и $A_{n}$ \\
% получается $C_{n}=\frac{1}{2}W_{n}^{2}-\frac{1}{2}A_{n}$ \\
Найдите $\E(C_{n})$ \\
Найдите к чему стремится $C_{n}$ в смысле $L^{2}$ \\
\item Пусть $D_{n}=\sum (\Delta W_{i})^{3}$ \\
Найдите к чему стремится $D_{n}$ в смысле $L^{2}$ \\
\item Пусть $X_{n}=\sum W_{i}W_{i-1}\Delta W_{i}$ \\
К чему стремится $X_{n}$ в смысле $L^{2}$?
\end{enumerate}

\begin{sol}

\end{sol}
\end{problem}

\begin{problem}
Пусть $W(t)$ — винеровский процесс. \\
К чему сходится $W(n)/n$ и в каком смысле?

\begin{sol}
Ответ: к 0, пожалуй во всех кроме поточенчном
\end{sol}
\end{problem}

\begin{problem}
  Let $Z$ be a normally distributed random variable, with mean
  $0$ and variance $1$, $Z \sim N(0,1)$.  Then consider the continuous
  time stochastic process $X(t) = \sqrt{t} Z$. Show that the distribution of
  $X(t)$ is normal with mean $0$ with variance $t$.  Is $X(t)$ a
  Brownian motion?

\begin{sol}

This problem is adapted from Baxter and Rennie, Problem 3.1, page 49.
No, $X(t)$ is not Brownian motion for two reasons in spite of the fact
that $\sqrt{t} Z \sim N(0,t)$ (which follows from being a scalar
multiple by $\sqrt{t}$ of the $N(0,1)$ random variable $Z$.)
First, for $t_1 < t_2 \le t_3 < t_4$, $X(t_2) - X(t_1) = (\sqrt{t_2} -
\sqrt{t_1})Z$ is not independent of $X(t_4) - X(t_3) = (\sqrt{t_4} -
\sqrt{t_3})Z$ since both are multiples of the same sample value $Z$
drawn from the $N(0,1)$ population.
Second, the distribution of $(\sqrt{t_2} - \sqrt{t_1})Z$ is normal
with variance $(\sqrt{t_2} - \sqrt{t_1})^2 \ne t_2 - t_1$.
Note nevertheless, that $X(0) = 0$ and $X(t)$ is continuous at $t=0$.
\end{sol}
\end{problem}

\begin{problem}
 Let $W_1(t)$ be a Brownian motion and $W_2(t)$ be another
  \emph{independent} Brownian motion, and $\rho$ is a constant between
  $-1$ and $1$.  Then consider the process $X(t) = \rho W_1(t) +
  \sqrt{1-\rho^2} W_2(t)$.  Is this $X(t)$ a Brownian motion?

\begin{sol}

 This problem is adapted from Baxter and Rennie, Problem 3.2, page 49.
Yes, $X(t)$ is a Brownian motion.  We can see this by verifying each
of the properties in turn:
\begin{enumerate}
  \item  \[
    X(t) - X(s) = \rho (W_1(t) - W_1(s)) + \sqrt{1 - \rho^2}( W_2(t) -
    W_2(s) )
   \]
   Since $(W_1(t) - W_1(s)) \sim N(0, t-s)$ and $(W_2(t) - W_2(s))
   \sim N(0, t-s)$ and  $W_1$ and $W_2$ are independent, so in
   particular the increments $(W_1(t) - W_1(s))$ and $(W_2(t) -
   W_2(s))$ are independent, then
   \[
      \rho (W_1(t) - W_1(s) + \sqrt{1 - \rho^2}( W_2(t) -
    W_2(s) ) \sim N(0, \rho^2 (t-s) + ( 1 - \rho^2) (t-s) ) =
      N(0,t-s).
   \]
  \item For every pair of disjoint time intervals $[t_1, t_2]$ and
          $[t_3, t_4]$, with $ t_1 \le t_2 \le t_3 \le t_4$, the
          increments
   \[
    X(t_2) - X(t_1) = \rho (W_1(t_2) - W_1(t_1) + \sqrt{1- \rho^2}( W_2(t_2) -
    W_2(t_1) )
   \]
   and
   \[
    X(t_4) - X(t_3) = \rho (W_1(t_4) - W_1(t_3) + \sqrt{1 - \rho^2}( W_2(t_4) -
    W_2(t_3) )
   \]
   are independent, since $(W_1(t_2) - W_1(t_1)$, $(W_2(t_2) -
   W_2(t_1)$, $(W_1(t_4) - W_1(t_3)$, and $(W_2(t_4) - W_2(t_3)$ are
   all independent of one another.
  \item
    \[
    X(0) = \rho W_1(0) + \sqrt{1 - \rho^2} W_2(0) = \rho 0 + \sqrt{1 - \rho^2} 0 = 0.
   \]
\item   Since $W_1(t)$ is continuous at $0$, and $W_2(t)$ is
  continuous at $0$, then $X(t) = \rho W_1(t) + \sqrt{1 - \rho^2}
  W_2(t)$ is continuous at $0$.
\end{enumerate}
\end{sol}
\end{problem}

\begin{problem}
 What is the distribution of $W(s) + W(t)$, for $0 \le s \le
  t$? (Hint: Note that $W(s)$ and $W(t)$ are not independent.  But you
  can write $W(s) + W(t)$ as a sum of independent variables.  Done
  properly, this problem requires almost no calculation.

\begin{sol}

\[
    W(s) + W(t) = (W(s) - W(0)) + W(s) + (W(t) - W(s)) = 2(W(s) -
    W(0)) + (W(t) - W(s))
\]
and $(W(s) - W(0))$ is independent from $(W(t) - W(s))$, so
\[
   W(s) + W(t) = 2( W(s) - W(0)) + (W(t) - W(s)) \sim N(0, 2^2 s +
   (t-s)) = N(0, 3 s + t).
\]
\end{sol}
\end{problem}

\begin{problem}
 Show that
   \[
      \operatorname{Cov}(W(s),W(t)) = E[W(s) W(t)] = \min(t,s)
   \]
  Assuming that a stock price moves according to Brownian motion,
  interpret this for a stock at $ t $ and $ t+1 $.

\begin{sol}

\[
   \operatorname{Cov}(W(s),W(t)) = E[W(s) - E[W(s)] \cdot W(t) -
   E[W(t)]] = E[W(s) \cdot W(t)].
\]
Without loss of generality, assume $s < t$.  Then
\[
   E[W(s) \cdot W(t)] = E[ W(s) \cdot ( W(t) - W(s) + W(s))]
   = E[ W(s) \cdot ( W(t) - W(s)) ] + E[ W(s) W(s))] = 0 +
   \D W(s) = s.
\]
The process is exactly analogous if $t < s$, so
\[
   \operatorname{Cov}(W(s),W(t)) = E[W(s) W(t)] = \min(t,s).
\]
The interpretation is that a stock above its mean value on day $
t $ is likely to also be above its mean on day $ t + 1 $.
\end{sol}
\end{problem}

\begin{problem}
  Show that the probability density function
  \[
      p(t;x,y) = \frac{1}{\sqrt{2\pi t}} \exp( -(x-y)^2/(2t) )
  \]
  satisfies the partial differential equation for heat flow (the
  \emph{heat equation})
  \[
     \frac{\partial p}{\partial t} = \frac{1}{2} \frac{\partial^2
     p}{\partial t^2}
  \]


\begin{sol}
\[
   \frac{\partial p}{\partial t}
    = \frac{-1}{\sqrt{2\pi}  t^{3/2} } \exp( -(x-y)^2/(2t) )
    + \frac{1}{\sqrt{2\pi t}} \exp( -(x-y)^2/(2t) ) \frac{ (x-y)^2}{2 t^2}
\]

\[
   \frac{\partial p}{\partial x}
    = \frac{1}{\sqrt{2\pi t}} \exp( -(x-y)^2/(2t) ) \frac{ (x-y)}{t}
\]

\[
   \frac{\partial^2 p}{\partial x^2}
    = \frac{1}{\sqrt{2\pi t}} \exp( -(x-y)^2/(2t) ) \frac{
   (x-y)^2}{t^2}
   + \frac{1}{\sqrt{2\pi t}} \exp( -(x-y)^2/(2t) ) \frac{1}{t}
\]
Now after a little combining,
the first summand in  $ \frac{\partial p}{\partial t} $ matches
the second summand in $ \frac{1}{2}  \frac{\partial^2 p}{\partial x^2}$
and the second summand in  $ \frac{\partial p}{\partial t} $ matches
the first summand in $ \frac{1}{2}  \frac{\partial^2 p}{\partial x^2}$.
\end{sol}
\end{problem}

\begin{problem}
 Let $ W(t) $ be standard Brownian motion.
  \begin{enumerate}
      \item
          Evaluate the probability that $ W(5) \le 3 $ given that $
          W(1) = 1 $.
      \item
          Find the number $ c $ such that $ \P[ W(9) > c | W(1)
          =1 ] = 0.10 $.
  \end{enumerate}


\begin{sol}

\begin{enumerate}
    \item
        Since $ W(5) - W(1) \sim N(0,4) $, the required
        probability is
        \begin{eqnarray*}
            \P[ W(5) > 3 | W(1) = 1] &=& \P[ W(5) - W(1) > 3 -
            1] \\
            &=& \P[ ( W(5) - W(1))/2 > 1] \\
            &=& 0.1586552539
        \end{eqnarray*}
    \item
        Since $ W(9) - W(1) \sim N(0,8) $, the required value
        can be deduced from
        \begin{eqnarray*}
            \P[ W(9) > c | W(1) = 1] &=& \P[ W(9) - W(1) > c -
            1] \\
            &=& \P[ ( W(9) - W(1))/(2\sqrt{2}) > (c -1)/(2\sqrt
            {2}) ] \\
            &=& 0.10
        \end{eqnarray*}
        Then $ (c -1)/(2 \sqrt{2}) = 1.281551566 $ and $ c=
        4.624775211 $.
\end{enumerate}
\end{sol}
\end{problem}

\begin{problem}
  Show that \( \E[T_a] = \infty \) for $ a > 0 $.

\begin{sol}

\end{sol}
\end{problem}

\begin{problem}
  Suppose that the fluctuations of a share of stock of a certain
  company are well described by a Brownian Motion process. Suppose
  that the company is bankrupt if ever the share price drops to
  zero.  If the starting share price is $ A(0) = 5 $, what is the
  probability that the company is bankrupt by $ t = 25 $?  What is
  the probability that the share price is above $ 10 $ at $ t = 25$?.
  %% Adapted from H. Taylor, S. Karlin, \emph{Introduction to
  %% Stochastic Modeling} 3rd Edition, Academic Press, 1998.


\begin{sol}

\end{sol}
\end{problem}

\begin{problem}
  Suppose you own one share of stock whose price changes according
  to a Brownian Motion Process.  Suppose you purchased the stock
  at a price $ b+c $, $ c > 0 $ and the present price is $ b $.
  You have decided to sell the stock either when it reaches the
  price $ b+c $ or when an additional time $ t $ goes by,
  whichever comes first.  What is the probability that you do not
  recover your purchase price?

\begin{sol}

\end{sol}
\end{problem}

\begin{problem}
 \tbf{The Distribution of the Maximum} \\
  Let $t$ be a given time, let $a>0$ be a given value, then
  \begin{eqnarray*}
    \P\{ \max_{0\le u \le t} W(u) \ge a \} &=& \P\{ T_a \le t\} \\
    &=& \frac{2}{\sqrt{2\pi}} \int_{a/\sqrt{t}}^\infty \exp(-y^2/2)\;dy
  \end{eqnarray*}

\begin{sol}

\end{sol}
\end{problem}

\begin{problem}
 \tbf{Quadratic Variation.} \\
  A function $f(x)$ is said to have \textbf{bounded variation} if, over the closed
  interval $[a,b]$, there exists an $M$ such that
  \[
    | f(t_1) - f(a)| + |f(t_2) - f(t_1)| + \ldots + |f(b) - f(t_n)| \le M
  \]
  \ni  for all partitions $a = t_0 < t_1 < t_2 < \ldots < t_n < t_{n+1} = b$
  of the interval.The idea is that we measure the total (hence the absolute value)
  up-and-down movement of a function.  This is roughly like the arclength of the
  graph of the function.  A monotone increasing or decreasing function has bounded
  variation.  An everywhere differentiable function has bounded variation.
  Some functions, for instance Brownian Motion do not have bounded variation. A function
  $f(t)$ is said to have \tbf{quadratic variation} if, over the closed interval $[a,b]$,
  there exists an $M$ such that
  \[
    (f(t_1) - f(a))^2 + (f(t_2) - f(t_1))^2 + \ldots + (f(b) - f(t_n))^2\le M
  \]
  \ni for all partitions $a=t_0<t_1<t_2<\ldots < t_n < t_{n+1} = b$ of the interval.
  Again, the idea is that we measure the total (hence the positive terms
  created by squaring) up-and-down movement of a function.  However, the squaring will
  make the small ups-and-downs smaller, so that perhaps a function without
  bounded variation may have quadratic variation.  In fact, this is the
  case for Brownian Motion. The \textbf{total quadratic variation} of $Q$ a function
  $f$ on an interval $[a,b]$ is
  \[
    Q = \sup_{P} \sum_{i=0}^{n} (f(t_{i+1}) - f(t_{i})^2
  \]
  where the supremum is taken over all partitions $P$ with
  $ a = t_0 < t _1 < \ldots < t_n < t_{n+1} = b$, with mesh size
  going to zero as the number of partition points $n$ goes to infinity. \\
  \ni\tbf{Quadratic Variation of Brownian Motion.} \\
  \ni We can guess that Brownian Motion might have quadratic variation by
  considering the quadratic variation of our coin-flipping fortune record
  first.  Consider the function piecewise linear function $S(t)$
  defined by $S_n = X_1 + \ldots + X_n$ with the Bernoulli random variables
  $X_i = +1$ with probability $p = 1/2$ and $X_i = -1$ with probability
  $q = 1-p = 1/2$. With a little analysis, it is easy to show that we need
  only consider the quadratic variation at points $1,2,3, \ldots, n$
  Then each term $(S(i+1) - S(i))^2 = X_{i+1}^2 = 1$.  Therefore, the
  quadratic variation is the total number of steps, $Q=n$. Now remember Brownian
  Motion is approximated by $(1/\sqrt{n}) S(nt)$.  Each step is size $1/\sqrt{n}$,
  then the quadratic variation of the step is $1/n$ and there are $n$ steps on $[0,1]$.
  The total quadratic variation of $(1/\sqrt{n})S(nt)$ on $[0,1]$ is $1$.
  We will not completely rigorously prove that the total quadratic variation of
  Brownian Motion is $t$, as claimed, but we will prove something very
  close to that.
  \begin{theorem}
  Let $W(t)$ be standard Brownian motion.  For every fixed $t > 0$
  \[
    \lim_{n \to \infty}
             \sum_{n=1}^{2^n}
               \left[
                  W\left( \frac{k}{2^n} t \right) -
                  W\left( \frac{k-1}{2^n} t \right)
               \right]^2
           = t
  \]
  \ni with probability $1$ (that is, almost surely).
  \end{theorem}
  !!!PROVE THE THEOREM!!!


\begin{sol}
  Introduce some briefer notation for the proof, let:
  \[
    \Dt_{nk} =
                  W\left( \frac{k}{2^n} t \right) -
                  W\left( \frac{k-1}{2^n} t \right)
          \qquad k = 1, \ldots, 2^n
  \]
  and
  \[
    W_{nk} = \Delta^{2}_{nk} - t/2^n \qquad k = 1, \ldots, 2^n.
  \]
  We want to show that $\sum_{k=1}^{2^n} \Delta^2_{nk} \to t$ or what
  is equivalent: $\sum_{k=1}^{2^n} W_{nk} \to 0$.  For each $n$, the
  random variables $W_{nk}, k = 1, \ldots, 2^n$ are independent, and
  identically distributed by properties 1 and 2 of the definition of
  standard Brownian motion.  Furthermore,
  \[
    E[ W_{nk} ] = E[ \Delta_{nk}^2 ] - t/2^n = 0
  \]
  by property 1 of the definition of standard Brownian motion.  Also, a
 routine (but omitted)
  computation of the fourth moment of the normal distribution
  shows that
  \[
    E[W^2_{nk} ] = 2t^2/4^n
  \]
  (Note that one way to do the computation is to differentiate the
  characteristic function $4$ times!)
  Finally, by property 2 of the definition of standard Brownian motion
  \[
    E[ W_{nk}W_{nj} ] = 0, k \ne j
  \]
  Now, expanding the square of the sum, and applying all of these
  computations
  $$
    E\left[ \left\{ \sum_{k=1}^{2^n} W_{nk} \right\}^2 \right] =
    \sum_{k=1}^{2^n} E[W^2_{nk}] = 2^{n+1}t^2/4^n = 2t^2/2^n
  $$
  Now apply Chebyshev's Inequality to see:
  $$
    \P\left[ \left| \sum_{k=1}^{2^n} W_{nk} \right| > \epsilon \right]
     \le \frac{2 t^2}{\epsilon^2} \left( \frac{1}{2}
    \right)^n
  $$
  Now since $\sum (1/2)^2$ is a convergent series, the Borel-Cantelli
  lemma implies that the event
  $$
    \left| \sum_{k=1}^{2^n} W_{nk} \right| > \epsilon
  $$
  can occur for only finitely many $n$.  That is, for any $\epsilon >
  0$, there is an $N$, such that for $n > N$
  $$
    \left| \sum_{k=1}^{2^n} W_{nk} \right| < \epsilon
  $$
  therefore we must have that
  $\lim_{n \to \infty} \sum_{k=1}^{2^n} W_{nk} = 0$,
  and we have established what we
  wished to show.
\end{sol}
\end{problem}

\begin{problem}
 {\it Comment}:  Starting from
  $$
    \lim_{n \to \infty}
             \sum_{n=1}^{2^n}
               \left[
                  W\left( \frac{k}{2^n} t \right) -
                  W\left( \frac{k-1}{2^n} t \right)
               \right]^2
           = t
  $$
  and without thinking too carefully about what it might mean, we can
  imagine an elementary calculus limit to the left side and write the
  formula:
  $$
    \int_0^t [ dW(\tau) ]^2 = t = \int_0^t d\tau
  $$
  In fact, with more advanced mathematics this can be made sensible ad
  mathematically sound.  Now from this relation, we could write it in
  differential form:
  $$
    dW(\tau)^2 = d\tau.
  $$
  \ni The important thing to remember here is that the formula suggests
  that Brownian motion has differentials that cannot be ignored in
  second (or squared, or quadratic) order.  Brownian motion ``wiggles''
  too much, even the total of the squared differences add up!  In
  retrospect, this is not so surprising given the law of the iterated
  logarithm.  We know that in any neighborhood $[t, t + dt]$ to the
  right of $t$, Brownian motion must come close to $\sqrt{2t \log \log
  t}$.  That is, intuitively, $W(t+dt) - W(t)$ must be about $\sqrt{2
  dt}$ in magnitude, so we would guess $dW^2 \approx 2dt$  The theorem
  makes it precise.

\begin{sol}

\end{sol}
\end{problem}

\begin{problem}
 {\it Comment}:
  Here's a more rigorous and somewhat different explanation of why the squared
  variation of Brownian motion may be guessed to be $t$:  Consider
  $$
      \sum_{i=1}^{n} \left( W\left( \frac{it}{n} \right) - W\left(
      \frac{(i-1)t}{n} \right)  \right)^2.
  $$
  Now let
  $$
  Z_{ni} = \frac{ \left( W\left( \frac{it}{n} \right) - W\left(
      \frac{(i-1)t}{n} \right) \right) }{ \sqrt{t/n} }
  $$
  Then for each $n$, the sequence $Z_{ni}$ is a sequence of independent,
  identically distributed standard normal $N(0,1)$ normal random
  variables.
  Now we can write the quadratic variation as:
  $$
      \sum_{i=1}^{n} \frac{t}{n} Z_{ni}^2 = t \left( \frac{1}{n}
      \sum_{i=1}^{n} Z_{ni}^2 \right)
  $$
  But notice that the expectation $\E(Z_{ni}^2)$ of each term is the same
  as calculating the variance of a standard normal $N(0,1)$ which is of
  course $1$..  Then the last term in parentheses above converges by the weak law of
  large numbers to $1$!  Therefore the quadratic variation of Brownian
  motion converges to $t$.  This little proof is in itself not
  sufficient to prove the theorem above because it relies on the weak
  law of large of number, hence establishes convergence in distribution
  only, and for the theorem above we want convergence almost surely.

\begin{sol}

\end{sol}
\end{problem}

\begin{problem}
  Докажите, что условное распределение $W(s)$ при условии $X(t)=B$,
  $s < t $, является нормальным со средним $B s/t$ и дисперсией $(s/t)(t-s)$:
  $$
    \P\left\{ W(s) \in (x,x+\Delta x) |\, W(t) = B\right\} =
    \frac{1}{\sqrt{2\pi (s/t)(t-s)}}\cd \exp(-(x-Bs/t)^2/2(t-s)) \cd\Dt x.
  $$


\begin{sol}

The conditional density is
\begin{eqnarray*}
  f_{s|t}(x|B) &=& ( f_s(x) f_{t-s}(B-x) )/ f_t(B) \\
  &=& K_1 \exp( -x^2/(2s) - (B-x)^2/( 2(t-s) ) ) \\
  &=& K_2 \exp(-x^2(1/(2 s)+ 1/(2 (t-s) )) + Bx/(t-s) ) \\
  &=& K_2 \exp( -t/( 2 s (t-s) ) ( x^2 - 2 s B x/t) ) \\
  &=& K_3 \exp( - (t (x - B s/t)^2/ (2 s (t-s)) ) )
\end{eqnarray*}
where $ K_1 $, $ K_2 $, and $ K_3 $ are constants that do not depend on $
x $.  For example, $ K_1 $ is the product of $ 1/\sqrt{2 \pi s} $ from
the $ f_s(x) $ term, and $ 1/\sqrt{2 \pi (t-s)} $ from the $ f_{t-s}(B-x)
$ term, times the $ 1/f_t(B) $ term in the denominator.  The $ K_2 $
term multiplies in an $ \exp(-B^2/(2 (t-s))) $ term.  The $ K_3 $ term
comes from the adjustment in the exponential to account for completing
the square.  We know that the result is a conditional density, so the $
K_3 $ factor must be the correct normalizing factor, and we recognize
from the form that the result is a normal distribution with mean $ B s/t
$ and variance $ (s/t) (t-s) $.
\end{sol}
\end{problem}

\begin{problem}
 Условная плотность $W(t)$, где $t_1 < t < t_2,$ при условии, что
  $W(t_1)=A$ и $W(t_2) = B,$ является нормальным со средним
  \[
    A + ( (B-A)/(t_2-t_1) ) (t-t_1)
  \]
  \ni и дисперсией
  \[
    (t_2-t)(t-t_1)/(t_2-t_1).
  \]


\begin{sol}
$ X(t) $ subject to the conditions $ X(t_1) = A $ and $ X(t_2)
= B $ has the same density as the random variable $ A + X(t-t_1) $,
under the condition $ X(t_2 - t_1) = B - A $ by condition (2) of the
definition of Brownian motion.  Then apply the theorem with $ s = t-t_1 $
and $ t = t_2 - t_1 $.
\end{sol}
\end{problem}

\begin{problem}
\tbf{Quadratic Variation.}
  \begin{theorem}
  Let $W(t)$ be standard Brownian motion.  For every fixed $t > 0$
  \[
    \lim_{n \to \infty}
    \sum_{n=1}^{2^n}
      \left[
         W\left( \frac{k}{2^n} t \right) -
         W\left( \frac{k-1}{2^n} t \right)
      \right]^2
		= t
  \]
  \ni with probability $1$ (that is, almost surely).
  \end{theorem}


\begin{sol}
Introduce some briefer notation for the proof, let:
\[
  \Dt_{nk} =
                W\left( \frac{k}{2^n} t \right) -
                W\left( \frac{k-1}{2^n} t \right)
        \qquad k = 1, \ldots, 2^n
\]
and
\[
  W_{nk} = \Delta^{2}_{nk} - t/2^n \qquad k = 1, \ldots, 2^n.
\]
We want to show that $\sum_{k=1}^{2^n} \Delta^2_{nk} \to t$ or what
is equivalent: $\sum_{k=1}^{2^n} W_{nk} \to 0$.  For each $n$, the
random variables $W_{nk}, k = 1, \ldots, 2^n$ are independent, and
identically distributed by properties 1 and 2 of the definition of
standard Brownian motion.  Furthermore,
$$
  E[ W_{nk} ] = E[ \Delta_{nk}^2 ] - t/2^n = 0
$$
by property 1 of the definition of standard Brownian motion.  Also, a
routine (but omitted)
computation of the fourth moment of the normal distribution
shows that
$$
  E[W^2_{nk} ] = 2t^2/4^n
$$
(Note that one way to do the computation is to differentiate the
characteristic function $4$ times!)
Finally, by property 2 of the definition of standard Brownian motion
$$
  E[ W_{nk}W_{nj} ] = 0, k \ne j
$$
Now, expanding the square of the sum, and applying all of these
computations
$$
  E\left[ \left\{ \sum_{k=1}^{2^n} W_{nk} \right\}^2 \right] =
  \sum_{k=1}^{2^n} E[W^2_{nk}] = 2^{n+1}t^2/4^n = 2t^2/2^n
$$
Now apply Chebyshev's Inequality to see:
$$
  \P\left[ \left| \sum_{k=1}^{2^n} W_{nk} \right| > \epsilon \right]
   \le \frac{2 t^2}{\epsilon^2} \left( \frac{1}{2}
  \right)^n
$$
Now since $\sum (1/2)^2$ is a convergent series, the Borel-Cantelli
lemma implies that the event
$$
  \left| \sum_{k=1}^{2^n} W_{nk} \right| > \epsilon
$$
can occur for only finitely many $n$.  That is, for any $\epsilon >
0$, there is an $N$, such that for $n > N$
$$
  \left| \sum_{k=1}^{2^n} W_{nk} \right| < \epsilon
$$
therefore we must have that
$\lim_{n \to \infty} \sum_{k=1}^{2^n} W_{nk} = 0$,
and we have established what we
wished to show.
\end{sol}
\end{problem}

\begin{problem}
 The above theorem can be amusingly summarized in the following
  way:  Let $dW(t) = W(t + dt) - W(t)$.  Let $dW(t)^2 = ( W(t + dt) -
  W(t) )^2.$  Then (although mathematically not rigorously) we can say:
  \begin{eqnarray*}
     dW(t) &\sim& N(0, dt) \\
     (dW(t))^2 &\sim& N(dt, 0). \\
  \end{eqnarray*}

  \begin{theorem}
  $$
   \lim_{n \to \infty}
            \sum_{n=1}^{2^n}
              \left|
                 W\left( \frac{k}{2^n} t \right) -
                 W\left( \frac{k-1}{2^n} t \right)
              \right|
          = \infty
  $$
  In other words, the total variation of a Brownian path is infinite,
  with probability $1$.
  \end{theorem}


\begin{sol}
$$
    \sum_{n=1}^{2^n}
      \left|
         W\left(  \frac{k}{2^n} t \right) -
         W\left( \frac{k-1}{2^n} t \right)
      \right|
   \ge
   \frac{
                 \sum_{n=1}^{2^n}
      \left|
         W\left(  \frac{k}{2^n} t \right) -
         W\left( \frac{k-1}{2^n} t \right)
      \right|^2
       }{
\max_{j=1, \ldots, 2^n} \left|
         W\left(  \frac{k}{2^n} t \right) -
         W\left( \frac{k-1}{2^n} t \right)
      \right|
     }
$$
The numerator on the right converges to $t$, while the denominator
goes to $0$ because Brownian paths are continuous, therefore
uniformly continuous on bounded intervals.  Therefore the faction
on the right goes to infinity.
\end{sol}
\end{problem}

\begin{problem}
 \tbf{Hitting Times.} \\
  Consider standard Brownian Motion $W(t)$, which starts at $W(0) = 0.$
  Let $ a > 0 $.  Let us denote the \emph{hitting time} $ T_a $ be the
  first time the Brownian Motion hits $ a $.  Specifically in notation
  from analysis
  $$
    T_a = \inf \{t > 0 :  W(t) = a \}.
  $$
  Note the very strong analogy with the duration of the game in the
  gambler's ruin. Some Brownian paths will hit $a>0$ fairly directly,
  others will make an excursion ( for example, to negative values) and take
  a long time to finally reach $a$. Thus $T_a$ will have a probability
  distribution. We will determine that distribution by a procedure similar
  to the first step analysis we made for coin-flipping fortunes. More specifically,
  we will consider a probability by conditioning, conditioning on whether or not
  $T_a \le t$, for some given value of $t$.
  $$
    \P\{W(t) \ge a\} = \P\{ W(t) \ge a | T_a \le t\} \P\{T_a \le t\} +
    \P\{W(t) \ge a | T_a > t\} \P\{T_a > t\}
  $$
  Now note that the second conditional probability is $0$ because it is
  an empty event.  Therefore:
  $$
    \P\{ W(t) \ge a\} = \P\{ W(t) \ge a | T_a \le t\} \P\{T_a \le t\}.
  $$
  Now, consider Brownian Motion ``started over'' again the time $T_a$
  when it hits $a$.  By the shifting transformation from the previous
  section, this would have just the distribution of Brownian Motion again,
  and so
  \begin{eqnarray*}
    \P\{ W(t) \ge a | T_a \le t\} &=& \P\{ W(t) \ge a | W(T_a)=a,T_a\le t\} \\
    &=&= \P\{ W(t) - W(T_a) \ge 0 | T_a \le t\} \\
    &=& 1/2.
  \end{eqnarray*}

  \vsiii\ni Actually, this ``proof'' contains a serious logical gap, since $ T_a $
  is a ``random time'' not a fixed time, whereas the shifting
  transformation depends on having a fixed time.  We ought to make sure
  that ``random times'' act like fixed times.  Under special conditions,
  random times can act like fixed times.  Specifically, the proof can be
  fixed and made completely rigorous by showing that Brownian motion has
  the strong Markov property and that $T_a$ is a Markov time corresponding
  to the event of first passage from $0$ to $a$. This argument is a specific
  example of the Reflection Principle for Brownian Motion. It says that Brownian
  Motion reflected about a first passage has the same distribution as the original
  motion. Thus
  $$
    \P\{W(t) \ge a\} = (1/2) \P\{T_a \le t\}.
  $$
  \ni or
  \begin{eqnarray*}
    \P\{ T_a \le t\} &=& 2 \P\{ W(t) \ge a\} \\
    &=& \frac{2}{\sqrt{2\pi t}} \int_a^\infty \exp(-x^2/(2t)) \; dx \\
    &=& \frac{2}{\sqrt{2\pi}} \int_{a/\sqrt{t}}^\infty \exp(-y^2/2) \;dy
  \end{eqnarray*}
  (note the change of variables $y = x/\sqrt{t}$ in the second integral).
  One can easily differentiate to obtain the probability distribution function
  (p.d.f)
  $$
    f_{T_a}(t) = \frac{a}{\sqrt{2\pi}} t^{-3/2} \exp(-a^2/(2t)).
  $$
  \ni Note that this is much stronger than the analogous result for the
  duration of the game until ruin in the coin-flipping game.  There we
  were only able to derive an expression for the expected value of the
  hitting time, not the probability distribution of the hitting time. Now
  we are able to derive the probability distribution of the hitting time
  fairly intuitively (although strictly speaking there is a gap). Here is
  a place where it is simpler to derive a quantity for Brownian Motion
  than it is to derive the corresponding quantity for random walk. \\
  \ni RESTORE THE DETAILS and GET RID OF NUMEROUS GAPS.

\begin{sol}

\end{sol}
\end{problem}

\begin{problem}
 \tbf{Hitting Times Continued.} \\
  Let us now consider the probability that Brownian motion hits $a > 0$,
  before hitting $-b<0$, where $b>0$. To compute this we will make use of
  the interpretation of Standard Brownian Motion as being the limit of the
  symmetric random walk. PROVE that in the case of the fair ($p = 1/2 = q$)
  coin-flipping game the probability that the random walk goes up to
  $a$ before going down to $b$ when the step size is \(\Dt x \) is
  $$
    \P\{\text{ to $a$ before $-b$ }\} = \frac{b \Dt x}{(a + b)\Dt x} = \frac{b}{a+b}.
  $$
  \ni Here is a place where it is easier to derive the result from the
  coin-flipping game and pass to the limit than to derive the result from
  Brownian Motion principles.

\begin{sol}

\end{sol}
\end{problem}

\begin{problem}
 \tbf{VERY DIFFICULT.} Prove that with probability $1$ (i.e. almost surely) a
  Brownian Motion is nowhere (except possibly on set of Lebesgue measure $0$)
  differentiable.

\begin{sol}

\end{sol}
\end{problem}

\begin{problem}
 Prove that a Brownian Motion path has no intervals of monotonicity.
  That is, there is no interval $[a,b]$ with $W(t_2)- W(t_1)>0$
  (or $W(t_2)-W(t_1)<0$) for all \(t_2,t_1\in [a,b]\) with $t_2>t_1.$

\begin{sol}

\end{sol}
\end{problem}

\begin{problem}
 Show that
  \begin{eqnarray*}
    \limsup_{n \to \infty} \frac{W(n)}{\sqrt{n}} &=& +\infty \\
    \liminf_{n \to \infty} \frac{W(n)}{\sqrt{n}} &=& -\infty \\
  \end{eqnarray*}

\begin{sol}

\end{sol}
\end{problem}

\begin{problem}
 Show that for arbitrarily large $t_1$, there is a $t_2>t_1$
  such that $W(t_2)=0$. That is, Brownian Motion paths cross the time-axis for
  arbitrarily large values of $t$.

\begin{sol}

\end{sol}
\end{problem}

\begin{problem}
 Using the fact that the inversion $t W(1/t)$ is also a standard
  Brownian motion show that $0$ is an accumulation  point of the zeros of $W(t)$.
  That is, Standard Brownian Motion crosses the time axis arbitrarily near $0$.

\begin{sol}

\end{sol}
\end{problem}

\begin{problem}
  \begin{enumerate}
    \item What is the probability that Geometric Brownian
    Motion with parameters $r = 0$ and $\sigma$ ever rises to more than
    twice its original value? In economic terms, if you buy
    stock whose fluctuations are described by Geometric
    Brownian Motion, what are your chances to double your
    money?

    \item What is the probability that Geometric Brownian
    Motion with parameters $r = -\sigma^2/2$ and $\sigma$ (so
    that the mean is constant) ever rises to more than twice
    its original value?  In economic terms, if you buy stock
    whose fluctuations are described by this Geometric Brownian
    Motion, what are your chances to double your money?
  \end{enumerate}

\begin{sol}

\begin{enumerate}
  \item Use the ``gambler's ruin'' formula for Geometric
  Brownian Motion:
  \[
     \P\left[
          \frac{ z_0 \exp( r T_{A,B}
                       + \sigma W(T_{A,B}))}{z_0} = B
          \right] =
          \frac{1 - A^{1 - (2r-\sigma^2)/\sigma^2}}
               {B^{1 - (2r-\sigma^2)/\sigma^2}
               - A^{1 - (2r-\sigma^2)/\sigma^2}}
  \]
  with $A = 0$ and $B = 2$ to obtain
  \[
     \P\left[
          \exp(\sigma W(T_{0,2})) = 2
          \right] =
          \frac{1}
               {2^{1 - (-\sigma^2)/\sigma^2}} =
              \frac{1}{2^2} = \frac{1}{4}
  \]

  \item Use the ``gambler's ruin'' formula for Geometric
  Brownian Motion again:
  with $A = 0$ and $B = 2$ to obtain
  \[
     \P\left[
          \exp(\sigma W(T_{0,2})) = 2
          \right] =
          \frac{1}
               {2^{1 - (-\sigma^2-\sigma^2)/\sigma^2}} =
            = \frac{1}{2^3} = \frac{1}{8},
  \]

\end{enumerate}
\end{sol}
\end{problem}

\begin{problem}
 Половина доказательства изометрии \\
Пусть $X(t)$ — адаптированный (согласованный) случайный процесс. \\
Пусть $t_{1}<t_{2}<\ldots<t_{n}$ — фиксированные моменты времени. \\


\begin{sol}
\[
E\left[\left(\sum_{i=1}^{n}X(t_{i})(W(t_{i+1})-W(t_{i}))\right)^{2}\right]=
  \sum_{i=1}^{n}\E(X^{2}(t_{i}))(t_{i+1}-t_{i})
\]
\end{sol}
\end{problem}

\begin{problem}
For standard Brownian Motion $W(t)$ find the probability $p \{ W(1)>0 \& W(2)<0 \}$


\begin{sol}
Straight calculation with densities gives
\[
  1/2\pi \cdot \int_0^{+\infty}{dx} \int_{-\infty}^0 dy \cdot exp(-x^2/2) \cdot exp(-(x-y)^2/2)
\]
but this is not the way you are supposed to find out the answer.
Just remember the property of independent increments.
We know, that $W(1)$ and $W(2)-W(1)$ are independent and even identically distributed.
So «we cannot distinguish»  them $\P( |W(2)-W(1)| > |W(1)| ) = 1/2$. Thus,
\[
  P\{ W(1)>0, W(2)<0 \} = P\{ W(1)>0, W(2)-W(1) < 0, |W(2)-W(1)| > |W(1)| \}
\]
For Brownian motion, signs and lengths are independent, so,
\[
  P\{ W(1)>0, W(2)<0 \} = P\{ W(1)>0) \P(W(2)-W(1) < 0\} \cdot P\{|W(2)-W(1)| > |W(1)|\} = 1/2*1/2*1/2=1/8.
\]
\end{sol}
\end{problem}

\begin{problem}
Let $B(t)$ and $W(t)$ be independant Brownian Motions. Find $E\max\{B(t),W(t)\}$


\begin{sol}
Note that
\[
  E\ \max\{B(t),W(t)\} =  E \{B(t)\} + E\{(W(t) - B(t))\cdot I_{W(t) - B(t) >0}\}.
\]
$W(t) - B(t)$ is a normal variate with mean $0$, variance $2t$, so the second expectation is
\[
  1/(4\pi t)\int_0^\infty y \exp(y^2/(4t))dy,
\]
which I think is $\sqrt{t/\pi}$.
\end{sol}
\end{problem}

\begin{problem}
Петя и Вася сделали ставки по 1 рублю на броуновское движение. Петя выигрывает у Васи ставку, если $W(t)$ достигнет уровня 1 раньше, чем уровня (-1) и наоборот. В игре можно удваивать ставки в любой момент времени. Если от одного игрока поступило предложение удвоить ставки, то другой игрок в ответ на это может либо сдаться (и потерять текущую ставку), либо согласиться на удвоение ставки. Один и тот же игрок не может удваивать ставку два раза подряд. Других запретов на удвоение ставки нет.
\begin{enumerate}
\item Какова оптимальная стратегия?
\item Дисперсия выигрыша при использовании обоими игроками оптимальных стратегий?
\end{enumerate}


\begin{sol}
Для «положительного» игрока: \\
- предлагать удвоение при $W(t)\geq 0.6$ \\
- соглашаться на удвоение при $W(t)\geq -0.6$ \\
решение: \\
Допустим на уровне $-a$ игроку безразлично, соглашаться или нет. Если не согласиться, получим проигрыш в ставку. Если согласиться, то возможно две ситуации: либо проиграем две ставки с вероятностью $\frac{2a}{1+a}$, либо достигаем уровня $a$, где противник безразличен между проигрышем в удвоенную ставку и еще чем-то. \\
Получаем: $-1=\frac{2a}{1+a}(-2)+\frac{1-a}{1+a}2$ \\
б) может быть явно не решается, а может и решается\ldots\\
Source: \url{plus.math.org/issue15/features/doubling}
\end{sol}
\end{problem}

\begin{problem}
С помощью компьютера постройте пять реализаций процессов:
\begin{enumerate}
\item $X_{t}=W_{t}$
\item $X_{t}=W_{\frac{t}{1-t}}$
\item $X_{t}=W_{t\wedge \tau}$, где $\tau=min\{t|W_{t}=-1\}$
\item $X_{t}=
\begin{cases}
W_{\frac{t}{1-t}\wedge\tau}, t\in [0;1) \\
-1, t\in[1;\infty)
\end{cases}$
\item $X_{t}=Z\sqrt{t}$, где $Z\sim N(0;1)$
\end{enumerate}

\begin{sol}

\end{sol}
\end{problem}

\begin{problem}
For $0<s<t$ calculate the probability $\P(W_t>0 \mid W_s>0)$ and $\P(W_s>0 \mid W_t>0)$.

\begin{sol}
Перейдем к двум независимым нормальным переменным
\end{sol}
\end{problem}

\begin{problem}
Цель этого упраженения — разобраться, сколько времени процесс $Y_t=at+W_t$ в среднем проводит ниже нуля. Другими словами, мы хотим найти $\E(T)$, где
\begin{equation}
T=\int_0^{+\infty} 1_{Y_t<0}\, dt
\end{equation}

\begin{enumerate}
\item Найдите $\E(T)$ для $a=1$
\item Используя тот факт, что $\frac{1}{k}W_{k^2t}$ — также броуновское движение, докажите, что величина $E(T)$ обратна пропорциональна $a$
\item Найдите $\E(T)$ для произвольного $a$
\item Найдите $\E(T)$, если $Y_t=\mu t+\sigma W_t$
\end{enumerate}

\begin{sol}
Для нахождения интеграла можно изменить порядок интегрирования. $\E(T)=\frac{\sigma^2}{2\mu^2}$
\end{sol}
\end{problem}

\begin{problem}
Цель этого упражнения — разобраться, сколько времени процесс $Y_t=at+W_t$ в среднем проводит в полоске $[0;y]$. Другими словами, мы хотим найти $\E(T)$, где
\begin{equation}
T=\int_0^{+\infty} 1_{Y_t\in [0;y]}\, dt
\end{equation}

\begin{enumerate}
\item ????
\item Найдите $\E(T)$, если $Y_t=\mu t+\sigma W_t$
\end{enumerate}

\begin{sol}
Для нахождения интеграла можно изменить порядок интегрирования. $\E(T)=y/\mu$
\end{sol}
\end{problem}

\begin{problem}
Найдите закон распределения случайной величины $2W_1-W_2$.

\begin{sol}
$2W_1-W_2=W_1-(W_2-W_1)$, величины $W_2-W_1$ и $W_1=W_1-W_0$ независимы, следовательно, $2W_2-W_1\sim N(0,2)$.
\end{sol}
\end{problem}


\section{Интеграл Ито}

\begin{problem}
Пусть $X_{t}=
\begin{cases}
1, & t\in[0;1) \\
-2, & t\in[1;2) \\
W_{1.5},& t\in[2;\infty) \\
\end{cases}$

Найдите $\int_{0}^{t}X_{u}dW_{u}$

\begin{sol}

\end{sol}
\end{problem}

\begin{problem}
Пусть $X_{t}=cos(t^{2}\cdot W_{t}^{3})$
\begin{enumerate}
\item  Найдите $dX_{t}$.
\item Перепишите найденное выражение для $dX_{t}$ с помощью интегралов.
\end{enumerate}

\begin{sol}

\end{sol}
\end{problem}

\begin{problem}
Для следующих случайных процессов найдите $dZ_{t}$ и выпишите соответствующую формулу в полной записи (с интегралами вместо $d$):
\begin{enumerate}
\item $Z_{t}=cos(t^{2}\cdot W_{t}^{3})$
\item $Z_{t}=W^{3}_{t}$ \\
\item $Z_{t}=t^{2}$ \\
\item $Z_{t}=e^{\alpha W_{t}}$ \\
\item $Z_{t}=\int_{0}^{t} aW_{a}dW_{a}$ \\
\item $Z_{t}=X^{2}_{t}$, где $dX_{t}=\alpha X_{t}dt+\sigma X_{t}dW_{t}$ \\
\item $Z_{t}=X^{-1}_{t}$, где $dX_{t}=\alpha X_{t}dt+\sigma X_{t}dW_{t}$
\end{enumerate}

\begin{sol}

\end{sol}
\end{problem}

\begin{problem}
 $dX_{t}=\alpha X_{t}dt+\sigma X_{t}dW_{t}$, $dY_{t}=\gamma Y_{t}dt+\delta Y_{t}dW_{t}$. $Z_{t}=X_{t}/Y_{t}$. Найдите $dZ_{t}$

\begin{sol}

\end{sol}
\end{problem}

\begin{problem}
 $dX_{t}=\alpha X_{t}dt+\sigma X_{t}dW_{1t}$, $dY_{t}=\gamma Y_{t}dt+\delta Y_{t}dW_{2t}$. $W_{1t}$ и $W_{2t}$ — два независимых винеровских процесса. $Z_{t}=X_{t}Y_{t}$. Найдите $dZ_{t}$

\begin{sol}

\end{sol}
\end{problem}

\begin{problem}
Являются ли процессы мартингалами:
\begin{enumerate}
\item $X_{t}=W_{t}^{2}-t$
\item  $\exp(-uW_{t}-\frac{u^{2}}{2}t)$
\end{enumerate}


\begin{sol}
$X_{t}=W_{t}^{2}-t$ — да, $\exp(-uW_{t}-\frac{u^{2}}{2}t)$ — да
\end{sol}
\end{problem}

\begin{problem}
Подберите нетривиальную функцию $f(t)$ так, чтобы процесс $X_t$ был мартингалом:
\begin{enumerate}
\item $X_t=f(t) \cos W_t$
\item $X_t=f(t) \sin W_t$
\end{enumerate}


\begin{sol}
$f(t)=e^{t/2}$
\end{sol}
\end{problem}




\begin{problem}
Пусть $f(t)$ — неслучайная дифференцируемая функция и $\int_{0}^{t}f^{2}(s)ds<\infty$ \\
Найдите интеграл $\int_{0}^{t}f(s)dW(s)$ \\
Уточнение: «найдите» в данном случае означает выразите через более привычный интеграл Римана

\begin{sol}
 $d(f(t)W_t)=f(t)dW_t+f'(t)W_tdt$, следовательно $\int_0^t f(s)dW_s=f(t)W_t-\int_0^t f'(s)W_sds$
\end{sol}
\end{problem}

\begin{problem}
Пусть $X(t)$ — геометрическое броуновское движение, т.е. $dX=\alpha Xdt+\sigma XdW$, и $Y=X^{n}$ \\
Докажите, что $Y$ — тоже геометрическое броуновское движение

\begin{sol}

\end{sol}
\end{problem}

\begin{problem}
считаем $\int B_{t}dt$ — разными способами: aops, 198662

\begin{sol}

\end{sol}
\end{problem}

\begin{problem}
 Определим $X_t=\int_0^t \exp(as)\,dW_s$. Найдите $\E(X_t)$, $\E(W_t\cdot X_t)$, $\E(X_t\cdot X_s)$

\begin{sol}
 $0$, Используя изометрию Ито:
\begin{equation}
\E(W_t\cdot X_t)=\E\left(\int_0^t\,dW_s \int_0^t\exp(as)\,dW_s\right)=\int_0^t 1\cdot \exp(as)\,ds
\end{equation}
\end{sol}
\end{problem}

\begin{problem}
Используя лемму Ито и свойства стохастического интеграла найдите
\begin{enumerate}
\item $dW^4_t$ и $\E(W^4_t)$
\item $dW^6_t$ и $\E(W^6_t)$
\item $dW^8_t$ и $\E(W^8_t)$
\item $dW^{2n}_t$ и $\E(W^{2n}_t)$
\end{enumerate}

\begin{sol}
 $\E(W^{2n}_t)=1\cdot 3\cdot 5 \cdots (2n-3)(2n-1)t^n$
\end{sol}
\end{problem}

\begin{problem}
Найдите производную $dY_t/dt$, если она существует
\begin{enumerate}
\item $Y_t=\int_0^t W_s \, ds$
\item $Y_t=\exp (\int_0^t W_s \, ds)$
\item $Y_t=\int_0^t W_s \, dW_s$
\end{enumerate}


\begin{sol}
\begin{enumerate}
\item $Y_t=\int_0^t W_s \, ds$, $dY_t/dt=W_t$
\item $Y_t=\exp (\int_0^t W_s \, ds)$, $dY_t/dt=W_t\exp (\int_0^t W_s \, ds)$
\item $Y_t=\int_0^t W_s \, dW_s$, не существует
\end{enumerate}
\end{sol}
\end{problem}

\begin{problem}
Постройтие две реализации случайных процессов:
\begin{enumerate}
\item $X_t=W_t$
\item $X_t=\int_0^t W_s \, ds$
\item $X_t=\int_0^t s \, dW_s$
\item $X_t=\int_0^t W_s \, dW_s$
\item Броуновского движения со сносом, $X_t=t+W_t$
\item Геометрического броуновского движения, $X_t=\exp(t+W_t)$
\end{enumerate}


\begin{sol}
\end{sol}
\end{problem}

\begin{problem}
Найдите интеграл Ито $\int_0^t e^{W_u - 0.5 u} \, dW_u$


\begin{sol}
Способ раз. Продифференцировать подынтегральное выражение. Способ два. Записать этот интеграл в краткой записи и получить обычное дифференциальное уравнение.
\end{sol}
\end{problem}

\begin{problem}
  Известно, что $X_0=3$, $dX_t = 5dW_t + tdt$. Найдите $X_t$, $\E(X_t)$ и $\Var(X_t)$.
\begin{sol}
  $X_t = 3 + 5W_t + t^2/2$
\end{sol}
\end{problem}

\begin{problem}
  Известно, что $X_0=2$, $dX_t = W_tdW_t + 3dt$. Найдите $X_t$, $\E(X_t)$ и $\Var(X_t)$.
\begin{sol}
  $X_t = 2 + W_t^2/2 + 2.5t$
\end{sol}
\end{problem}


\section{Стохастические ДУ}

\begin{problem}
Пусть случайные процессы $X$ и $Y$ являются решениями системы стохастических дифференциальных уравнений:
\[
\begin{cases}
dX=aXdt-YdW, X(0)=x_{0} \\
dY=aYdt+XdW, Y(0)=y_{0}
\end{cases}
\]
\begin{enumerate}
\item Верно ли, что процесс $R(t)=X^{2}(t)+Y^{2}(t)$ является детерминистическим?
\item Найдите $\E(X(t))$.
\end{enumerate}
% в) С помощью системы стохастических дифференциальных уравнений опишите то, что логично было бы назвать «броуновским движением на окружности» (??? не уверен)


\begin{sol}

\end{sol}
\end{problem}

\begin{problem}
 Vasicek interest rate model \\
Пусть процентная ставка удовлетворяет уравнению: \\
$dR=a(b-R)dt+\sigma dW$, $a$, $b$, $\sigma$ — положительные константы. $R(0)$ — также константа.
\begin{enumerate}
\item Прокомментируйте смысл $b$, почему $a$ положительное?
\item Решите уравнение, то есть найдите $R(t)$ (в ответе будет один неупрощаемый интеграл Ито).
\item Найдите $\E(R(t))$, $\Var(R(t))$.
\item Найдите $\lim_{t\to\infty}\E(R(t))$ и $\lim_{t\to\infty}\Var(R(t))$.
\end{enumerate}

\begin{sol}


$ r(t) = r(0) e^{-a t} +  b \left(1- e^{-a t}\right) + \sigma e^{-a t}\int_0^t e^{a s}\,dW_s$ \\
$\E(r_t)=r_{0}e^{-at}+b(1-e^{-at})$ \\
$\Var(r_t) = \frac{\sigma^2}{2 a}(1 - e^{-2at})$
\end{sol}
\end{problem}

\begin{problem}
Решите $dX_t=\frac{1}{X_t}dt+X_tdW_t$ с начальным условием  $X_0=x$

\begin{sol}
Сделаем замену $Y_t = \mathrm{e}^{t- 2 W_t} X_t^2$. Применяем лемму Ито:
\begin{equation}
  \mathrm{d} Y_t = \mathrm{e}^{t- 2 W_t} \mathrm{d} (X_t^2) + X_t^2 \mathrm{d} \mathrm{e}^{t- 2 W_t} + \mathrm{d} \mathrm{e}^{t- 2 W_t} \cdot \mathrm{d} (X_t^2)
\end{equation}

Заметим, что $\mathrm{d} \mathrm{e}^{t- 2 W_t} = \mathrm{e}^{t- 2 W_t} \left( 3 \mathrm{d} t - 2 \mathrm{d} W_t \right)$ и $\mathrm{d}(X_t^2) = (X_t^2 + 2) \mathrm{d} t + 2 X_t^2 \mathrm{d} W_t$.

Получаем
\begin{equation}
  \mathrm{d} Y_t = 2 \mathrm{e}^{t- 2 W_t} \mathrm{d} t
\end{equation}
Вспоминаем начальное условие $Y_0 = \mathrm{e}^{0 - 2 W_0}X_0^2 = x^2$. Получаем:
\begin{equation}
   \mathrm{e}^{t- 2 W_t} X_t^2 = Y_t =  x^2 + \int_0^t 2 \mathrm{e}^{s - 2 W_s} \mathrm{d} s
\end{equation}
Решение
\begin{equation}
  X_t = \operatorname{sign}(x) \sqrt{ x^2 \mathrm{e}^{2 W_t - t} + 2 \mathrm{e}^{2 W_t - t} \int_0^t \mathrm{e}^{s - 2 W_s} \mathrm{d} s  }
\end{equation}
Источник: \url{http://math.stackexchange.com/questions/80118/}

\end{sol}
\end{problem}

\begin{problem}
Consider the following stochastic differential equation
\begin{equation}
dX_t=(\sqrt{1-X_t^2}-0.5X_t)dt+\sqrt{1-X_t^2}dW_t, \quad X_0=0
\end{equation}
\begin{enumerate}
\item Consider a substitution $X_t=\sin Y_t$, where $Y_t$ is some Ito process $dY_t=A_tdt+B_tdW_t$. Using Ito's lemma find $dX_t$.
\item State conditions on $A_t$ and $B_t$ such that the two $dX_t$ coincides
\item Find $Y_t$ and $X_t$
\end{enumerate}

\begin{sol}
$X_t=\sin(t+W_t)$
\end{sol}
\end{problem}


\begin{problem}
Предположим, что $S_t$ является геометрическим броуновским движением, то есть
\[
dS_t = \mu S_t \, dt + \sigma S_t \, dW_t
\]


\begin{enumerate}
\item Найдите $\E(S_t)$ и $\Var(S_t)$
\item Являются ли геометрическими броуновскими движениями процессы $X_t=S_t^{-1}$ и $X_t=S_t^{2}$? Если да, то с какими коэффициентами сноса и волатильности?
\end{enumerate}


\begin{sol}

\end{sol}
\end{problem}


\section{Модель Блэка-Шоулса}


\begin{problem}
В рамках модели Блэка-Шоулса определите, сколько стоят в момент времени $t=0$ следующие активы:
\begin{enumerate}
\item Актив, выплачивающий в момент времени $T$ величину $\frac{1}{S(T)}$.
\item Актив, выплачивающий в момент времени $T$ величину $\max\{ln(S(T),0\}$.
\end{enumerate}

\begin{sol}

\end{sol}
\end{problem}

\begin{problem}
В рамках стандартной модели Блэка-Шоулза рассмотрим актив, который выплачивает величину $S^{2}(T_{1})/S^{2}(T_{0})$ в момент времени $T_{1}$. \\
Моменты времени $T_{0}<T_{1}$ фиксированы. \\
Найдите стоимость этого актива в момент времени $t=0$

\begin{sol}

\end{sol}
\end{problem}

\begin{problem}
Цена акции в долларах подчиняется уравнению $dS=\alpha Sdt+\sigma SdW_{1}$ \\
Валютный курс (евро за доллар) подчиняется уравнению $dY=\beta Ydt+\delta YdW_{2}$ \\
$W_{1}$ и $W_{2}$ — независимые броуновские движения \\
Опцион платит $\ln(SY)$ (евро) в момент времени $T$ \\
$r$ — процентная ставка в евро. \\
Найдите цену этого опциона в евро в момент времени $0$ (выразите ее через цену акции в евро).

\begin{sol}

\end{sol}
\end{problem}

\begin{problem}
 уравнение Блэка-Шоулза для акций с дивидендами \\
Допустим по акции платится постоянный дивиденд $\delta$. Т.е. за промежуток времени $dt$ владелец акции получает $\delta Sdt$. Рассмотрим портфель из $(-1)$ опциона колл, $a_{t}$ акций и $b_{t}$ наличности.
\begin{enumerate}
\item Подберите $a_{t}$ так, чтобы портфель был безрисковым.
\item Подберите $b_{t}$ так, чтобы выполнялось бюджетное ограничение.
\item Выведите уравнение Блэка-Шоулза.
Подсказка 1: При отсутствии арбитража безрисковый портфель должен приносить такую же доходность, как\ldots \\
Подсказка 2: Готовый ответ отличается всего одним слагаемым.
\end{enumerate}

\begin{sol}

\end{sol}
\end{problem}

\begin{problem}
В рамках модели Блэка-Шоулза известны безрисковая процентная ставка, $r=0.05$, волатильность, $\sigma=0.2$ и стартовая цена акций, $S_0=100$. С помощью численного эксперимента на компьютере найдите стоимость опциона, который в момент времени $T=1$ выплачивает $\max\{ S_{0.2},S_{0.8}\}$.

\begin{sol}
\end{sol}
\end{problem}


\section{Модель Блэка-Шоулса-2}


\section{Компьютерное моделирование}

\begin{problem}
  Flip a coin $25$ times, recording whether it comes up Heads  or Tails each time.  Scoring $X_i = +1$ for each Heads and $X_i = -1$ for each flip, also keep track of the accumulated sum $S_n = \sum_{i=1}^n X_i$ for $i = 1 \dots 25$ representing the net fortune at any time.  Plot the resulting $S_n$ versus $n$ on the interval   $[0,25]$.  Finally, using $N=5$ plot the rescaled approximation  $W_5(t) = (1/\sqrt{5}) S(5t)$ on the interval $[0,5]$ on the same graph.

\begin{sol}
\end{sol}
\end{problem}



%\section{Asian, Lookback, Knock-out}


\Closesolutionfile{solution_file}
