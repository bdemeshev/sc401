\documentclass[a4paper]{article}

\usepackage{fontspec}
\usepackage{polyglossia}

\setmainlanguage{russian}
\setotherlanguages{english}

% download "Linux Libertine" fonts:
% http://www.linuxlibertine.org/index.php?id=91&L=1
\setmainfont{Linux Libertine O} % or Helvetica, Arial, Cambria
% why do we need \newfontfamily:
% http://tex.stackexchange.com/questions/91507/
\newfontfamily{\cyrillicfonttt}{Linux Libertine O}

\usepackage{amsmath} % Математические окружения AMS
\usepackage{amsfonts} % Шрифты AMS
\usepackage{amssymb} % Символы AMS
\usepackage{mathtext} % Русские буквы в фомулах
\usepackage{graphicx} % Вставить pdf- или png-файлы

\usepackage{xcolor}

\newcommand{\highlight}[1]{%
  \colorbox{red!50}{$\displaystyle#1$}}

\usepackage{color}
\usepackage{bbold}

\usepackage{booktabs}

\usepackage{mathrsfs} % Красивый шрифт

\usepackage{longtable}  % Длинные таблицы
\usepackage{multirow} % Слияние строкв таблице

\usepackage{indentfirst} % Отступ в первом абзаце.
\usepackage{tikz}

\newcommand*{\hm}[1]{#1\nobreak\discretionary{}%
            {\hbox{$\mathsurround=0pt #1$}}{}}

\usepackage{verbatim}

\DeclareMathOperator{\sgn}{\mathop{sgn}}
\DeclareMathOperator{\card}{\mathop{card}}
%\newcommand{\e}[1]{{\mathbb E}\left[ #1 \right]}
\DeclareMathOperator{\E}{E}

\let\P\relax
\DeclareMathOperator{\P}{P}
\newcommand{\cN}{\mathcal{N}}

\usepackage{enumitem}
\usepackage{mathtools, bm, etoolbox}

\usepackage{fancyhdr}
\usepackage[margin=1in]{geometry}

\providecommand\given{}
\DeclarePairedDelimiterXPP\Aver[1]{\mathbb{E}}{[}{]}{}{
\renewcommand\given{  \nonscript\:
  \delimsize\vert
  \nonscript\:
  \mathopen{}
  \allowbreak}
#1
}




\newcommand{\shrug}[1][]{%
\begin{tikzpicture}[baseline,x=0.8\ht\strutbox,y=0.8\ht\strutbox,line width=0.125ex,#1]
\def\arm{(-2.5,0.95) to (-2,0.95) (-1.9,1) to (-1.5,0) (-1.35,0) to (-0.8,0)};
\draw \arm;
\draw[xscale=-1] \arm;
\def\headpart{(0.6,0) arc[start angle=-40, end angle=40,x radius=0.6,y radius=0.8]};
\draw \headpart;
\draw[xscale=-1] \headpart;
\def\eye{(-0.075,0.15) .. controls (0.02,0) .. (0.075,-0.15)};
\draw[shift={(-0.3,0.8)}] \eye;
\draw[shift={(0,0.85)}] \eye;
% draw mouth
\draw (-0.1,0.2) to [out=15,in=-100] (0.4,0.95);
\end{tikzpicture}}


\AddEnumerateCounter{\asbuk}{\russian@alph}{щ} % для списков с русскими буквами
\setlist[enumerate, 2]{label=\asbuk*),ref=\asbuk*}


\pagestyle{fancy} \makeatletter \fancyhead[L]{\footnotesize ICEF, 2016/17, «Mathematics for Economists»}

\makeatletter
\newcommand*{\rom}[1]{\expandafter\@slowromancap\romannumeral #1@}
\makeatother

\usepackage{wrapfig}

 \begin{document}
 \begin{center}
 {\Large{Конспект лекции 20.12.16}}
 \end{center}
  \begin{center}
 {\large{Аня Терещенко}}
 \end{center}

\section*{Упражнения по стохастическому анализу}

\section*{Упражнение 1}

\parindent=1cm

Рассмотрим модель Блэка-Шоулза

$\mu=0.2$

$r=0.1$

$\delta=0.1$

Найти вероятности:

\begin{enumerate}

\item $\P(W_2>0)$

\item $\widetilde{\P}(W_2>0)$

\item $\P(\widetilde{W_2}>0)$

\item $\widetilde{\P}(\widetilde{W_2}>0)$
\end{enumerate}

Сразу можно сказать, что в пункте 1) и 4) ответ равен $0.5$,

то есть $\P(W_2>0)=0,5$ и  $\widetilde{\P}(\widetilde{W_2}>0)=0.5$.

$W_2$ распределен нормально с математическим ожиданием 0 и дисперсией 2.

$X_2 \sim N(0;2)$

То есть справа от нуля лежит половина распределения.

А значит, вероятность попадания в правую часть равна 0,5.

$\widetilde{W_t}=W_t+\frac{\mu -r}{\delta}$

2) $\widetilde{\P}(W_2>0)=\widetilde{\P}(\widetilde{W_2}-\frac{0.2-0.1}{0.1}\cdot 2 >0)=\widetilde{\P}(\widetilde{W_2}>2)=\widetilde{\P}(N(0;1)>\frac{2}{\sqrt{2}})=1-F(1.4142135)$

Относительно p процесс $W_t$- броуновское движение.

Относительно  $\widetilde{\P}$ процесс $\widetilde{W_t}$- броуновское движение.

4) $\widetilde{\P}(\widetilde{W_2}>0)=\P(W_2+\frac{0.2-0.1}{0.1}\cdot 2>0)=\P(W_2>-2)=\P(N(0;1)>-\sqrt{2})=F(1.41)$,

где $F$ - функция распределения для $\cN(0;1)$.

\section*{Пусть $S_0=100$}

\parindent=1cm

Найти:

\begin{enumerate}

\item $\P(S_1>110)$

\item $\widetilde{\P}(S_1>110)$

\item $\E_{\P}(S_1)$

\item $\E_{\widetilde{\P}}(S_1)$

\end{enumerate}

$S_t=S_0e^{(\mu-\frac{\delta^2}{2})\cdot t+\delta W_t}$

$\P(S_1>110)=\P(S_0e^{(0.2-\frac{0.01}{0.2})\cdot 1+0.1 W_1}>110)=\P(100e^{0.195+0.1\cdot W_1}>110)=\P(e^{0.195+0.1 \cdot W_1)}>1.1)=\P(e^{0.195+0.1 \cdot W_1}>e^{\ln 1.1})=\P(0.195+0.1 \cdot W_t> \ln 1.1)=\P(W_1>\frac{\ln 1.1 -0.195}{0.1})=1-F(\frac{\ln 1.1-0.195}{0.1})=0.84$

$W_1 \sim N(0;1)$

$W_1=\widetilde{W_1}-\frac{0.2-0.1}{0.1}$

$\E_{\P}(S_1)=\E(S_0e^{(\mu-\frac{\delta^2}{2})\cdot t+\delta W_t})=\E(100e^{(0.2-\frac{0.01}{0.2})\cdot 1+0.1 W_t})=100e^{0.195}\E_{\P}(e^{0.1 W_1})$

Если $W_1 \sim \cN(\mu;\delta^2)$,

то $\E(e^{\alpha X})=e^{\alpha \mu} +\frac{\alpha^2 \delta^2}{2}$

Так как $W_1 \sim \cN(0;1)$,

то $100e^{0.195}\E_{\P}(e^{0.1 W_1})=100e^{0.195}e^{0.1 \cdot 0+\frac{0.1^2 \cdot 1}{2}}=100e^{0.2}$

$\E_{\widetilde{\P}}(S_1)$

Свойство $\widetilde{\P}$

$X_0=\E_{\widetilde{\P}}(e^{-r t} X_t|\mathcal{F}_0)$

Цена акции $S_1$ определяется броуновским движением и от прошлого не зависит (не зависит от $\mathcal{F}_0$)

$S_0=\E_{\widetilde{\P}}(e^{-r \cdot 1} S_1|\mathcal{F}_0)=e^{-0.1}E_{\P}(S_1)$

$\E_{\widetilde{\P}}(S_1)=100e^{0.1}$

\section*{Упражнение 2}

В рамках модели Блэка-Шоулза оцените в T=0 актив, который в момент времени Т стоит $X_T=\ln S$

$\widetilde{W_t}=W_t+\frac{\mu -r}{\delta}$

$X_0=\E_{\widetilde{\P}}(e^{-rT}\cdot X_T|\mathcal{F}_0)=e^{-rT}\cdot \E_{\widetilde{\P}}(\ln S_T|\mathcal{F}_0)=e^{-rt}\cdot \E_{\widetilde{\P}}(\ln (S_0 \cdot e^{(\mu -\frac{\delta^2}{2})\cdot T+\delta \cdot W_T}|\mathcal{F}_0)=e^{-rT}(\ln S_0+(\mu -\frac{\delta^2}{2})\cdot T + \E_{\widetilde{\P}}(\delta W_t|\mathcal{F}_0))$

$\E_{\widetilde{\P}}(\delta W_t|\mathcal{F}_0)=\E_{\widetilde{\P}}(\delta \cdot (\widetilde{W}-\frac{\mu - r}{\delta}\cdot T)=\delta(\E_{\widetilde{\P}}(\widetilde{W_T})-\frac{\mu-r}{\delta}\cdot T)=-\delta \cdot (\mu-r)\cdot T$,

$\widetilde{W_T} \sim \cN(0;T-s)$

следовательно $\E_{\widetilde{\P}}(\widetilde{W_T})=0$

$X_0=e^{-rT}(\ln S_0 +(\mu -\frac{\delta^2}{2})\cdot T-T(\mu-r)$

$X_0=e^{-rT}(\ln S_0-T \cdot \frac{\delta^2}{2}+rT)$


\end{document}
