\documentclass[a4paper]{article}

\usepackage{fontspec}
\usepackage{polyglossia}

% download "Linux Libertine" fonts:
% http://www.linuxlibertine.org/index.php?id=91&L=1
\setmainfont{Linux Libertine O} % or Helvetica, Arial, Cambria
% why do we need \newfontfamily:
% http://tex.stackexchange.com/questions/91507/
\newfontfamily{\cyrillicfonttt}{Linux Libertine O}





\usepackage{amsmath} % Математические окружения AMS
\usepackage{amsfonts} % Шрифты AMS
\usepackage{amssymb} % Символы AMS
\usepackage{mathtext} % Русские буквы в фомулах
\usepackage{graphicx} % Вставить pdf- или png-файлы

\usepackage{color}
\usepackage{bbold}

\usepackage{booktabs}

\usepackage{mathrsfs} % Красивый шрифт

\usepackage{longtable}  % Длинные таблицы
\usepackage{multirow} % Слияние строкв таблице

\usepackage{indentfirst} % Отступ в первом абзаце.
\usepackage{tikz}

\newcommand*{\hm}[1]{#1\nobreak\discretionary{}%
            {\hbox{$\mathsurround=0pt #1$}}{}}

\usepackage{verbatim}

\DeclareMathOperator{\sgn}{\mathop{sgn}}
\DeclareMathOperator{\card}{\mathop{card}}

\usepackage{enumitem}


\usepackage{fancyhdr}
\usepackage[margin=1in]{geometry}





\pagestyle{fancy} \makeatletter \fancyhead[L]{\footnotesize ICEF, 2016/17, «Mathematics for Economists»}

\makeatletter
\newcommand*{\rom}[1]{\expandafter\@slowromancap\romannumeral #1@}
\makeatother

 \begin{document}
 \begin{center}
 {\Large{Конспект лекции 22.11.16}}
 \end{center}
  \begin{center}
 {\large{Марданов Тимур}}
 \end{center}
 \par {\bf\underline{Упражнение 8.}} Пошла Маша в лес и собрала 100 грибов: $L$ лисичек, $R$ рыжиков и $M$ мухоморов. Пусть $p_{L}, p_{R}, p_{M}$ - вероятности найти гриб определенного типа, такие что $\begin{cases} p_{L}, p_{R}, p_{M} > 0\\p_{L} + p_{R} +p_{M} = 1\end{cases}$.

 Пусть также расположение каждого гриба не зависит от расположения других.

 Найдите:
\begin{enumerate}
	\item $E(R+L \mid M)$, $E(M \mid R+L)$;
	\item $E(R \mid L)$;
	\item $Var(R \mid L)$;
	\item $E(R+L \mid L+M)$;
	\item $P(E(R \mid L))=0$;
	\item $P(R=0 \mid L)$;
	\item $E((\frac{p_{M}}{p_{R}+p_{L}})^{100-L})$;
	\item $P(R=7\mid L)$.
\end{enumerate}
\par{\underline{Решение:}}
\begin{enumerate}
	\item Если известно, что Маша собрала $M$ мухоморов, то очевидно, что она собрала $100 - M$ лисичек и рыжиков. $E(R+L \mid M) = 100 - M$.
	Аналогично $E(M \mid R+L) = 100 - R - L$.
	\item Если Маша собрала $L$ лисичек, то тогда $R\in \left \{0, 1\ldots, 100 - L  \right \}$. Заметим, что случайная величина $R$ имеет биномиальное распределение $R \sim Bin(n=100, p =p_{R}) \rightarrow \\ \rightarrow (R\mid L) \sim Bin(n =100-L,p =\frac{p_{R}}{p_{R}+p_{M}})$. \\ Матожидание биномиального распределения $E(R) = np$. Тогда $E(R \mid L) = (100-L)(\frac{p_{R}}{p_{M}+p_{R}})$.
	\item Дисперсия биномиального распределения $Var(R) = np(1-p)$. \\ Тогда $Var(R\mid L)=(100-L)(\frac{p_{R}}{p_{M}+p_{R}})(1-\frac{p_{R}}{p_{M}+p_{R}})$.
	\item $E(R+L\mid L+M) = E(R\mid L+M) + E(L\mid L +M) = (100-(L+M)) + ((L+M)\frac{p_{L}}{p_{L}+p_{M}})$. Здесь главное заметить, что $L\mid L +M$ имеет биномиальное распределние.
	\item Необходимо просто подставить ответ из пункта 2) \\ $P(E(R\mid L)=0) = P((100-L)(\frac{p_{R}}{p_{M}+p_{R}})=0) = P(L=100) =p_{L}^{100}$.
	\item Иногда для нахождения общего решения удобно подставлять конкретные значения вместо случайных величин и находить частное решение. Попробуем найти не $P(R=0\mid L)$,  а \\ $P(R=0\mid L=7) = P(M=93\mid L=7) = (\frac{p_{M}}{p_{M}+p_{R}})^{93}$. Тогда $P(R=0\mid L) = (\frac{p_{M}}{p_{M}+p_{R}})^{100 - L}$.
	\item $E((\frac{p_{M}}{p_{M}+p_{R}})^{100 - L})=E(P(R=0\mid L))=\left \{ E(E(X\mid \mathcal{F}))=E(X) \right \} = P(R=0)=(1-p_{R})^{100}$.
	\item Вспомнив формулу для биномиального распределения $P(R=x)=C_{n}^{x}p^{x}(1-p)^{n-x}$ и результаты пункта 2), получим $P(R=7\mid L)=\begin{cases}
	0 & \text{ if } L= 93\\
	C_{100-L}^{7}(\frac{p_{R}}{p_{R}+p_{M}})^{7}(1-\frac{p_{R}}{p_{R}+p_{M}})^{100-L-7} & \text{ if } L\neq 93 \end{cases}$.
\end{enumerate}
\par {\bf\underline{Упражнение 9 (8.58 из задачника).}}
\\
\par{$x_{1},\ldots, x_{100} \sim \mathcal{U}[0;1]$ и независимые.

$L = max\left \{  x_{1},\ldots, x_{80}\right \}$;
$R = max\left \{  x_{81},\ldots, x_{100}\right \}$;
$M = max\left \{L,R\right \}$.}

Найдите:
\begin{enumerate}
\item $P(L>R\mid L)$;

\item $E(x_{1}\mid L)$;

\item $E(min\left \{  x_{1},\ldots, x_{100}\right \}\mid M)$;

\item $E(min\left \{  x_{1},\ldots, x_{100}\right \}\mid x_{1})$.
\end{enumerate}
\par{\underline{Решение:}}
\begin{enumerate}
\item Предположим $L=0.7$. Тогда $P(L>R\mid L=0.7) = P(0.7>R\mid L=0.7) = \left \{R\  \text{и} \ L - \text{независимы}  \right \} = P(R<0.7)= P(x_{81}<0.7,\ldots, x_{100}<0.7)=$\\$
=\left \{  x_{i} - \text{независимы}\right \} = P(x_{81}<0.7) * \ldots  * P(x_{100}<0.7) = 0.7^{20} \rightarrow P(L>R\mid L) = L^{20}$ \\
$(*) \ E(L^{20})=E(P(L>P\mid L))=\left \{ E(E(X\mid \mathcal{F}))=E(X) \right \}=P(L>R)=\frac{80}{100}$.

\item Попробуем сначала найти $E(x_{1}\mid max\left \{x_{1},x_{2},x_{3}  \right \} = 0.7)= p_{x_{1}=max}0.7+p_{x_{1}\neq max}E(x_{1}) =\frac{1}{3}0.7+ \\ +\frac{2}{3}0.35$.
\\Тогда $E(x_{1}\mid L) =\frac{1}{80}L+\frac{79}{80}\frac{L}{2}$.

\item Попробуем сначала найти $E(min\left \{  x_{1},\ldots, x_{100}\right \}\mid M = 0.7) =  \left \{  y \sim \mathcal{U}[0;0.7] \right \}=E(min\left \{y_{1},\ldots, y_{99}\right \})$. Проведём мысленный эксперимент: возьмём отрезок [0,0.7] и отметим на нём 99 точек случайным образом. Так мы поделим отрезок на 100 маленьких отрезков. И хотя их длинна различна, в среднем она равна $\frac{0.7}{100}$. А величина, которую мы ищем, равна рсстоянию от нуля до первой точки = средняя длинна маленького отрезка. Тогда $E(min\left \{  x_{1},\ldots, x_{100}\right \}\mid M = 0.7) = \frac{0.7}{100}$, а в общем случае $E(min\left \{  x_{1},\ldots, x_{100}\right \}\mid M) = \frac{M}{100}$.
\item $E(min\left\{x_{1},\ldots, x_{100} \right\} \mid x_{1})= p_{x_{1}=min}x_{1} + p_{x_{1}\neq min}E(min \left \{x_{2},\ldots, x_{100}  \right \}) = \frac{1}{100}x_{1} + \frac{99}{100}\frac{1}{100}$.
\end{enumerate}
  \begin{center}
	{\large{Мартингалы}}
\end{center}
\par{\bf\underline{Определение.}} Фильтрация $(\mathcal{F}_{n})$ — последовательность $\sigma$-алгебр таких, что $\mathcal{F}_{1}\subseteq \mathcal{F}_{2} \subseteq \ldots  \subseteq \mathcal{F}_{n}$. Иногда фильтрацию называют потоком $\sigma$-алгебр.
\\
\par{\bf\underline{Определение.}} Случайный процесс ($X_{n}$) — последовательность случайных величин $X_{n}$.
\\
\par{\bf\underline{Определение.}} Случайный процесс ($X_{n}$) адаптирован к фильтрации $(\mathcal{F}_{n})$, если $\forall n$ величина $X_{n}$ является $\mathcal{F}_{n}$-измеримой.
\\
\par{\bf\underline{Определение.}} Случайный процесс ($X_{n}$) — мартингал по отношению к фильтрации $(\mathcal{F}_{n})$, если
\begin{enumerate}
	\item $E(X_{n})$ - существует $\forall n$;
	\item ($X_{n}$) адаптирован к фильтрации $(\mathcal{F}_{n})$;
	\item $\forall n \ E(X_{n+1} \mid \mathcal{F}_{n})=X_{n}$ (лучший прогноз на завтра - сегодняшнее значение).
\end{enumerate}
\par\underline{Пример 1.} Имеется колода из 52 карт. Достаем по очереди одну карту. $X_{n}\equiv$ доля тузов в неоткрытой части колоды после открытия $n$ карт. Тогда:
\\
\[\begin{tabular}{cc}
$X_{0}$&$\frac{4}{52}$\\
\midrule
P&$1$\\
\end{tabular}
\quad \quad
\begin{tabular}{ccc}
$X_{1}$&$\frac{3}{51}$&$\frac{4}{51}$\\
\midrule
P&$\frac{4}{52}$&$\frac{48}{52}$\\
\end{tabular}
\quad \quad \begin{tabular}{ccc}
$X_{51}$&$1$&$0$\\
\midrule
P&$\frac{48}{52}$&$\frac{4}{52}$\\
\end{tabular}\]
Является ли $X_{n}$ мартингалом? Первые два условия выполняются. Проверим третье:

\[\begin{tabular}{c|c|c}
&сейчас&следующий момент\\
\hline
извлечено карт&$n$&$n+1$\\
\hline
осталось карт&$52-n$&$51-n$\\
\hline
доля тузов в колоде&$X_{n}$&$X_{n+1}$\\
\hline
штук тузов в колоде&$X_{n}(52-n)$&$X_{n+1}(51-n)$\\
\hline
тузов открыто&$4-X_{n}(52-n)$&$4-X_{n+1}(51-n)$\\
\end{tabular}\]
$ E(X_{n+1}\mid \mathcal{F}_{n} )  =\left\lbrace X_{n} - \text{условная пвероятность извлечь туза} \right\rbrace = X_{n}(\frac{X_{n}(52-n)-1}{51-n}) + (1 -X_{n})(\frac{X_{n}(52-n)}{51-n}) = \\ = \frac{X_{n}(52-n)(X_{n}+1-X_{n})-X_{n}}{51-n} = X_{n}\rightarrow X_{n}$ - мартингал.
\clearpage
\par\underline{Пример 2.} $ \mathcal{F}_{n} = \sigma(z_{1},\ldots, z_{n}); z_{i} -$ случайные величины, независимые и одинаково распределённые. $P(z_{i}=-1)=P(z_{i}=1)=\frac{1}{2}$. Являются ли мартингалами следующие случайные процессы?
\begin{enumerate}
	\item $(z_{n})$;
	\item $(X_{n}), \ X_{n}=\sum_{i = 1}^{n}z_{i}$;
	\item $(R_{n}), \ R_{n}=X_{n}^{2}$;
	\item $(L_{n}), \ L_{n}=X_{n}^{2}-n$.
\end{enumerate}
\par{{Решение:}} все процессы адаптированы к $(\mathcal{F}_{n})$
\begin{enumerate}
	\item $E(z_{n+1}\mid \mathcal{F}_{n})=E(z_{n+1}\mid z_{1},\ldots, z_{n})=\left\lbrace z_{i} - \text{независимы}\right\rbrace = E(z_{n+1}) = 0 \neq z_n=1 \ or -1 \rightarrow \\ \rightarrow(z_{n}) -$ не мартингал.
	\item $E(X_{n+1}\mid \mathcal{F}_{n})=E(\sum_{i=1}^{n+1}z_{i}\mid z_{1},\ldots, z_{n})=\sum_{i=1}^{n}z_{i}+E(z_{n+1}\mid z_{1},\ldots, z_{n}) = \sum_{i=1}^{n}z_{i} = X_{n} \rightarrow \\ \rightarrow(X_{n}) - $ мартингал.
	\item $E(R_{n+1}\mid \mathcal{F}_{n})=E(X_{n+1}^{2}\mid z_{1},\ldots, z_{n})=\left\lbrace X_{n+1}=X_{n}+z_{n+1}\right\rbrace = E((X_{n}+z_{n+1})^{2}\mid z_{1},\ldots, z_{n})= \\ = E(X_{n}^{2}\mid z_{1},\ldots, z_{n}) + E(2X_{n}z_{n+1}\mid z_{1},\ldots, z_{n})+E(z_{n+1}^{2}\mid z_{1},\ldots, z_{n})=X_{n}^2+0+1 \neq R_{n}\rightarrow \\ \rightarrow(R_{n}) - $ не мартингал.
	\item $E(L_{n+1}\mid \mathcal{F}_{n})=E(X_{n+1}^{2}-(n+1)\mid z_{1},\ldots, z_{n})=E(X_{n+1}^{2}\mid z_{1},\ldots, z_{n})-E((n+1)\mid z_{1},\ldots, z_{n})= \\ =E(R_{n+1}\mid z_{1},\ldots, z_{n})-n-1=X_{n}^2+1-n-1=X_{n}^2-n =L_{n} \rightarrow (L_{n}) - $ мартингал.
\end{enumerate}
\begin{center}{\large Свойства мартингалов}\end{center}
Если $(X_{n})$ — мартингал, то:
\begin{enumerate}
	\item $E(X_{n})=const$ \\ Доказательство: $E(X_{n+1}\mid \mathcal{F}_{n})=X_{n} \rightarrow E(E(X_{n+1}\mid \mathcal{F}_{n}))=E(X_{n}) \rightarrow E(X_{n+1})=E(X_{n}) \ $ для $\forall n$.
	\item $E(X_{n+k}\mid \mathcal{F}_{n})=X_{n}$ для $\forall k\geq1$ \\ Доказательство:
	\begin{itemize}
		\item Случай $k = 1$: \quad $E(X_{n+1}\mid \mathcal{F}_{n})=X_{n}$
		\item Предположим $E(X_{n+i}\mid \mathcal{F}_{n})=X_{n}$: \quad $E(X_{n+i+1}\mid \mathcal{F}_{n})=\left\lbrace \mathcal{F}_{n} \subseteq \mathcal{F}_{n+i}  \right\rbrace = \\ =E(E(X_{n+i+1}\mid \mathcal{F}_{n+i})\mid \mathcal{F}_{n})=\left\lbrace   E(X_{n+i+1}\mid \mathcal{F}_{n+i}) = X_{n+i} \right\rbrace =E(X_{n+i}\mid \mathcal{F}_{n})=X_{n}$.
	\end{itemize}
\end{enumerate}
\par{\bf\underline{Определение.}} Случайная величина $\tau$ называется моментом остановки (stopping time) по отношению к фильтрации $(\mathcal{F}_{n})$, если:
\begin{enumerate}
	\item $\tau \in \left\lbrace 0,1,2,3,\ldots \right\rbrace \cup \left\lbrace +\infty \right\rbrace $.
	\item $\forall n$ индивид, различающий события из $\mathcal{F}_{n}$, способен понять: наступил момент $\tau$ или нет. Формально: событие $\left\lbrace \tau \leq n \right\rbrace$ лежит в $\sigma$-алгебре $\mathcal{F}_{n}$.
\end{enumerate}
\par{Пример:} внук играет с кошкой во дворе. Предположим, что  у внука есть часы и он знает, убежала ли кошка и во сколько. Бабушка может установить следующие правила, когда нужно идти домой:
\begin{enumerate}
	\item $\tau_{1}=$ вернуться через час после того, как кошка убежит.
	\item $\tau_{2}=$ вернуться за час до того, как кошка убежит.
\end{enumerate}
Величина $\tau_{1}$ является моментом остановки, так как внук всегда может сказать, когда убежала кошка и сколько времени прошло. \\
Величина $\tau_{2}$ не является моментом остановки, так как внук не может сказать, когда убежит кошка, не заглядывая в будущее. \\
\par{\bf\underline{Теорема об остановке мартингала}} (теорема Дуба или stopping time theorem). Если $(X_{n}$ — мартингал по отношению к $(\mathcal{F}_{n})$; $\tau$ - момент остановки по отношению к $(\mathcal{F}_{n})$ и выполнено хотя бы одно из условий 1-3, то $E(X_{\tau})=E(X_{1})$
\par{Условия:}
\begin{enumerate}
	\item Существует число $m$ такое, что $\tau < m$.
	\item $P(\tau = +\infty) =0$ и при это существует такое число $m$, что $|X_{min\left\lbrace n,\tau \right\rbrace }|<m$
	\item $E(\tau)<\infty$ и существует такое число $m$, что $E(|X_{min\left\lbrace n+1,\tau \right\rbrace } - X_{min\left\lbrace n,\tau \right\rbrace }|\mathcal{F}_{n})<m$
\end{enumerate}
\end{document}
