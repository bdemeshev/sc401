\documentclass[pdftex,12pt,a4paper]{article}

\input{title_bor}

\begin{document}

\begin{enumerate}
\item Игральный кубик подбрасывают два раза. Пусть $X$ --- это остаток от деления на 3 результата первого броска, а $Y$ --- остаток от деления на 3 суммарного результата двух бросков. Обозначим $\hat{Y}=\E( Y \mid X)$. 
\begin{enumerate}
\item Какие значения принимает $\hat{Y}$ и с какими вероятностями? 
\item Найдите $\P(\hat{Y}=X)$.
\end{enumerate}

\item Случайный процесс $Z_t$ задан выражением $Z_t=\exp(-1+3W_t+bt)$, где $b$ --- это константа. 
\begin{enumerate}
\item Найдите $dZ_t$
\item Выпишите формулу для $dZ_t$ в полной записи (с интегралами)
\item При каком $b$ процесс $Z_t$ будет мартингалом?
\end{enumerate}

\item Пусть $X_t=e^{t/2}\sin W_t$,  а $Y=e^{-t/2}\cos W_t$. 
\begin{enumerate}
\item Найдите $dX_t$, $dY_t$
\item Являются ли процессы $X_t$ и $Y_t$ мартингалами?
\item Найдите $\E(X_t)$ и $\E(Y_t)$
\end{enumerate}

\item Саша и Маша играют в орлянку. Саша немного жульничает, поэтому вероятность его победы в каждой отдельной партии равна $0.6$. Ничья невозможна, они решили играть до преимущества одного из игроков в 10 побед, не обязательно подряд. Какова вероятность того, что победителем серии партии окажется Саша?


\item  В рамках модели Блэка-Шоулса найдите текущую цену актива, который в момент времени $T=2$ выплачивает Вам  сумму равную $S_2/S_1$.




\end{enumerate}



\end{document}