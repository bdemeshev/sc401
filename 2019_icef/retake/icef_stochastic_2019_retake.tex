\documentclass[12pt, a4paper]{article}

\usepackage[top=1.5cm, left=2cm, right=2cm, bottom=1.5cm]{geometry} % размер текста на странице

\usepackage{tikz} % картинки в tikz
\usepackage{microtype} % свешивание пунктуации

\usepackage{verbatim} % для комментов

\usepackage{array} % для столбцов фиксированной ширины

\usepackage{indentfirst} % отступ в первом параграфе

\usepackage{sectsty} % для центрирования названий частей
\allsectionsfont{\centering}

\usepackage{amsmath} % куча стандартных математических плюшек
\usepackage{amssymb} % и символов
\usepackage{bbm}

\usepackage{multicol} % текст в несколько колонок

\usepackage{lastpage} % чтобы узнать номер последней страницы

\usepackage{enumitem} % дополнительные плюшки для списков
%  например \begin{enumerate}[resume] позволяет продолжить нумерацию в новом списке




\usepackage{fontspec} % хз
\usepackage{polyglossia} % для выбора языка в xelatex

\setmainlanguage{english}
\setotherlanguages{english}

% download "Linux Libertine" fonts:
% http://www.linuxlibertine.org/index.php?id=91&L=1
\setmainfont{Linux Libertine O} % or Helvetica, Arial, Cambria
% why do we need \newfontfamily:
% http://tex.stackexchange.com/questions/91507/
\newfontfamily{\cyrillicfonttt}{Linux Libertine O}

\AddEnumerateCounter{\asbuk}{\russian@alph}{щ} % для списков с русскими буквами
%\setlist[enumerate, 2]{label=\asbuk*),ref=\asbuk*} % списки уровня 2 будут буквами а) б) ...

\usepackage{todonotes} % для вставки в документ заметок о том, что осталось сделать
% \todo[inline]{Здесь надо коэффициенты исправить}
% \missingfigure{Здесь будет картина Последний день Помпеи}
% команда \listoftodos — печатает все поставленные \todo'шки

\usepackage{booktabs} % красивые таблицы
% заповеди из документации:
% 1. Не используйте вертикальные линии
% 2. Не используйте двойные линии
% 3. Единицы измерения помещайте в шапку таблицы
% 4. Не сокращайте .1 вместо 0.1
% 5. Повторяющееся значение повторяйте, а не говорите "то же"


% \usepackage[left=1cm,right=1cm,top=1cm,bottom=1cm]{geometry}

\usepackage{fancyhdr} % весёлые колонтитулы
\pagestyle{fancy}
\lhead{Mathematics, exam}
\chead{}
\rhead{2020-01-20}
\lfoot{}
\cfoot{}
\rfoot{\thepage/\pageref{LastPage}}
\renewcommand{\headrulewidth}{0.4pt}
\renewcommand{\footrulewidth}{0.4pt}

\DeclareMathOperator{\E}{\mathbb{E}}
\let\P\relax
\DeclareMathOperator{\P}{\mathbb{P}}
\DeclareMathOperator{\Var}{\mathbb{V}ar}
\DeclareMathOperator{\Cov}{\mathbb{C}ov}



%% эконометрические сокращения
\def \hb{\hat{\beta}}
\DeclareMathOperator{\sVar}{sVar}
\DeclareMathOperator{\sCov}{sCov}
\DeclareMathOperator{\sCorr}{sCorr}

\def \1{\mathbbm{1}}

\def \hs{\hat{s}}
\def \hy{\hat{y}}
\def \hY{\hat{Y}}
\def \he{\hat{\varepsilon}}
\def \v1{\vec{1}}
\def \cN{\mathcal{N}}
\def \e{\varepsilon}
\def \z{z}

\def \hVar{\widehat{\Var}}
\def \hCorr{\widehat{\Corr}}
\def \hCov{\widehat{\Cov}}

\DeclareMathOperator{\tr}{tr}
\DeclareMathOperator*{\plim}{plim}

%% лаг
\renewcommand{\L}{\mathrm{L}}


\begin{document}


%%%%% !!! retake взят из 2018-2019 года, лишь пара цифр исправлена !!!!
\section*{Stochastic Calculus}

Standard Wiener process is denoted by $W_t$.

\begin{enumerate}
  \item Consider the following stochastic integral:
\[
I_t = \int_0^t 4W_u^2+3W_u + 6 \, dW_u
\]

\begin{enumerate}
\item {[3 points]} Find $d\E(I_t|I_s)$ for $t>s$.
\item {[7 points]} Find $\Var(I_t|I_s)$ for $t>s$.
\end{enumerate}


\item {[10 points]} Find $\E(W_2 | W_1, W_5)$ and $\E(W_2^3 | W_1, W_5)$.

  %Consider the process $R_t = W_t^4 - 6tW_t^2$.
  %\begin{enumerate}
%    \item Find $dR_t$. Is $R_t$ a martingale?
%    \item Find $f(t)$ such that $M_t = R_t + f(t)$ is a martingale.
%  \end{enumerate}

  \item {[10 points]} Suppose $X_t$ satisfies the stochastic differential equation
  \[
    dX_t = X_t \, dt + X_t^2 \, dW_t
  \]

  Determine constants $a$, $b$ and $c$ such that $Y_t = \exp(aX_t^b + ct)$ is a martingale.

  \item {[10 points]} Consider the framework of the Black and Scholes model.
The asset $X$ will pay you one share at fixed time $T$ if the price of a share
has increased by more than 10\% during the time period $[0;T]$.

What is the non-arbitrage price $X_0$ of this asset?
\item {[20 points]} Consider the Vasicek interest rate model, $dR_t=a(b-R_t) \, dt+s \, dW_t$, where $a$, $b$ and $s$ are positive constants.
\begin{enumerate}
\item Using the substitution $Y_t=e^{at} R_t$ find the solution of the stochastic differential equation;
\item Find $\E(R_t)$ and $\Var(R_t)$.
\end{enumerate}

\end{enumerate}




\end{document}
