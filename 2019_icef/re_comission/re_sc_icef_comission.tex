\documentclass[12pt, a4paper]{article}

\usepackage[top=1.5cm, left=2cm, right=2cm, bottom=1.5cm]{geometry} % размер текста на странице

\usepackage{tikz} % картинки в tikz
\usepackage{microtype} % свешивание пунктуации

\usepackage{verbatim} % для комментов

\usepackage{array} % для столбцов фиксированной ширины

\usepackage{indentfirst} % отступ в первом параграфе

\usepackage{sectsty} % для центрирования названий частей
\allsectionsfont{\centering}

\usepackage{amsmath} % куча стандартных математических плюшек
\usepackage{amssymb} % и символов
\usepackage{bbm}

\usepackage{multicol} % текст в несколько колонок

\usepackage{lastpage} % чтобы узнать номер последней страницы

\usepackage{enumitem} % дополнительные плюшки для списков
%  например \begin{enumerate}[resume] позволяет продолжить нумерацию в новом списке




\usepackage{fontspec} % хз
\usepackage{polyglossia} % для выбора языка в xelatex

\setmainlanguage{english}
\setotherlanguages{english}

% download "Linux Libertine" fonts:
% http://www.linuxlibertine.org/index.php?id=91&L=1
\setmainfont{Linux Libertine O} % or Helvetica, Arial, Cambria
% why do we need \newfontfamily:
% http://tex.stackexchange.com/questions/91507/
\newfontfamily{\cyrillicfonttt}{Linux Libertine O}

\AddEnumerateCounter{\asbuk}{\russian@alph}{щ} % для списков с русскими буквами
%\setlist[enumerate, 2]{label=\asbuk*),ref=\asbuk*} % списки уровня 2 будут буквами а) б) ...

\usepackage{todonotes} % для вставки в документ заметок о том, что осталось сделать
% \todo[inline]{Здесь надо коэффициенты исправить}
% \missingfigure{Здесь будет картина Последний день Помпеи}
% команда \listoftodos — печатает все поставленные \todo'шки

\usepackage{booktabs} % красивые таблицы
% заповеди из документации:
% 1. Не используйте вертикальные линии
% 2. Не используйте двойные линии
% 3. Единицы измерения помещайте в шапку таблицы
% 4. Не сокращайте .1 вместо 0.1
% 5. Повторяющееся значение повторяйте, а не говорите "то же"


% \usepackage[left=1cm,right=1cm,top=1cm,bottom=1cm]{geometry}

\usepackage{fancyhdr} % весёлые колонтитулы
\pagestyle{fancy}
\lhead{Mathematics, retake exam}
\chead{}
\rhead{2020-02-25}
\lfoot{}
\cfoot{}
\rfoot{\thepage/\pageref{LastPage}}
\renewcommand{\headrulewidth}{0.4pt}
\renewcommand{\footrulewidth}{0.4pt}

\DeclareMathOperator{\E}{\mathbb{E}}
\let\P\relax
\DeclareMathOperator{\P}{\mathbb{P}}
\DeclareMathOperator{\Var}{\mathbb{V}ar}
\DeclareMathOperator{\Cov}{\mathbb{C}ov}



%% эконометрические сокращения
\def \hb{\hat{\beta}}
\DeclareMathOperator{\sVar}{sVar}
\DeclareMathOperator{\sCov}{sCov}
\DeclareMathOperator{\sCorr}{sCorr}

\def \1{\mathbbm{1}}

\def \hs{\hat{s}}
\def \hy{\hat{y}}
\def \hY{\hat{Y}}
\def \he{\hat{\varepsilon}}
\def \v1{\vec{1}}
\def \cN{\mathcal{N}}
\def \e{\varepsilon}
\def \z{z}

\def \hVar{\widehat{\Var}}
\def \hCorr{\widehat{\Corr}}
\def \hCov{\widehat{\Cov}}

\DeclareMathOperator{\tr}{tr}
\DeclareMathOperator*{\plim}{plim}

%% лаг
\renewcommand{\L}{\mathrm{L}}


\begin{document}

\subsubsection*{Stochastic calculus part}

Here $W_t$ always denotes the standard Wiener process.

\vspace{10pt}

\begin{enumerate}


\item (10 points) Find $\E(W_6^2|W_4)$ and $\P(W_6>0|W_4)$.

Hint: the answer may contain the normal cumulative distribution function.

\item (10 points) I wait until Wiener process will hit the level $5$ or the level $-3$, 
that is up to the moment $\tau=\min\{t|W_{t}=5 \cup W_{t}=-3\}$.

Find $\P(W_{\tau} = 5)$ and $\E(\tau)$.

Hint: you may use the martingales $W_{t}$ and $W_{t}^2-t$.


\item (10 points) The process $X_t$ is defined as
\[
X_{t}=\begin{cases}
1, & t\in[0;1) \\
-2, & t\in[1;2) \\
W_1,& t\in[2;\infty) \\
\end{cases}
\]

\begin{enumerate}
\item Explicitely find $Z_t = \int_{0}^{t}X_{u}dW_{u}$ for every $t>0$.
\item Find $\E(Z_t)$ and $\Var(Z_t)$.
\end{enumerate}


\item (10 points) Find the price of the «Asset-or-nothing» call option at time $t=0$ in the framework of Black and Scholes model. 
The risk-free interest rate is equal to $r$. The volatility of the share is equal to $\sigma$. 
The current share price is $S_0$. 
The «Asset-or-nothing» call option pays you at fixed time $T$ the sum $S_T$ 
if $S_T$ is higher than the strike price $K$ or nothing otherwise.

Hint: the correct answer will contain the normal cumulative distribution function $F()$.

\end{enumerate}

\begin{comment}

\item (20 points) Solve the stochastic differential equation
\[
dX_t = X_t dt + 2X_t dW_t + 3 dt + 4 dW_t, \quad X_0 = 1.
\]

You may use or not use the following guiding steps:

\begin{enumerate}
\item Using substitution $S_t = \ln Y_t$ solve a simplier equation
\[
dY_t = Y_t dt + 2Y_t dW_t
\]
\item Represent $X_t$ as $X_t = Y_t \cdot Z_t$ and write down the equation for $dZ_t$.
\item Solve the equation for $Z_t$.
\item Finalise the solution and find $X_t$.
\end{enumerate}


\end{comment}

\end{document}
