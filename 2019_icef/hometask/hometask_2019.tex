\documentclass[12pt]{article}

\usepackage{tikz} % картинки в tikz
\usepackage{microtype} % свешивание пунктуации

\usepackage{array} % для столбцов фиксированной ширины

\usepackage{indentfirst} % отступ в первом параграфе

\usepackage{sectsty} % для центрирования названий частей
\allsectionsfont{\centering}

\usepackage{amsmath, amsthm, amssymb} % куча стандартных математических плюшек

\usepackage{comment}
\usepackage{amsfonts}

\usepackage[top=2cm, left=1cm, right=1cm, bottom=2cm]{geometry} % размер текста на странице

\usepackage{lastpage} % чтобы узнать номер последней страницы

\usepackage{enumitem} % дополнительные плюшки для списков
%  например \begin{enumerate}[resume] позволяет продолжить нумерацию в новом списке
\usepackage{caption}

\usepackage{hyperref} % гиперссылки

\usepackage{multicol} % текст в несколько столбцов


\usepackage{fancyhdr} % весёлые колонтитулы
\pagestyle{fancy}
\lhead{Stochastic calculus — ICEF}
\chead{2019}
\rhead{Home assignment}
\lfoot{}
\cfoot{}
\rfoot{}
\renewcommand{\headrulewidth}{0.4pt}
\renewcommand{\footrulewidth}{0.4pt}



\usepackage{todonotes} % для вставки в документ заметок о том, что осталось сделать
% \todo{Здесь надо коэффициенты исправить}
% \missingfigure{Здесь будет Последний день Помпеи}
% \listoftodos — печатает все поставленные \todo'шки


% более красивые таблицы
\usepackage{booktabs}
% заповеди из докупентации:
% 1. Не используйте вертикальные линни
% 2. Не используйте двойные линии
% 3. Единицы измерения - в шапку таблицы
% 4. Не сокращайте .1 вместо 0.1
% 5. Повторяющееся значение повторяйте, а не говорите "то же"


\usepackage{fontspec}
\usepackage{polyglossia}

\setmainlanguage{english}
\setotherlanguages{russian}

% download "Linux Libertine" fonts:
% http://www.linuxlibertine.org/index.php?id=91&L=1
\setmainfont{Linux Libertine O} % or Helvetica, Arial, Cambria
% why do we need \newfontfamily:
% http://tex.stackexchange.com/questions/91507/
\newfontfamily{\cyrillicfonttt}{Linux Libertine O}

\AddEnumerateCounter{\asbuk}{\russian@alph}{щ} % для списков с русскими буквами
% \setlist[enumerate, 2]{label=\asbuk*),ref=\asbuk*}

%% эконометрические сокращения
\DeclareMathOperator{\Cov}{Cov}
\DeclareMathOperator{\Corr}{Corr}
\DeclareMathOperator{\Var}{Var}
\DeclareMathOperator{\E}{E}
\def \hb{\hat{\beta}}
\def \hs{\hat{\sigma}}
\def \htheta{\hat{\theta}}
\def \s{\sigma}
\def \hy{\hat{y}}
\def \hY{\hat{Y}}
\def \v1{\vec{1}}
\def \e{\varepsilon}
\def \he{\hat{\e}}
\def \z{z}
\def \hVar{\widehat{\Var}}
\def \hCorr{\widehat{\Corr}}
\def \hCov{\widehat{\Cov}}
\def \cN{\mathcal{N}}
\def \P{\mathbb{P}}


\begin{document}





\section{Discrete time}

\begin{enumerate}

\item Some questions about $\sigma$-algebras. 
\begin{enumerate}
  \item You observe the result of 10 coin tosses. How many elements the $\sigma$-algebra of your information contains?
  \item Prove that a finite $\sigma$-algebra can contain only $2^k$ elements.
  \item Is union of two $\sigma$-algebras always a $\sigma$-algebra? Prove your statement.
  \item Is intersection of two $\sigma$-algebras always a $\sigma$-algebra? Prove your statement.
\end{enumerate}

\item Prove the following statement or provide a counter-example.
For any two $\sigma$-algebras $\mathcal{F}$ and $\mathcal{H}$ and a random variable $Y$
\[
  \E(\E(Y|\mathcal F)|\mathcal H) = \E(Y|\mathcal F \cap \mathcal H)
\]

\item I throw a fair die until the first six appears. 
Let's denote the total number of throws by $X$ and the number of odd integers thrown by $Y$.

\begin{enumerate}
  \item Find $\P(Y=y|X)$, $\E(Y|X)$, $\Var(Y|X)$;
  \item Find $\E(X|Y)$.
\end{enumerate}

\item I throw 100 coins. Let's denote by $X$ the number of coins that show «heads».
I throw these $X$ coins once again, leaving other coins as they are. 
Let's denote by $Y$ the number of coins that show «heads» now.

Find $\P(Y=y|X)$, $\E(Y|X)$, $\Var(Y|X)$, $\E(Y)$, $\Var(Y)$.


% \item Let $X\sim U[0;1]$, and $Y$ takes values 0 and 1 with equal probabilities. 
% Random variables $X$ and $Y$ are independent and $Z=X^{Y}$.

% Find $\E(Z|Y)$, $\Var(Z|Y)$, $\E(Z|X)$, $\Var(Z|X)$.

\item Random variables $X$ and $Y$ have joint normal distribution with zero means and covariance matrix
\[
\begin{pmatrix}
4 & -1 \\
-1 & 9 \\
\end{pmatrix}.
\]

\begin{enumerate}
  \item Find $\E(Y|X)$, $\Var(Y|X)$, $\E(XY|X)$ and $\Var(XY|X)$.
  \item Using standard normal cumulative distribution function find $\P(YX > 2019 | X)$.
\end{enumerate}


\item The random variables $Z_1$, $Z_2$, \ldots{}  are independent and identically distributed with 
$\P(Z_n = 1) = p$ and $\P(Z_n = -1) = 1-p$. Consider the cumulative sum process, $S_n = Z_1 + \ldots + Z_n$ with $S_0=0$.

\begin{enumerate}
  \item For which value of $p$ the process $1.5^{S_n}$ will be a martingale.
  \item Let $p=0.4$. 
  If possible find the constants $\alpha$ and $\beta$ such that $Y_n = S_n^2 + \alpha S_n + \beta n$ is a martingale?
\end{enumerate}

\item Anna and Boris throw a coin infinite number of times.
Anna wins if the sequence HTHH appears first, Boris wins if the sequence TTHH appears first. 
The coin is biased with $0.4$ probability of head.

\begin{enumerate}
  \item What is the expected number of throws to obtain HTHH? 
  \item What is the expected number of throws to obtain TTHH?
  \item What is the probability that Anna will win?
  \item What is the expected number of throws to obtain HTHH or TTHH?
\end{enumerate}

\item Have a look in the past exams collection. How many pages does it contain?


\end{enumerate}



\newpage
\section{Continuous time stochastic processes}

Due to late posting date this part is not mandatory. Be happy and study stochastic calculus :)

\begin{enumerate}

\item The processes $(X_t)$ and $(Y_t)$ are independent Wiener processes with respect to filtration $(\mathcal{F}_t)$.
The process $Z_t = aX_t + b Y_t$ is also a Wiener process. 
\begin{enumerate}
  \item For which values of constants $a$ and $b$ is it possible?
  \item Find covariance $\Corr(Z_t, X_t)$.
  \item Find $\E(Z_3 | X_2)$ and $\Var(Z_3 | X_2)$.
  \item Find $\E(Z_3 | \mathcal{F}_2)$ and $\Var(Z_3 | \mathcal{F}_2)$.
\end{enumerate}


\item The process $C_t = W_t^3 + aW_t^2 +b W_t + c + d\cdot t\cdot W_t$ is a martingale.
\begin{enumerate}
  \item For which values of constants $a$, $b$, $c$ and $d$ is it possible?
  \item Find covariance $\Cov(C_t, \int W^2_u dW_u)$.
\end{enumerate}

\item In the framework of Black and Scholes model derive the current price of a European call option with 
strike price $K$ and maturity date $T$. 

The European call option pays you the sum
\[
X_T =   \begin{cases}
S_T - K, \text{ if } S_T > K; \\
0, \text{ otherwise.}
\end{cases}
\]

\item Let $Y_t = W_t + 3t$. The moment $\tau$ is the first moment when Wiener process hits $10$. 
\begin{enumerate}
  \item Let $\alpha$ be a constant. Find the function $f(t)$ such that $M_t = f(t)\exp(\alpha Y_t)$ is a martingale.
  \item Using Doob's theorem find $\E(\exp(-s \tau))$ for arbitrary constant $s$. 
\end{enumerate}

\item Let's consider the process $X_t = \int_0^t u^2 dW_u$. 
Prove that this process can be represented as a time changed Wiener process. 
That means that there is a deterministic time-scaling $t(s)$ 
such that $Y_s = X_{t(s)}$ is a Wiener process with respect to some filtration. 

\item Solve the stochatic differential equation
\[
dY_t = - Y_t dt + dW_t, \; Y_0 = 1
\]

If you are have no clues you may try a substitution $Z_t = f(t) Y_t$. 
Do not forget that the final answer may contain integrals that can't be calculated explicitely. It's ok.

\item Solve the stochatic differential equation
\[
dY_t = Y_t dt + (t^3 + 4Y_t) dW_t, \; Y_0 = 1
\]

If you are have no clues you may try to represent the process as $Y_t = A_t B_t$, 
where $A_t$ is the solution of the equation $dA_t = A_t dt + 4A_t dW_t$. 


\end{enumerate}


\end{document}
