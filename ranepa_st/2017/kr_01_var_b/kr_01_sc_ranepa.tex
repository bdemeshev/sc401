\documentclass[12pt]{article}


\usepackage{amsmath, amssymb}
\usepackage{tikz} % картинки в tikz
\usepackage{microtype} % свешивание пунктуации

\usepackage{array} % для столбцов фиксированной ширины

\usepackage{indentfirst} % отступ в первом параграфе

\usepackage{sectsty} % для центрирования названий частей
\allsectionsfont{\centering}

\usepackage{amsmath} % куча стандартных математических плюшек

\usepackage[top=2cm, left=1.5cm, right=1.5cm, bottom=2cm]{geometry} % размер текста на странице

\usepackage{lastpage} % чтобы узнать номер последней страницы

\usepackage{enumitem} % дополнительные плюшки для списков
%  например \begin{enumerate}[resume] позволяет продолжить нумерацию в новом списке
\usepackage{caption}


\usepackage{fancyhdr} % весёлые колонтитулы
\pagestyle{fancy}
\lhead{Стохастический анализ}
\chead{}
\rhead{2017-10-27, праздник номер раз :)}
\lfoot{}
\cfoot{}
\rfoot{\thepage/\pageref{LastPage}}
\renewcommand{\headrulewidth}{0.4pt}
\renewcommand{\footrulewidth}{0.4pt}



\usepackage{todonotes} % для вставки в документ заметок о том, что осталось сделать
% \todo{Здесь надо коэффициенты исправить}
% \missingfigure{Здесь будет Последний день Помпеи}
% \listoftodos --- печатает все поставленные \todo'шки


% более красивые таблицы
\usepackage{booktabs}
% заповеди из докупентации:
% 1. Не используйте вертикальные линни
% 2. Не используйте двойные линии
% 3. Единицы измерения - в шапку таблицы
% 4. Не сокращайте .1 вместо 0.1
% 5. Повторяющееся значение повторяйте, а не говорите "то же"


\usepackage{bbm}
\usepackage{fontspec}
\usepackage{polyglossia}

\setmainlanguage{russian}
\setotherlanguages{english}

% download "Linux Libertine" fonts:
% http://www.linuxlibertine.org/index.php?id=91&L=1
\setmainfont{Linux Libertine O} % or Helvetica, Arial, Cambria
% why do we need \newfontfamily:
% http://tex.stackexchange.com/questions/91507/
\newfontfamily{\cyrillicfonttt}{Linux Libertine O}

\AddEnumerateCounter{\asbuk}{\russian@alph}{щ} % для списков с русскими буквами


%% эконометрические сокращения
\DeclareMathOperator{\card}{card}
\DeclareMathOperator{\Cov}{Cov}
\DeclareMathOperator{\Corr}{Corr}
\DeclareMathOperator{\Var}{Var}
\DeclareMathOperator{\E}{E}
\def \hb{\hat{\beta}}
\def \hs{\hat{\sigma}}
\def \htheta{\hat{\theta}}
\def \s{\sigma}
\def \hy{\hat{y}}
\def \hY{\hat{Y}}
\def \v1{\vec{1}}
\def \e{\varepsilon}
\def \he{\hat{\e}}
\def \z{z}
\def \hVar{\widehat{\Var}}
\def \hCorr{\widehat{\Corr}}
\def \hCov{\widehat{\Cov}}
\def \cN{\mathcal{N}}


\renewcommand{\P}{\mathbb{P}}

\begin{document}

\begin{enumerate}

\item Рассмотрим множество $M$ всех многочленов с натуральными коэффициентами. Например, $5x^2+6x+7 \in M$, а $\cos x \notin M$.
\begin{enumerate}
  \item Найдите мощность множества $M$.
  \item (*) Джейсон Борн хочет узнать все коэффициенты этого многочлена. Джейсон Борн имеет право задавать только вопросы о значениях этого многочлена в натуральных точках по своему выбору. Сколько вопросов ему достаточно задать?
\end{enumerate}

\item У майора Пронина есть несколько множеств:

\begin{itemize}
\item $A$ — все подмножества натуральных чисел; например, $\{5, 6, 11\} \in A$, а $\{5, 6.5\} \notin A$.
\item $B$ — все подмножества натуральных чисел, состоящие из одного «куска» подряд идущих чисел; например, $\{5, 6, 7\} \in B$, а $\{5, 6, 11\} \notin B$.
\item $C$ — все подмножества натуральных чисел, состоящие из конечного числа «кусков»; например, $\{5, 6, 7, 19, 20\} \in C$, а $\{2k | k \in \mathbb{N}\} \notin C$.
\end{itemize}

Помогите майору Пронину определить, какие из этих множеств равномощны $\mathbb{N}$, и какие равномощны $\mathbb{R}$.

\item Джон Смит равновероятно выбирает одну из четырех точек на плоскости: $(0, 0)$, $(1, 1)$, $(0, 2)$, $(1, -3)$. Пусть $X$ и $Y$ — координаты выбранной точки.

Найдите $\E(Y|X)$, $\Var(Y|X)$.

\item Шерлок Холмс подбрасывает кубик бесконечное количество раз. Пусть $X$ — номер броска, когда впервые выпало пять, а $Y$ — когда впервые выпало пять два раза подряд.

Найдите $\E(Y|X)$, $\Var(Y|X)$.

\item Маша пошла в лес по грибы да по ягоды. Она собирает лисички, шампиньоны и рыжики до тех пор, пока не соберет $n$ грибов. Лисички, шампиньоны и рыжики встречаются в лесу с некоторыми фиксированными вероятностями. Пусть $X$ — количество лисичек, а $Y$ — рыжиков. Известно, что $\E(X)=60$, $\E(Y)=20$, $\Var(X)=24$.

Найдите $\E(Y|X)$, $\Var(Y|X)$.

\item Джеймс Бонд подбрасывает правильную монетку 30 раз. Величина $X$ — количество орлов при первых 20 бросках, $Y$ — количество орлов при последних 20 бросках.

Найдите $\E(X|Y)$, $\E(Y|X)$, $\Var(Y|X)$.

\end{enumerate}

\end{document}
