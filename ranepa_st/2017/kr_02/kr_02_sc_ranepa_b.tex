\documentclass[12pt]{article}


\usepackage{amsmath, amssymb}
\usepackage{tikz} % картинки в tikz
\usepackage{microtype} % свешивание пунктуации

\usepackage{array} % для столбцов фиксированной ширины

\usepackage{indentfirst} % отступ в первом параграфе

\usepackage{sectsty} % для центрирования названий частей
\allsectionsfont{\centering}

\usepackage{amsmath} % куча стандартных математических плюшек

\usepackage[top=2cm, left=1.5cm, right=1.5cm, bottom=2cm]{geometry} % размер текста на странице

\usepackage{lastpage} % чтобы узнать номер последней страницы

\usepackage{enumitem} % дополнительные плюшки для списков
%  например \begin{enumerate}[resume] позволяет продолжить нумерацию в новом списке
\usepackage{caption}


\usepackage{fancyhdr} % весёлые колонтитулы
\pagestyle{fancy}
\lhead{Стохастический анализ}
\chead{}
\rhead{2018-01-18, праздник номер два :)}
\lfoot{}
\cfoot{}
\rfoot{\thepage/\pageref{LastPage}}
\renewcommand{\headrulewidth}{0.4pt}
\renewcommand{\footrulewidth}{0.4pt}



\usepackage{todonotes} % для вставки в документ заметок о том, что осталось сделать
% \todo{Здесь надо коэффициенты исправить}
% \missingfigure{Здесь будет Последний день Помпеи}
% \listoftodos --- печатает все поставленные \todo'шки


% более красивые таблицы
\usepackage{booktabs}
% заповеди из докупентации:
% 1. Не используйте вертикальные линни
% 2. Не используйте двойные линии
% 3. Единицы измерения - в шапку таблицы
% 4. Не сокращайте .1 вместо 0.1
% 5. Повторяющееся значение повторяйте, а не говорите "то же"


\usepackage{bbm}
\usepackage{fontspec}
\usepackage{polyglossia}

\setmainlanguage{russian}
\setotherlanguages{english}

% download "Linux Libertine" fonts:
% http://www.linuxlibertine.org/index.php?id=91&L=1
\setmainfont{Linux Libertine O} % or Helvetica, Arial, Cambria
% why do we need \newfontfamily:
% http://tex.stackexchange.com/questions/91507/
\newfontfamily{\cyrillicfonttt}{Linux Libertine O}

\AddEnumerateCounter{\asbuk}{\russian@alph}{щ} % для списков с русскими буквами


%% эконометрические сокращения
\DeclareMathOperator{\card}{card}
\DeclareMathOperator{\Cov}{Cov}
\DeclareMathOperator{\Corr}{Corr}
\DeclareMathOperator{\Var}{Var}
\DeclareMathOperator{\E}{E}
\def \hb{\hat{\beta}}
\def \hs{\hat{\sigma}}
\def \htheta{\hat{\theta}}
\def \s{\sigma}
\def \hy{\hat{y}}
\def \hY{\hat{Y}}
\def \v1{\vec{1}}
\def \e{\varepsilon}
\def \he{\hat{\e}}
\def \z{z}
\def \hVar{\widehat{\Var}}
\def \hCorr{\widehat{\Corr}}
\def \hCov{\widehat{\Cov}}
\def \cN{\mathcal{N}}


\renewcommand{\P}{\mathbb{P}}

\begin{document}





\begin{enumerate}

\item Для процесса $Y_t = W_t^3\cos W_t + tW_t$ найдите $dY_t$ и выпишите ответ в полной форме записи.

\item Для процесса $Y_t = f(t) \exp(t+W_t)$ найдите $dY_t$ и подберите функцию $f(t)$ так, чтобы процесс $Y_t$ был мартингалом.


\item Найдите $\E(\int_0^t W_s \cos s dW_s)$ и $\Var(\int_0^t W_s \cos s dW_s)$

\item Для броуновского движения $W_t$ определим величину $Y$ равную единице, если $W_2>0$, и нулю иначе. Найдите $\Cov(Y,W_3)$, $\Cov(Y, W_1)$.

\item The process $X_t$ is given by
\[
  X_t = 2017 + t^2 W_t^2 + \int_0^t u \, dW_u
\]
\begin{enumerate}
  \item Find $dX_t$;
  \item Is $X_t$ a martingale?
  \item Find $\E(X_t)$.
\end{enumerate}

\item The process $Y_t$ is given by $Y_t=2W_t+5t$. The stopping time $\tau$ is given by $\tau=\min\{t|Y_t^2=100\}$. Find the distribution of the random variable $Y_\tau$ and the expected value $\E(\tau)$.


Hint: you may find the martingales $a^{Y_t}$ and $Y_t-f(t)$ useful


\item Consider the framework of the Black and Scholes model. You agreed with Warren Buffett that at fixed time $T$ he will pay you the strange sum
\[
X_T = \ln S_T \cdot \ln S_{T/2},
\]
where $S_t$ is the price of a share.

What is the non-arbitrage price $X_0$ of this agreement?




\end{enumerate}

\end{document}
