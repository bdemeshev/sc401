\documentclass[12pt]{article}


\usepackage{amsmath, amssymb}
\usepackage{tikz} % картинки в tikz
\usepackage{microtype} % свешивание пунктуации

\usepackage{array} % для столбцов фиксированной ширины

\usepackage{indentfirst} % отступ в первом параграфе

\usepackage{sectsty} % для центрирования названий частей
\allsectionsfont{\centering}

\usepackage{amsmath} % куча стандартных математических плюшек

\usepackage[top=2cm, left=1.5cm, right=1.5cm, bottom=2cm]{geometry} % размер текста на странице

\usepackage{lastpage} % чтобы узнать номер последней страницы

\usepackage{enumitem} % дополнительные плюшки для списков
%  например \begin{enumerate}[resume] позволяет продолжить нумерацию в новом списке
\usepackage{caption}


\usepackage{fancyhdr} % весёлые колонтитулы
\pagestyle{fancy}
\lhead{Стохастический анализ}
\chead{}
\rhead{2016-11-25, праздник номер раз :)}
\lfoot{}
\cfoot{}
\rfoot{\thepage/\pageref{LastPage}}
\renewcommand{\headrulewidth}{0.4pt}
\renewcommand{\footrulewidth}{0.4pt}



\usepackage{todonotes} % для вставки в документ заметок о том, что осталось сделать
% \todo{Здесь надо коэффициенты исправить}
% \missingfigure{Здесь будет Последний день Помпеи}
% \listoftodos --- печатает все поставленные \todo'шки


% более красивые таблицы
\usepackage{booktabs}
% заповеди из докупентации:
% 1. Не используйте вертикальные линни
% 2. Не используйте двойные линии
% 3. Единицы измерения - в шапку таблицы
% 4. Не сокращайте .1 вместо 0.1
% 5. Повторяющееся значение повторяйте, а не говорите "то же"


\usepackage{bbm}
\usepackage{fontspec}
\usepackage{polyglossia}

\setmainlanguage{russian}
\setotherlanguages{english}

% download "Linux Libertine" fonts:
% http://www.linuxlibertine.org/index.php?id=91&L=1
\setmainfont{Linux Libertine O} % or Helvetica, Arial, Cambria
% why do we need \newfontfamily:
% http://tex.stackexchange.com/questions/91507/
\newfontfamily{\cyrillicfonttt}{Linux Libertine O}

\AddEnumerateCounter{\asbuk}{\russian@alph}{щ} % для списков с русскими буквами


%% эконометрические сокращения
\DeclareMathOperator{\card}{card}
\DeclareMathOperator{\Cov}{Cov}
\DeclareMathOperator{\Corr}{Corr}
\DeclareMathOperator{\Var}{Var}
\DeclareMathOperator{\E}{E}
\def \hb{\hat{\beta}}
\def \hs{\hat{\sigma}}
\def \htheta{\hat{\theta}}
\def \s{\sigma}
\def \hy{\hat{y}}
\def \hY{\hat{Y}}
\def \v1{\vec{1}}
\def \e{\varepsilon}
\def \he{\hat{\e}}
\def \z{z}
\def \hVar{\widehat{\Var}}
\def \hCorr{\widehat{\Corr}}
\def \hCov{\widehat{\Cov}}
\def \cN{\mathcal{N}}


\renewcommand{\P}{\mathbb{P}}

\begin{document}





\begin{enumerate}

\item Рассмотрим множество последовательностей из произвольных натуральных чисел, обозначим его буквой $A$. Например, одним элементом $A$ является последовательность $(1, 2, 3, 4, \ldots)$. Определим подмножество $B \subset A$, последовательностей в которых единица упомянута не больше 1 раза, двойка — не более двух раз, тройка — не более трёх и так далее. Определим подмножество $C \subset A$, последовательностей, в которых все числа кроме числа 2016 упоминаются конечное количество раз, а число 2016 может упоминаться любое количество раз.

Найдите $\card A$, $\card B$, $\card C$

\item У Буратино есть три монетки: одна целиком зелёная, вторая — целиком красная и третья — со стороны орла зелёная, со стороны решки — красная. Сначала Буратино подбрасывает цветную монетку. Если цветная монетка выпадает красной стороной, то Буратино подбрасывает красную монетку, если зелёной — то зелёную. Вероятности выпадения орла равны: $0.2$ — для красной монетки, $0.4$ — для зелёной, $0.7$ — для цветной. Пусть $X$ — индикатор того, выпал ли орёл на цветной монетке, а $Y$ — индикатор того, выпал ли орёл при втором броске.

Найдите $\E(Y|X)$, $\E(X|Y)$, $\Var(X|Y)$

\item Величины $X_{1}, \ldots,  X_{100}$ независимы и равномерны на  отрезке $ [0;1] $. Пусть $L=\max\{X_{1},X_{2},\ldots, X_{80}\}$ а $R=\max\{X_{81},X_{82},\ldots,X_{100}\}$ и $M=\max\{X_{1},\ldots,X_{100}\}$

Найдите
\begin{enumerate}
\item $ \P(L>R|L)$ и $ \P(L>R|R) $ и $ \P(L>R|M)$, $\P(L>R|L,M) $
\item $\E( X_1 | L )$, $\E( X_1 | \min \{ X_1, \ldots, X_{100} \})$
\end{enumerate}

\item You throw a fair coin infinite number of times. Let's denote the result of the second toss by $Y_2$ ($0$ for tail and $1$ for head) and the number of throws to get the first «head» by $N$. Find $\E(Y_2|N)$, $\Var(Y_2|N)$ and $\E(N|Y_2)$

\item It is known that $\E(Y|X)=0$. Which of the following quantities must be zero: $\E(Y)$? $\E(X)$? $\Cov(X,Y)$? $\Cov(X^2,Y)$? $\Cov(X,Y^2)$? Prove or provide a counter-example.

\item The random variables $X_1$, $X_2$, \ldots, $X_n$, \ldots are independent uniformly distributed on $[0; 1]$. I am summing them until the first $X_i$ greater than 0.5 is added. After this term I stop. Let’s denote by $S$ the total sum and by $N$ — the number of terms added. Find $\E(S|N)$, $\Var(S|N)$, $E(S)$


\end{enumerate}

\end{document}
