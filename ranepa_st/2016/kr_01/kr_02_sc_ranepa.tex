\documentclass[12pt]{article}


\usepackage{amsmath, amssymb}
\usepackage{tikz} % картинки в tikz
\usepackage{microtype} % свешивание пунктуации

\usepackage{array} % для столбцов фиксированной ширины

\usepackage{indentfirst} % отступ в первом параграфе

\usepackage{sectsty} % для центрирования названий частей
\allsectionsfont{\centering}

\usepackage{amsmath} % куча стандартных математических плюшек

\usepackage[top=2cm, left=1.5cm, right=1.5cm, bottom=2cm]{geometry} % размер текста на странице

\usepackage{lastpage} % чтобы узнать номер последней страницы

\usepackage{enumitem} % дополнительные плюшки для списков
%  например \begin{enumerate}[resume] позволяет продолжить нумерацию в новом списке
\usepackage{caption}


\usepackage{fancyhdr} % весёлые колонтитулы
\pagestyle{fancy}
\lhead{Стохастический анализ}
\chead{}
\rhead{2016-12-23, праздник номер два :)}
\lfoot{}
\cfoot{}
\rfoot{\thepage/\pageref{LastPage}}
\renewcommand{\headrulewidth}{0.4pt}
\renewcommand{\footrulewidth}{0.4pt}



\usepackage{todonotes} % для вставки в документ заметок о том, что осталось сделать
% \todo{Здесь надо коэффициенты исправить}
% \missingfigure{Здесь будет Последний день Помпеи}
% \listoftodos --- печатает все поставленные \todo'шки


% более красивые таблицы
\usepackage{booktabs}
% заповеди из докупентации:
% 1. Не используйте вертикальные линни
% 2. Не используйте двойные линии
% 3. Единицы измерения - в шапку таблицы
% 4. Не сокращайте .1 вместо 0.1
% 5. Повторяющееся значение повторяйте, а не говорите "то же"


\usepackage{bbm}
\usepackage{fontspec}
\usepackage{polyglossia}

\setmainlanguage{russian}
\setotherlanguages{english}

% download "Linux Libertine" fonts:
% http://www.linuxlibertine.org/index.php?id=91&L=1
\setmainfont{Linux Libertine O} % or Helvetica, Arial, Cambria
% why do we need \newfontfamily:
% http://tex.stackexchange.com/questions/91507/
\newfontfamily{\cyrillicfonttt}{Linux Libertine O}

\AddEnumerateCounter{\asbuk}{\russian@alph}{щ} % для списков с русскими буквами


%% эконометрические сокращения
\DeclareMathOperator{\card}{card}
\DeclareMathOperator{\Cov}{Cov}
\DeclareMathOperator{\Corr}{Corr}
\DeclareMathOperator{\Var}{Var}
\DeclareMathOperator{\E}{E}
\def \hb{\hat{\beta}}
\def \hs{\hat{\sigma}}
\def \htheta{\hat{\theta}}
\def \s{\sigma}
\def \hy{\hat{y}}
\def \hY{\hat{Y}}
\def \v1{\vec{1}}
\def \e{\varepsilon}
\def \he{\hat{\e}}
\def \z{z}
\def \hVar{\widehat{\Var}}
\def \hCorr{\widehat{\Corr}}
\def \hCov{\widehat{\Cov}}
\def \cN{\mathcal{N}}


\renewcommand{\P}{\mathbb{P}}

\begin{document}





\begin{enumerate}

\item Для процесса $Y_t = W_t^5 + tW_t$ найдите $dY_t$ и выпишите ответ в полной форме записи.

\item Для процесса $Y_t = f(t) \exp(4W_t)$ найдите $dY_t$ и подберите функцию $f(t)$ так, чтобы процесс $Y_t$ был мартингалом.



\item Найдите $\E(\int_0^t sW_s dW_s)$ и $\Var(\int_0^t sW_s dW_s)$

\item Для броуновского движения $W_t$ найдите $\E(W_5(W_6 - W_3))$.

\item Find $\Var\left(  \int_0^t W_s \, ds  \right)$.

You may use the following guiding steps:

\begin{enumerate}
\item Find $d(tW_t)$ in short and full form
\item Find $\E\left(2t W_t\int_0^t s \, dW_s\right)$
\item Find $\E\left(\left(\int_0^t s \, dW_s\right)^2 \right)$
\item Find $\E\left(  \int_0^t W_s \, ds  \right)$
\item $(a-b)^2=a^2-2ab+b^2$ :)
\end{enumerate}

\item Researcher Veniamin throws a fair dice until 6 appears. Let denote by $T$ the total number of throws and by $N$ the number of throws when 5 appeared. Find $\E(N | T)$, $\Var(N | T)$, $\E(N)$, $\Var(N)$ and $\E(T|N)$.

The joint distribution of the random vector $(X,Y)$ is given by its probability density function
\[
f(x,y)=
\left\{
  \begin{array}{l}
    c e^{x-y}, \text{ for } 0\leq x,y\leq 1 \\
    0, \text{ otherwise}
  \end{array}
\right.
\]
where $c$ is a normalization constant. Find $\E(X\mid Y)$.


\item В рамках модели Блэка-Шоулза предполагается, что $S_t = S_0 \exp((\mu- \sigma^2/2)t + \sigma W_t)$. Переходи к риск-нейтральной вероятности сопровождается заменой $\tilde{W}_t = W_t + \frac{\mu-r}{\sigma}t$. В рамках данных обозначений рассчитайте текущую стоимость актива, который через $T$ лет стоит $\ln S_T$.





\end{enumerate}

\end{document}
