\documentclass[11pt]{article}


\usepackage{amsmath, amssymb}
\usepackage{tikz} % картинки в tikz
\usepackage{microtype} % свешивание пунктуации

\usepackage{array} % для столбцов фиксированной ширины

\usepackage{indentfirst} % отступ в первом параграфе

\usepackage{sectsty} % для центрирования названий частей
\allsectionsfont{\centering}

\usepackage{amsmath} % куча стандартных математических плюшек

\usepackage[top=2cm, left=1.5cm, right=1.5cm, bottom=2cm]{geometry} % размер текста на странице

\usepackage{lastpage} % чтобы узнать номер последней страницы

\usepackage{enumitem} % дополнительные плюшки для списков
%  например \begin{enumerate}[resume] позволяет продолжить нумерацию в новом списке
\usepackage{caption}


\usepackage{fancyhdr} % весёлые колонтитулы
\pagestyle{fancy}
\lhead{Стохастический анализ}
\chead{}
\rhead{2019-11-08, праздник номер раз :)}
\lfoot{}
\cfoot{}
\rfoot{\thepage/\pageref{LastPage}}
\renewcommand{\headrulewidth}{0.4pt}
\renewcommand{\footrulewidth}{0.4pt}



\usepackage{todonotes} % для вставки в документ заметок о том, что осталось сделать
% \todo{Здесь надо коэффициенты исправить}
% \missingfigure{Здесь будет Последний день Помпеи}
% \listoftodos --- печатает все поставленные \todo'шки


% более красивые таблицы
\usepackage{booktabs}
% заповеди из докупентации:
% 1. Не используйте вертикальные линни
% 2. Не используйте двойные линии
% 3. Единицы измерения - в шапку таблицы
% 4. Не сокращайте .1 вместо 0.1
% 5. Повторяющееся значение повторяйте, а не говорите "то же"


\usepackage{bbm}
\usepackage{fontspec}
\usepackage{polyglossia}

\setmainlanguage{russian}
\setotherlanguages{english}

% download "Linux Libertine" fonts:
% http://www.linuxlibertine.org/index.php?id=91&L=1
\setmainfont{Linux Libertine O} % or Helvetica, Arial, Cambria
% why do we need \newfontfamily:
% http://tex.stackexchange.com/questions/91507/
\newfontfamily{\cyrillicfonttt}{Linux Libertine O}

\AddEnumerateCounter{\asbuk}{\russian@alph}{щ} % для списков с русскими буквами


%% эконометрические сокращения
\DeclareMathOperator{\card}{card}
\DeclareMathOperator{\Cov}{Cov}
\DeclareMathOperator{\Corr}{Corr}
\DeclareMathOperator{\Var}{Var}
\DeclareMathOperator{\E}{E}
\def \hb{\hat{\beta}}
\def \hs{\hat{\sigma}}
\def \htheta{\hat{\theta}}
\def \s{\sigma}
\def \hy{\hat{y}}
\def \hY{\hat{Y}}
\def \v1{\vec{1}}
\def \e{\varepsilon}
\def \he{\hat{\e}}
\def \z{z}
\def \hVar{\widehat{\Var}}
\def \hCorr{\widehat{\Corr}}
\def \hCov{\widehat{\Cov}}
\def \cN{\mathcal{N}}


\renewcommand{\P}{\mathbb{P}}

\begin{document}

\begin{enumerate}

\item У Илона Маска есть множество $A$ всех периодических последовательностей натуральных чисел.
Например, $(2, 3, 1, 2, 3, 1, 2, 3, 1, \ldots) \in A$,
$(1, 2, 3, 4, 5, 6, 7, \ldots) \notin A$.

\begin{enumerate}
  \item Помогите Илону Маску определить,  равномощно ли множество $A$ множеству $\mathbb{N}$? 
  Равномощно ли оно $\mathbb{R}$?
 \item Помогите Илону Маску определить,  равномощно ли множество всех последовательностей натуральных чисел 
 множеству $\mathbb{N}$? 
 Равномощно ли оно $\mathbb{R}$?
\end{enumerate}



\item Грета Тунберг подбрасывает правильную монетку $n$ раз. 
Обозначим $Y_i$ индикатор выпадения орла при $i$-ом броске. 

Обозначим также $L = \min \{Y_1, Y_2, \ldots, Y_n\}$ и $R = \max \{Y_1, Y_2, \ldots, Y_n\}$.

\begin{enumerate}
  \item Найдите $\P(R=1|L)$, $\P(L=1|R)$;
  \item Найдите $\E(R|L)$, $\E(L|R)$;
  \item Найдите $\Var(R|L)$, $\Var(L|R)$.
\end{enumerate}


\item Эксперимент может окончится одним из четырёх исходов, $a$, $b$, $c$ или $d$.

\begin{enumerate}
  \item Выпишите минимальную сигма-алгебру $\mathcal{F}$, содержащую события $\{a,b,c\}$ и $\{b,c,d\}$.
  \item Приведите пример случайной величины, измеримой относительно $\mathcal{F}$, и пример величины,
  не измеримой относительно $\mathcal{F}$.
  \item Сколько существуют случайных величин, принимающих значения из множестве $\{1, 2, 3, 4\}$, 
  измеримых относительно $\mathcal{F}$?
\end{enumerate}


\item Известно, что величина $X$ распределена равномерно на отрезке $[0;1]$, 
а величина $Y$ при фиксированном $X$ распределена равномерно на отрезке $[1; 1/X]$.

\begin{enumerate}
  \item Найдите $\E(Y|X)$, $\Var(Y|X)$;
  \item Найдите $\E(Y)$, $\Var(Y)$;
  \item Найдите $\E(X|Y)$, $\Var(X|Y)$;
\end{enumerate}


\item Рассмотрим независимые величины $Y_n$, принимающие значения $+1$ или $-1$, $\P(Y_n = 1) = p$.
Определим $S_n=\sum_{i=1}^n Y_n$.

\begin{enumerate}
  \item При каком $p$ процесс $S_n+n/10$ будет мартингалом?
  \item При каком $p$ процесс $0.7^{S_n}$ будет мартингалом?
  \item Пусть $p=1/2$. При какой функции $h(n)$ процесс $S_n^2 + h(n)$ будет мартингалом?
\end{enumerate}


\item Приезжающих из армии или от двора встречают $n$ женщин. Они одновременно подбрасывают
вверх $n$ чепчиков. Ловят чепчики наугад, каждая женщина ловит один чепчик. Женщины,
поймавшие свой чепчик уходят. А женщины, поймавшие чужой чепчик, снова подбрасывают его
вверх. Подбрасывание чепчиков продолжается до тех пор, пока каждая не поймает свой чепчик.
Найдите:

\begin{enumerate}
  \item Cреднее количество женщин, поймавших свой чепчик при одном подбрасывании;
  \item Cреднее количество подбрасываний.
\end{enumerate}



\end{enumerate}

\end{document}
