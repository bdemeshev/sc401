\documentclass[13pt]{article}


\usepackage{amsmath, amssymb}
\usepackage{tikz} % картинки в tikz
\usepackage{microtype} % свешивание пунктуации

\usepackage{array} % для столбцов фиксированной ширины

\usepackage{indentfirst} % отступ в первом параграфе

\usepackage{sectsty} % для центрирования названий частей
\allsectionsfont{\centering}

\usepackage{amsmath} % куча стандартных математических плюшек

\usepackage[top=2cm, left=1.5cm, right=1.5cm, bottom=2cm]{geometry} % размер текста на странице

\usepackage{lastpage} % чтобы узнать номер последней страницы

\usepackage{enumitem} % дополнительные плюшки для списков
%  например \begin{enumerate}[resume] позволяет продолжить нумерацию в новом списке
\usepackage{caption}


\usepackage{fancyhdr} % весёлые колонтитулы
\pagestyle{fancy}
\lhead{Стохастический анализ}
\chead{}
\rhead{2020-01-16, праздник номер два :)}
\lfoot{}
\cfoot{}
\rfoot{} % {\thepage/\pageref{LastPage}}
\renewcommand{\headrulewidth}{0.4pt}
\renewcommand{\footrulewidth}{0.4pt}



\usepackage{todonotes} % для вставки в документ заметок о том, что осталось сделать
% \todo{Здесь надо коэффициенты исправить}
% \missingfigure{Здесь будет Последний день Помпеи}
% \listoftodos --- печатает все поставленные \todo'шки


% более красивые таблицы
\usepackage{booktabs}
% заповеди из докупентации:
% 1. Не используйте вертикальные линни
% 2. Не используйте двойные линии
% 3. Единицы измерения - в шапку таблицы
% 4. Не сокращайте .1 вместо 0.1
% 5. Повторяющееся значение повторяйте, а не говорите "то же"


\usepackage{bbm}
\usepackage{fontspec}
\usepackage{polyglossia}

\setmainlanguage{russian}
\setotherlanguages{english}

% download "Linux Libertine" fonts:
% http://www.linuxlibertine.org/index.php?id=91&L=1
\setmainfont{Linux Libertine O} % or Helvetica, Arial, Cambria
% why do we need \newfontfamily:
% http://tex.stackexchange.com/questions/91507/
\newfontfamily{\cyrillicfonttt}{Linux Libertine O}

\AddEnumerateCounter{\asbuk}{\russian@alph}{щ} % для списков с русскими буквами


%% эконометрические сокращения
\DeclareMathOperator{\card}{card}
\DeclareMathOperator{\Cov}{Cov}
\DeclareMathOperator{\Corr}{Corr}
\DeclareMathOperator{\Var}{Var}
\DeclareMathOperator{\E}{E}
\def \hb{\hat{\beta}}
\def \hs{\hat{\sigma}}
\def \htheta{\hat{\theta}}
\def \s{\sigma}
\def \hy{\hat{y}}
\def \hY{\hat{Y}}
\def \v1{\vec{1}}
\def \e{\varepsilon}
\def \he{\hat{\e}}
\def \z{z}
\def \hVar{\widehat{\Var}}
\def \hCorr{\widehat{\Corr}}
\def \hCov{\widehat{\Cov}}
\def \cN{\mathcal{N}}


\renewcommand{\P}{\mathbb{P}}

\begin{document}

\begin{enumerate}

  \item Вспомним свойства броуновского движения :)
  \begin{enumerate}
    \item Найдите $\E(W_3 | W_1)$, $\Var(W_3 | W_1)$;
    \item Найдите $\E(W_1 | W_3)$, $\E(W_1 | \mathcal{F}_3)$, $\Var(W_1 | W_3)$, $\Var(W_1 | \mathcal{F}_3)$;
    \item Найдите $\E(W_3^3 | W_1)$, $\E(W_1^3 | W_3)$.
  \end{enumerate}
  

  \item Рассмотрим процесс $C_t = W_t^3 + 7 W_t + 5 + d\cdot t\cdot W_t$.
  \begin{enumerate}
    \item Найдите $dC_t$.
    \item При каких значениях $d$ процесс $C_t$ будет мартингалом?
    \item Для полученного значения $d$ найдите $\Cov(C_t, \int W^2_u dW_u)$.
  \end{enumerate}

  \item Рассмотрим процесс $Y_t = \exp(\sigma W_t + bt)$. 
  \begin{enumerate}
    \item Найдите $dY_t$.
    \item Как должны быть связаны $b$ и $\sigma$, чтобы $Y_t$ был мартингалом?
    \item Найдите $\E(\exp(\sigma W_t))$.
  \end{enumerate}
  

  \item Рассмотрим процесс $Y_t = W_t + 2t$. Пусть $\tau$ — первый момент времени, когда $W_t$ окажется равным $5$. 
\begin{enumerate}
  \item Пусть $\alpha$ — произвольная константа, найдите такую функцию $f(t)$, 
  что процесс $M_t = f(t)\exp(\alpha Y_t)$ является мартингалом. 
  \item Используя теорему Дуба найдите $\E(\exp(-s \tau))$ для произвольной константы $s$. 
\end{enumerate}

Подсказка: для первого пункта найдите $dM_t$ и что-то там к чему-то там приравняйте :)

\item Эта задача объясняет, для чего искать такую полезную штуку, как $\E(\exp(-s \tau))$.
При этом ничего из предыдущей задачи не используется :) Итак, пусть $m(s) = \E(\exp(-s \tau))$.
\begin{enumerate}
  \item Найдите $m'(s)$, $m'(0)$, $m''(s)$, $m''(0)$.
  \item Известно, что $m'(0)= -5$, а $m''(0)=100$, найдите $\E(\tau)$ и $\Var(\tau)$.
\end{enumerate}


\item Рассмотрим модель Блэка-Шоулза с параметрами $S_0$, $\mu$, $\sigma$, $r$.
  
Найдите цену актива $X_0$, если известно, что в момент времени $T=2$ актив выплачивает 
сумму равную $X_2 = S_1 + 6 \ln S_2$.


\end{enumerate}

\end{document}
